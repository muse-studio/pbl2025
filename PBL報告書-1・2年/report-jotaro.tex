% !TEX root = _main.tex
% ========================================
% 卒業論文 本文
% ========================================
\section{はじめに}
\section{はじめに}
現代の音楽制作において,Digital Audio Workstation (DAW) は中心的な役割を担っている.市販のDAWは多機能かつ高性能であるが,その一方で,必ずしも全てのユーザの特定のニーズやワークフローに合致しているとは限らない.また,機能が豊富であるために動作が重く,低スペックなPCでは快適な操作が困難な場合がある.さらに,ユーザが独自に機能を拡張したり,不要な機能を削除したりするカスタマイズの自由度は極めて低いのが現状である.

このような背景から,本研究では,ユーザが自身の制作スタイルに合わせて必要な機能を自ら取捨選択し,構築できる「DIY (Do It Yourself)」の思想を取り入れた新しいDAWのコンセプトとして「DAWIY」を提案する\cite{DAWIY-No.0}.DAWIYの目的は,柔軟なGUI,軽量な動作,そして何よりも高い拡張性を備えたクロスプラットフォームな音楽制作環境の実現である.この要件を満たすため,本研究ではWeb技術(HTML, JavaScript, TypeScript, CSS)を開発の基盤として採用した.

本稿では,DAWIYコンセプトの実現に向けた第一歩として,オープンソースソフトウェア(OSS)であるWeb DAW「WAM-Studio」を基盤とし,その機能拡張と改良を行った初期開発について報告する.

\section{関連研究}
\subsection{dawproject}
DAW間のプロジェクトファイルの互換性の低さは,音楽制作者にとって長年の課題であった.\texttt{dawproject} \cite{dawproject-github}は,この問題に対処するために提案された,特定のDAWに依存しない共通のデータ保存形式である.この形式に対応することにより,異なるDAW間でのプロジェクトデータの移行や共同作業が容易になることが期待される.

\subsection{Web Audio Modules (WAMs) と WAM-Studio}
Web Audio Modules (WAMs)\cite{WAM-paper-buffa-2022}は,WebベースのDAWにおいて,VST (Virtual Studio Technology) プラグインのように音源やエフェクト(FX)を追加するための共通規格である.

WAMsの技術を用いて開発されたOSSのWeb DAWがWAM-Studio\cite{WAM-studio-paper-2023}である.Webブラウザ上で動作するためクロスプラットフォーム性に優れる一方,現状では本格的な音楽制作を行うにはいくつかの課題が存在する.具体的には,MIDIシーケンスを視覚的に編集するためのピアノロール機能が存在しないこと,独自のファイル保存システムを採用しており外部のDAWとの連携が困難であること,そして最大の問題として,機能の追加や改良,削除が容易な構造になっていない点が挙げられる.

\section{提案手法}
本研究では,関連研究で述べたWAM-Studioの課題を克服し,DAWIYコンセプトの土台を構築することを目的とする.具体的には,WAM-Studioをベースに以下の3つの改良を行うことを目標とした.

\begin{enumerate}
    \item \textbf{基本的なシーケンス編集機能の実装:} 音楽制作の核となるMIDI編集機能として,ピアノロールを実装する.
    \item \textbf{dawproject形式への対応:} DAW間の連携を促進するため,\texttt{dawproject}形式でのプロジェクト入出力を可能にする.
    \item \textbf{拡張性の確保:} 機能の自由な追加・削除を可能にするアーキテクチャを模索し,そのプロトタイプを実装する.
\end{enumerate}

\section{実装}
提案手法で掲げた目標に基づき,WAM-Studioのフォークに対して以下の機能を実装した.開発には主にJavaScriptとTypeScriptを用いた.

\begin{figure}[htbp]
  \centering
  \includegraphics[width=0.9\linewidth]{../fig/pianoroll.png}
  \caption{実装したピアノロール}
  \label{fig:pianoroll}
\end{figure}

\subsection{ピアノロールの実装}
DAWとしての基本的な編集機能を提供するため,MIDIノートを視覚的に配置・編集できるピアノロール画面を実装した\textbf{図\ref{fig:pianoroll}}.具体的には,マウス操作によるノートの追加・削除,再生位置を示す再生バーの表示,ノートの長さの伸縮,複数ノートの範囲選択,および基本的なキーボードショートカットなどの機能を実現した.

\subsection{dawproject形式での入出力}
他のDAWとの互換性を確保するため,\texttt{dawproject}形式のプロジェクトファイルを読み込み,DAW内に展開するインポート機能を実装した.逆に,本DAW上で作成したシーケンス情報を\texttt{.dawproject}ファイルとして書き出すエクスポート機能も実装した.これにより,Studio OneやCubaseといった他の主要DAWとプロジェクトデータを交換する準備が整った.

\subsection{プラグイン機構のプロトタイピング}

\begin{figure}[htbp]
  \centering
  \includegraphics[width=0.9\linewidth]{../fig/Stochastic_Note_Generator.png}
  \caption{実装した確率的メロディ生成機能}
  \label{fig:Stochastic_Note_Generator}
\end{figure}

本研究の重要な目標である「機能の自由な追加・削除」の実現可能性を探るため,その第一弾として「確率的メロディ生成機能」をプラグインとして実装した\textbf{図\ref{fig:Stochastic_Note_Generator}}.これは,指定された音域や音価の範囲内で,確率的にノートを生成する機能である.

\begin{figure}[htbp]
  \centering
  \includegraphics[width=0.9\linewidth]{../fig/PackageManager.png}
  \caption{実装したパッケージマネージャーのプロトタイプ}
  \label{fig:PackageManager}
\end{figure}

さらに,この機能を管理するための簡易的なパッケージマネージャーのプロトタイプも実装した\textbf{図\ref{fig:PackageManager}}.これにより,将来的にはユーザがUIを通して,必要とする機能を任意にインストールまたはアンインストールできる環境の構築を目指す.

\section{考察}
本研究で実装した各機能について,その達成度と残された課題を考察する.

ピアノロールと\texttt{dawproject}入出力機能については,基本的な動作を実現し,DAWIYの基盤となりうる最低限のDAWとしての体裁を整えることができた.しかし,ピアノロールの操作性や対応するショートカットの種類など,改良の余地は多く残されている.

プラグインとして実装した確率的メロディ生成機能は,本プラットフォームの拡張性を示す一例として機能した.一方で,生成されるメロディは音楽理論的に単純なものに留まっており,今後はスケール(階調)の概念や和音の生成など,より高度な機能を実装する必要があると思われる.

本研究における最大の目標であった「機能の自由な追加・削除」については,サーバとの連携による動的なプラグインのインストール機構など,本格的な実装には至っておらず,コンセプトの検証段階に留まった.これは今後の最大の課題である.

\section{おわりに}
本研究では,ユーザが自由に機能をDIYできる音楽制作環境「DAWIY」の実現に向け,その基盤となるプラットフォームをWAM-Studioベースで開発した.成果として,ピアノロールによるシーケンス編集機能,\texttt{dawproject}形式によるデータ入出力機能,そしてプラグイン機構のプロトタイプを実装した.

今後の展望として,外部から機能を安全かつ自由に追加・削除できる,より洗練されたプラグイン環境の構築が最重要課題となる.これには,サーバサイドとの連携や,依存関係の解決を含む本格的なパッケージ管理システムの確立が不可欠である.本研究の成果が,今後のDAWIY開発の礎となることを期待する.