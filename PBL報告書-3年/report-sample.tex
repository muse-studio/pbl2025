% !TEX root = _main3nen.tex
% ========================================
% 卒業論文 本文
% ========================================
\section{はじめに}

ゲームは人間の感覚を用いて多様な楽しさを提供する娯楽である。視覚による映像表現や触覚による振動表現に加え、聴覚を通じた音の演出は、ゲームへの没入感を高める重要な要素であるとされている。近年のゲームは、これらの要素が高水準で統合されているものが多い。一方で、スマートフォン向けのソーシャルゲームでは、公共交通機関の利用時など、周囲への配慮から音を消した状態でプレイされる場面が多い。筆者自身も無音でゲームをプレイする機会が多く、その際にゲーム体験が単調に感じられることがあった。しかし、音を出さない状態でのゲームプレイが、実際にどの程度ゲーム体験に影響を及ぼしているのかについては、明確に検証されているとは言い難い。そこで本研究では、ゲームにおける音の重要性に着目し、シューティングゲームを用いて、音の有無がプレイヤーのゲーム体験に与える影響を調査することを目的とする。調査は、不特定多数の参加者が集まる福桔祭において実施した。

調査に適したゲームを選定するにあたり、RPG、FPS、ノベルゲーム、シューティングゲームの四種類を候補として検討した。これらの中から調査に用いるゲームを選ぶため、三つの基準を設定した。

一つ目の基準は作成難易度である。筆者はプログラミング経験が浅いため、制作に関する資料が豊富に存在し、比較的実装が容易なゲームジャンルを選択する必要があった。二つ目の基準はプレイ時間である。調査は福桔祭というイベント会場で実施するため、来場者に長時間のプレイを求めることは難しい。そのため、一回のプレイが二分から三分程度で完了することを条件とした。三つ目の基準は、プレイヤーが楽しめることである。本研究の目的は音の影響を調査することであるため、ゲーム自体が過度に複雑であったり、操作が難解であったりすることは避ける必要があった。

これらの基準に基づき、各ゲームジャンルを評価した。RPGおよびノベルゲームは、物語の進行に時間を要するため、福桔祭のような短時間での体験を前提とした調査には適さないと判断した。FPSについては、先行研究で既に取り扱われている点に加え、リロード操作などが初心者には難しい可能性があること、また作成難易度が比較的高いことから採用を見送った。

一方で、シューティングゲームは、ゲーム性が単純であり、制作に関する資料も豊富に存在する。敵を撃って倒し、敵の攻撃を回避するという分かりやすいルールで構成されているため、短時間でもプレイしやすく、来場者にとって理解しやすいと考えられた。以上の理由から、本研究ではシューティングゲームを用いて調査を行うこととした。




%2
\section{調査の準備}


%2.1
\subsection{ゲーム作成}

本研究では、11月1日に開催された福桔祭での調査実施を目的として、オリジナルのシューティングゲームを作成した。ゲーム作成にあたっては、文献\cite{AkichunPG}を参考に基本構造を実装した。また、プレイデータの自動記録機能や画面レイアウトなどの細部については、GitHub Copilot\cite{GitHubCopilotAnatano}を用いて調整を行った。
完成したゲームの試遊動画及び画面を、\figref{zu7}\modified{、}{および}\url{https://x.gd/wfByc}に示す。
\begin{figure}[tb]
	\centering
	\includegraphics[width=\hsize]{../fig/7.png}
	\caption{試遊画面}
	\label{zu7}
\end{figure}
ゲームのクリア条件は、スコアが5000点に到達すること、または三分間の制限時間内に被弾回数が四回以内で生存することとした。一方、被弾回数が五回に達した場合はゲームオーバーとした。

%2.2
\subsection{アンケート作成}

調査を実施するにあたり、プレイ後に回答するアンケートを作成した。アンケート設計に際しては、ゲームオーディオの役割と没入感への影響を評価した先行研究\cite{andersenAudioInfluenceGame2021}を参考にした。この研究は、調査手順や評価項目が本研究と類似しており、設問設計の指針として有用であった。

先行研究から得られた知見として、匿名性を過度に高めると正確な属性情報が得られにくいこと、設問数を増やしすぎると回答精度が低下すること、効果音がプレイヤー体験に大きな影響を与えることが示されている。これらを踏まえ、本研究では以下の設問を設定した。
<<<<<<< HEAD

・年齢

・日常的なゲームのプレイ頻度、または最も頻繁にプレイしていた時期のプレイ頻度

・効果音が弾を避ける際にどの程度役立ったか

・効果音が弾を撃つ際にどの程度役立ったか

・ゲームのプレイ難易度に関する評価
=======
\begin{enumerate}
	\item 年齢
	\item 日常的なゲームのプレイ頻度、または最も頻繁にプレイしていた時期のプレイ頻度
	\item 効果音が弾を避ける際にどの程度役立ったか
	\item 効果音が弾を撃つ際にどの程度役立ったか
	\item ゲームのプレイ難易度に関する評価
\end{enumerate}
>>>>>>> 990c3d1db372b03a0f873b707c7f62f771c06469

年齢およびゲームのプレイ頻度は、被験者の属性とゲーム経験を把握するために設定した。効果音に関する設問は、先行研究において効果音の影響が大きいとされていたため、本研究においてもその影響の程度を確認する目的で設定した。また、プレイ難易度に関する設問は、ゲームが難しすぎることによって音への注意が阻害される可能性を考慮し、補助的な指標として設けた。

%3
\section{調査}
\subsection{調査環境}

本調査は、2025年11月1日に開催された福桔祭において実施した。調査時間は開始時刻である10時から終了時刻である18時までとし、水夕会サークルの展示スペースを使用した。会場には不特定多数の来場者が行き交っており、周囲には話し声や他展示による音が存在する環境であった。

このような環境では音への集中が妨げられる可能性があるため、被験者にはヘッドホンを装着してもらうことで対応した。調査当日の様子を\figref{zu1}に示す。
\begin{figure}[tb]
	\centering
	\includegraphics[width=\hsize]{../fig/1.jpg}
	\caption{実際にプレイして貰っている様子}
	\label{zu1}
\end{figure}
%3.2
\subsection{被験者}
調査に参加した被験者は、合計34名であった。
被験者の年齢分布を\figref{zu2}に示す。\figref{zu2}では、赤色が20~24歳、緑色が15~19歳、橙色が10~14歳、青色が6~8歳、茶色が70~74歳、紫色が25~30歳を表している。ただし、70~74歳および25~30歳の割合はいずれも2.9\% と少数であったため、図中では省略した。

また、被験者のゲームプレイ時間の分布を\figref{zu3}に示す。30分以上1時間未満が8件、1時間以上2時間未満が9件、2時間以上3時間未満が5件、3時間以上4時間未満が7件、4時間以上が2件、30分未満が3件であった。

全体として、年齢層は10代後半から20代が中心であり、日常的にゲームをプレイする者から、普段あまりゲームを行わない者まで、幅広い層が含まれていた。
年齢層は主に 10 代後半から 20 代が中心であり、日常的にゲームをプレイする者から、普段あまりゲームを行わない者まで含まれていた。
\begin{figure}[tb]
	\centering
	\includegraphics[width=\hsize]{../fig/2.png}
	\caption{年齢}\label{zu2}
\end{figure}
\begin{figure}[tb]
	\centering
	\includegraphics[width=\hsize]{../fig/3.png}
	\caption{プレイ時間}\label{zu3}
\end{figure}


%3.3
\subsection{調査手順}
被験者には、同一のシューティングゲームをサウンドONおよびサウンドOFFの二条件でプレイしてもらった。

プレイ順序による影響を抑えるため、最初の30秒間は計測を行わず、操作に慣れる時間とした。その後、調査の目的として、ゲームにおいて音がどの程度影響を与えるかを調査していることを説明した上で、本実験を開始した。

最初の16名はサウンドOFFの状態でプレイした後、サウンドONの状態でプレイした。次の14名はこの順序を逆にした。以降は、両条件の順序を交互に入れ替えながら、調査終了時刻である18時まで実施した。なお、ゲームの終了状態は、ゲームクリアまたはゲームオーバーのいずれかとした。
%3.4
\subsection{取得データ}
各プレイにおいて、ゲームの結果(ゲームクリアまたはゲームオーバー)、プレイ時間、スコア、サウンド設定(ON/OFF)を自動的に記録した。

取得したデータはテキストファイル形式で保存し、後の分析に使用した。
34件分のプレイデータは、\url{https://x.gd/t5Qhh}に示す。


%3.5
\subsection{調査条件の統制と配慮}
ゲーム内容、操作方法、および難易度は、全ての被験者で共通とした。ただし、調査開始時のトラブルにより、サウンドONからプレイした被験者とサウンドOFFからプレイした被験者の人数に一名分の差が生じた。

また、福桔祭という環境上、周囲の騒音や被験者の集中度には個人差が存在する可能性がある。この点は、本研究における制約条件であり、今後の改善点である。


%4
\section{分析}
\subsection{分析の目的}

本研究の目的は、ゲームプレイにおいて音の有無がプレイヤーのゲーム体験に影響を与えるかを検証することである。

本章では、その検証を行うために、プレイ結果の指標としてスコアおよびプレイ時間に着目し、サウンドONおよびサウンドOFFの条件間に統計的に有意な差が存在するかを分析する。


%4.
\subsection{使用するデータと前処理}

分析には、福桔祭で実施した調査により得られた34名分のプレイデータを使用した。

各被験者は、サウンドONおよびサウンドOFFの条件でそれぞれ一回ずつプレイしており、同一被験者のデータを一組として対応づけた。記録漏れや異常値の有無を確認した結果、分析に支障をきたす問題は認められなかったため、全データを分析対象とした。


%4.
\subsection{統計的手法}
%4.3.1
\subsubsection{検定方法の選択理由}

本研究では、同一の被験者が異なる二条件でゲームをプレイしている。そのため、被験者間の個人差を考慮した比較が可能な、対応のあるt検定を用いた。

この手法を用いることで、プレイヤーごとの能力差や経験の違いといった影響を抑え、音の有無による影響のみを抽出して評価することが可能となる。

%4.2.1
\subsubsection{仮説設定}

本研究では、スコアおよびプレイ時間について、それぞれ以下の仮説を設定した。
まず、スコアに関する仮説を次のように設定した。


帰無仮説:サウンドONとOFFによって平均スコアに差はない。

対立仮説:サウンドONとOFFによって平均スコアに差がある。

次に、プレイ時間に関する仮説を次のように設定した。

帰無仮説:サウンドONとOFFによって平均プレイ時間に差はない。

対立仮説:サウンドONとOFFによって平均プレイ時間に差がある。

いずれの検定においても、有意水準は5\% とした。


%4.2.2
\subsubsection{計算手順}

対応のあるt検定では、各被験者についてサウンドON条件とサウンドOFF条件のスコアおよびプレイ時間の差を算出し、それらの差に基づいて検定を行った。

この手法により、被験者ごとの個人差の影響を抑えた上で、サウンド条件の違いがゲームプレイに与える影響を評価した。

以上の手順により計算を行った結果、\tabref{gurahu}が得られた。

\[t = \frac{\bar{d}}{\frac{s_d}{\sqrt{n}}}\]

\begin{table}[tb]
	\centering
<<<<<<< HEAD
	\caption{サウンドON/OFF条件におけるスコアおよびプレイ時間のt検定結果}
	\label{gurahu}
=======
	\caption{\commented{表タイトルつけること}}\label{gurahu}
>>>>>>> 990c3d1db372b03a0f873b707c7f62f771c06469
	\begin{tabular}{crr}
		\hline\hline
		& スコア & プレイ時間 \\\hline
		t値 & -1.437 & -0.633 \\
		p値 & 0.160 & 0.531 \\\hline
	\end{tabular}
\end{table}


\begin{figure*}[tb]
	\centering
	\begin{minipage}[t]{0.32\hsize}
		\centering
		\includegraphics[width=\hsize]{../fig/4.png}
		\subcaption{効果音が弾を避けるのに役立ったか?}\label{zu4}
	\end{minipage}
	\begin{minipage}[t]{0.32\hsize}
		\centering
		\includegraphics[width=\hsize]{../fig/5.png}
		\subcaption{効果音が弾を撃つのに役立ったか?}\label{zu5}
	\end{minipage}
	\begin{minipage}[t]{0.32\hsize}
		\centering
		\includegraphics[width=\hsize]{../fig/6.png}
		\subcaption{ゲーム難易度}\label{zu6}
	\end{minipage}
	\caption{参加者に対するアンケート結果}\label{anquete}
\end{figure*}
%5
\section{結果}

\tabref{gurahu}で得られた分析結果より、スコアについては有意水準5\%(p$<$0.05)を満たさなかったため、帰無仮説は棄却されなかった。

同様に、プレイ時間についても有意水準5\%(p$<$0.05)を満たさなかったため、帰無仮説は棄却されなかった。

以上より、本研究で得られたデータからは、サウンドONおよびサウンドOFFの違いが、スコアおよびプレイ時間に統計的に有意な影響を与えるとは言えない結果となった。

%6
\section{考察}
\subsection{アンケート結果}

<<<<<<< HEAD
\begin{figure}[tb]
\centering
	
	\includegraphics[width=\hsize]{../fig/4.png}
	\caption{効果音が弾を避けるのに役立ったか?}
	\label{zu4}
\end{figure}
\begin{figure}[tb]
\centering
	
	\includegraphics[width=\hsize]{../fig/5.png}
	\caption{効果音が弾を撃つのに役立ったか?}
	\label{zu5}
\end{figure}
効果音がゲームプレイに与える影響について、弾を避ける場面および弾を撃つ場面に分けて評価を行った。その結果を\figref{zu4}および\figref{zu5}に示す。どちらも数値が低いほど役立っていないことを示す。
=======

効果音が弾を避ける際に役立ったか、また弾を撃つ際に役立ったかについてのアンケート結果を、それぞれ\figref{zu4}および\figref{zu5}に示す。
>>>>>>> 990c3d1db372b03a0f873b707c7f62f771c06469

弾を避ける際に効果音が役立ったかについての平均値は4.8であり、弾を撃つ際に役立ったかについての平均値は4.9であった。いずれの項目においても、評価は尺度の中央付近に集中しており、効果音がゲームプレイに対して強く肯定的に作用したとは言い難い結果であった。

<<<<<<< HEAD
このことから、本研究で使用したシューティングゲームにおいては、効果音がプレイヤーの行動を大きく補助する要素として認識されていなかった可能性がある。一方で、評価が極端に低い値を示していないことから、効果音が全く無意味であったと結論づけることもできない。

また、ゲーム難易度に関するアンケート結果を\figref{zu6}に示す。横軸はゲーム難易度を表し、数値が低いほど難易度が低く、数値が高いほど難易度が高いことを示す。縦軸は被験者数を表している。ゲーム難易度の平均値は5.1であり、多くの被験者にとって難易度は概ね適切であったと推察される。このことから、ゲームが過度に難しかったために音の効果を感じ取れなかった可能性は低いと推察される。
\begin{figure}[tb]
\centering
	
	\includegraphics[width=\hsize]{../fig/6.png}
	\caption{ゲーム難易度}
	\label{zu6}
\end{figure}

=======
これらの結果から、弾を避ける際に効果音が役立ったと回答した被験者と、役立たなかったと回答した被験者の割合はおおむね同程度であった。平均値は4.8であった。同様に、弾を撃つ際に効果音が役立ったかについても、肯定的および否定的な回答はほぼ同数であり、平均値は4.9であった。

次に、ゲーム難易度に関するアンケート結果を\figref{zu6}に示す。横軸はゲーム難易度を表し、数値が低いほど難易度が低く、数値が高いほど難易度が高いことを示す。縦軸は被験者数を表している。ゲーム難易度の平均値は5.1であり、多くの被験者にとって難易度は概ね適切であったと推察される。
>>>>>>> 990c3d1db372b03a0f873b707c7f62f771c06469
%6.2
\subsection{ゲームにおける音の重要性}

t検定およびアンケート結果から、本研究においては、音の有無がゲーム体験に大きな影響を与えるとは言えない結果が得られた。

この結果に至った要因の一つとして、調査環境の影響が考えられる。福桔祭という開放的な空間では、ヘッドホンを使用していたとしても、周囲の視覚情報や人の動きにより集中が妨げられた可能性がある。この点は、先行研究とは異なる条件であり、結果に影響を及ぼした可能性がある。

また、被験者の年齢層が幅広く、高齢者や低年齢の被験者においては、身体的特徴によりヘッドホンの装着や音の認識が十分でなかった可能性も考えられる。さらに、10代から20代の被験者が多かったことから、日常的に無音でゲームをプレイする習慣が影響し、音の有無による差が感じにくかった可能性も否定できない。

%7
\section{今後の展望}

本研究では、ゲーム体験において音の有無が統計的に有意な影響を与えるとは言えない結果が得られた。しかし、本研究で使用したゲームにはBGMを導入しておらず、効果音のみを用いた構成であった。そのため、BGMがもたらす雰囲気の演出や没入感への影響については検証できていない。

今後の調査では、BGMを含めた音響要素を導入したゲームを用いるとともに、より集中しやすい環境で実験を実施することが重要であると考えられる。これにより、ゲームにおける音の役割を、より多角的に評価できると期待される。



\endinput
%4.3
\subsection{数式}\label{sec:Item}

%4.3.1
\subsubsection{本文中の数式}

本文中の数式は \|$| と \|$|, \|\(| と \|\)|, あるいは \|math| 環境のいず
れで囲んでもよい.

%4.3.2
\subsubsection{別組の数式}

別組数式(displayed math)については \|$$| と \|$$| は使用せずに,
\|\[| と \|\]| で囲むか,
\|displaymath|, \|equation|, \|eqnarray| のいずれかの環境を用いる.
これらは
%
\begin{equation}
	\Delta_l = \sum_{i=l|1}^L\delta_{pi}
\end{equation}
%
のように,センタリングではなく固定字下げで数式を出力し,
かつ背が高い数式による行送りの乱れを吸収する機能がある.

%4.3.3
\subsubsection{eqnarray環境}

互いに関連する別組の数式が2行以上連続して現れる場合には,
単に\|\[| と \|\]|,
あるいは \|\begin{equation}| と\|\end{equation}| で囲った数式を書き並べるのではなく,
\|\begin|\allowbreak\|{eqnarray}| と \|\end{eqnarray}| を使って,
等号(あるいは不等号)の位置で縦揃えを行なった方が読みやすい.


%4.3.4
\subsubsection{数式のフォント}

\LaTeX が標準的にサポートしているもの以外の特殊な数式用フォントは,
できるだけ使わないようにされたい.
どうしても使用しなければならない場合には,
その旨申し出て頂くとともに,
組版工程に深く関与して頂くこともあることに留意されたい.


\begin{figure}[tb]
	\setbox0\vbox{
		\hbox{\|\begin{figure}[tb]|}
				\hbox{\quad \|<|図本体の指定\|>|}
				\hbox{\|\caption{<|和文見出し\|>}|}
				\hbox{\|\label{| $\ldots$ \|}|}
				\hbox{\|\end{figure}|}
	}
	\centerline{\fbox{\box0}}
	\caption{1段幅の図}
	\label{fig:single}
	%\addcontentsline{lof}{figure}{へへへ}
\end{figure}


%4.6
\section{参考文献}


本文中で参考文献を参照する場合には\|\cite|を使用する.
参照されたラベルは自動的にソートされ,
\|[]|でそれぞれ区切られる.
%
\begin{musequote}
	文献 \|\cite{okumura, Goossens94}| は\LaTeX の総合的な解説書である.
\end{musequote}
%
と書くと;
%
\begin{musequote}
	文献\cite{okumura, Goossens94}は\LaTeX の総合的な解説書である.
\end{musequote}
%
が得られる.

%4.6.2
\subsubsection{参考文献リスト}
参考文献リストはZoteroを用いて任意のフォルダに収集し,BetterBibTeXを使って\|.bib|ファイルを生成する.
TeX文書内では,本文の一番最後に
\begin{musequote}
	\|\bibliographystyle{muselabunsrt} |\\
	\|\bibliography{muselab-sample} | \\
	※muselab-sample.bib のこと
\end{musequote}
を記入する.

なお,製版用のファイル群には\verb+.bib+ファイルではなく\verb+.bbl+ファイルが必要となる.
\verb+.bbl+ファイルは\verb+BiBTeX+を実行すると自動生成されるので,普段は気にしなくても良い.
ただし,文字コードなど,何らかの理由で表示が困難な場合は\verb+.bbl+ファイルを直接編集してもよい\footnote{BiBTeXを実行すると上書きされるので注意.}.



