% !TEX root = _main.tex
% ========================================
% 卒業論文 本文
% ========================================
\section{はじめに}

福知山公立大学情報学部では,4年次に,卒業研究(以下,卒研と呼ぶ)において,卒業論文もしくは卒業制作(以下,特に断りがない限りは両方を合わせて「卒論」と呼ぶ)を実施し,審査を経て,単位取得並びに卒業(学位取得)を目指す.

卒論の執筆にあたっては,おおむね秋ごろに学科から配布される執筆ガイドラインに沿って,卒研生が各自でWordあるいは\LaTeX  を用いて進めることが原則である.ただ,現実問題として,
\begin{enumerate}
	\item[(1)]ほぼ全ての卒研生にとって,十数ページ以上に及ぶ長大な文章を,簡潔かつ論理的に記述していくこと自体が初めての経験となる上に,
	\item[(2)]完成版にはある程度の見栄えの良さも求められるため,執筆時点でWordや\LaTeX の各種機能についてもそれなりに通じていなければならない.
\end{enumerate}
個人の能力・努力だけで(1)も(2)もこなすのは大変なハードワークである.効率的に執筆作業を手助けするための執筆の参考となる見本書の整備が求められる.

本稿では,主に本研究室の在学生を対象として,\LaTeX のフォーマットを整えつつ論文を執筆するための基本的な手順を解説する.以下,第2章では,卒業論文の執筆にかかる基本的な手順について説明する.第\ref{config}章では,\LaTeX 文書の視覚的な体裁を整える方法について示す.第\ref{sec:contents}章では,論文執筆,ひいては卒業研究実施についての心構えについて述べる.



%2
\section{執筆の流れ}


%2.1
\subsection{準備}

本学部卒論文の\LaTeX スタイルファイルを含む論文執筆キットは
\begin{quote}
	\small
	\|https://scrapbox.io/muselab/卒業論文(卒論)の書き方|
\end{quote}
から入手する.論文執筆キットは以下のファイルを含んでいる.
\begin{enumerate}
	\item \|ipsj.cls   |: 原稿用クラスファイルのベース
	\item \|ipsjtech.sty|: 論文スタイルファイルのベース
	\item \|muselabtech.sty|: 橋田ゼミの論文スタイルファイル
	\item \|muselab-correction.sty|: 添削用スタイルファイル
	\item \|muselab-sample.tex |: 本稿のソースファイル
	\item \|muselabunsrt.bst |: jBibTEX スタイル(出現順)
	\item \|muselab-sample.bib |: 文献リストのサンプル
\end{enumerate}
実行環境としては\LaTeXe を前提としている.TeXWorksやVS Codeなどで準備すること.



上記に含まれるスタイルファイルは,
極力特別なコマンドは使わずに,標準的な\LaTeX のスタイルを踏襲している.
論文フォーマットに関しては,第\ref{config}章で後述する指針に従ってもらうが,
そこに規定されていること以外は標準的な\LaTeX のコマンドをそのまま使うことができる.
本稿は,そのスタイルファイルを実際に使っているので,論文執筆の際に参考にしてほしい.



%2.2
\subsection{原稿の作成と投稿}

本稿に従って用意した投稿用原稿の\LaTeX ソースからpdfファイルを作成し,
Adobeのpdf readerで読めることを確認した後,
毎年秋ごろに大学ポータルから案内される論文投稿フォームを用いて,指示に従い投稿する.




%4
\section{論文の構成}
\label{config}
この章では,\LaTeX でのそれぞれの文章体裁を中心とした「見た目」の書き方について解説する.
\LaTeX を用いた一般的な文章作成技術については,
文献\cite{okumura, Goossens94} 等を参考にすること.


%4.1
\subsection{表題・著者名等}

表題,著者名とその所属,および概要を前述のコマンドや環境により{\bf 和文と
英文の双方について}定義した後,\|\maketitle| によって出力する.


%4.1.1
\subsubsection{表題}

表題は,\|\title| および \|\etitle| で定義する.コンパイルすると\ruby{中央寄せ}{センタリング}になる.
文字数の多いものについては,適宜 \|\\| を挿入して改行する.

%4.1.2
\subsubsection{著者名・所属}

著者名は \|\author| で定義する.学籍番号と氏名,英文氏名,
所属ラベル(\|FUKU|)とメールアドレスを記入する.
メールアドレス部分は省略してもよい.

%4.2
\subsection{本文}

%4.2.1
\subsubsection{見出し}

節や小節の見出しには \verb|\section|, \verb|\subsection|, \verb|\subsubsection|,
\verb|\paragraph| といったコマンドを使用する.

「定義」,「定理」などについては,\|\newtheorem|で適宜環境を宣言し,そ
の環境を用いて記述する.

%4.2.2
\subsubsection{行送り}

2段組を採用しており,左右の段で行の基準線の位置が一致することを原則としている.
また,節見出しなど,
行の間隔を他よりたくさんとった方が読みやすい場所では,
この原則を守るようにスタイルファイルが自動的にスペースを挿入する.
したがって本文中では \|\vspace| や \|\vskip| を用いたスペースの調整を行なわないようにすること.


%4.2.3
\subsubsection{フォントサイズ}

フォントサイズは,スタイルファイルによって自動的に設定されるため,
基本的には著者が自分でフォントサイズを変更する必要はない.

%4.2.4
\subsubsection{句読点}

句点には全角の「.」,
読点には全角の「,」を用いる.
ただし英文中や数式中で「.」や「,」を使う場合には,
半角文字を使う.「.」や「,」は使わない.



%4.2.5
\subsubsection{全角文字と半角文字}

全角文字と半角文字の両方にある文字は次のように使い分ける.

\begin{enumerate}
	\item 括弧は全角の「(」と「)」を用いる.但し,英文の概要,図表見出し,
	      書誌データでは半角の「(」と「)」を用いる.

	\item 英数字,空白,記号類は半角文字を用いる.ただし,句読点に関しては,
	      前項で述べたような例外がある.

	\item カタカナは全角文字を用いる.

	\item 引用符では開きと閉じを区別する.
	      開きには \|``| を用い,閉じには\|''| を用いる.
\end{enumerate}

%4.2.6
\subsubsection{箇条書き}

箇条書きに関する形式を特に定めていない.場合に応じて標準的な \|enumerate|,
\|itemize|, \|description| の環境を用いてよい.


%4.2.7
\subsubsection{脚注}

脚注は \|\footnote| コマンドを使って書くと,
ページ単位に\footnote{脚注の例.}や\footnote{二つめの脚注.}のような参照記号とともに脚注が生成される.
なお,ページ内に複数の脚注がある場合,参照記号は\LaTeX を2回実行しないと正しくならないことに注意されたい.



また場合によっては,
脚注をつけた位置と脚注本体とを別の段に置く方がよいこともある.
この場合には,\|\footnotemark| コマンドや \|\footnotetext| コマンドを使って対処していただきたい.


なお,脚注番号は論文内で通し番号で出力される.




%4.2.8
\subsubsection{OverfullとUnderfull}

組版時にはoverfullを起こさないことを原則としている.
従って,まず提出するソースが著者の環境でoverfullを起こさないように,
文章を工夫するなどの最善を尽くしてほしい.
ただし,\|flushleft| 環境,\|\\|,\|\linebreak| などによる両端揃えをしない形でのoverfullの回避は,
できるだけ避けていただきたい.
また著者の執筆時点では発生しないoverfullが,
組版時の環境では発生することもある.
このような事態をできるだけ回避するために,
文中の長い数式や \|\verb| を避ける,
パラグラフの先頭付近では長い英単語を使用しない,
などの注意を払うようにして頂きたい.




%4.3
\subsection{数式}\label{sec:Item}

%4.3.1
\subsubsection{本文中の数式}

本文中の数式は \|$| と \|$|, \|\(| と \|\)|, あるいは \|math| 環境のいず
れで囲んでもよい.

%4.3.2
\subsubsection{別組の数式}

別組数式(displayed math)については \|$$| と \|$$| は使用せずに,
\|\[| と \|\]| で囲むか,
\|displaymath|, \|equation|, \|eqnarray| のいずれかの環境を用いる.
これらは
%
\begin{equation}
	\Delta_l = \sum_{i=l|1}^L\delta_{pi}
\end{equation}
%
のように,センタリングではなく固定字下げで数式を出力し,
かつ背が高い数式による行送りの乱れを吸収する機能がある.

%4.3.3
\subsubsection{eqnarray環境}

互いに関連する別組の数式が2行以上連続して現れる場合には,
単に\|\[| と \|\]|,
あるいは \|\begin{equation}| と\|\end{equation}| で囲った数式を書き並べるのではなく,
\|\begin|\allowbreak\|{eqnarray}| と \|\end{eqnarray}| を使って,
等号(あるいは不等号)の位置で縦揃えを行なった方が読みやすい.


%4.3.4
\subsubsection{数式のフォント}

\LaTeX が標準的にサポートしているもの以外の特殊な数式用フォントは,
できるだけ使わないようにされたい.
どうしても使用しなければならない場合には,
その旨申し出て頂くとともに,
組版工程に深く関与して頂くこともあることに留意されたい.


\begin{figure}[tb]
	\setbox0\vbox{
		\hbox{\|\begin{figure}[tb]|}
				\hbox{\quad \|<|図本体の指定\|>|}
				\hbox{\|\caption{<|和文見出し\|>}|}
				\hbox{\|\label{| $\ldots$ \|}|}
				\hbox{\|\end{figure}|}
	}
	\centerline{\fbox{\box0}}
	\caption{1段幅の図}
	\label{fig:single}
	%\addcontentsline{lof}{figure}{へへへ}
\end{figure}



%4.4
\subsection{図}

1段の幅におさまる図は,
\figref{fig:single} の形式で指定する.
位置の指定に \|h| は使わない.
また,図の下に和文と英文の双方の見出しを
\|\caption| で指定する.
文字数が多い見出しは自動的に改行して最大幅の行を基準にセンタリングするが,
見出しが2行になる場合には適宜 \|\\| を挿入して改行したほうが良い結果となることがしばしばある
(\figref{fig:single} の英文見出しを参照).
図の参照は \|\figref{<|ラベル\|>}| を用いて行う.

\begin{figure}[tb]
	\begin{minipage}[t]{0.5\columnwidth}
		\footnotesize
		\setbox0\vbox{
			\hbox{\|\begin{minipage}[t]%|}
					\hbox{\|  {0.5\columnwidth}|}
					\hbox{\|\CaptionType{table}|}
					\hbox{\|\caption{| \ldots \|}|}
					\hbox{\|\label{| \ldots \|}|}
					\hbox{\|\makebox[\textwidth][c]{%|}
						\hbox{\|\begin{tabular}[t]{lcr}|}
								\hbox{\|\hline\hline|}
								\hbox{\|left & center & right \\\hline|}
								\hbox{\|L1   & C1     & R1    \\|}
								\hbox{\|L2   & C2     & R2    \\\hline|}
								\hbox{\|\end{tabular}}|}
					\hbox{\|\end{minipage}|}}
		\hbox{}
		\centerline{\fbox{\box0}}
		\caption{\protect\tabref*{tab:right} の中身}
		\label{fig:left}
	\end{minipage}%
	\begin{minipage}[t]{0.5\columnwidth}
		\CaptionType{table}
		\caption{\protect\figref*{fig:left} で作成した表}
		\label{tab:right}
		\vskip1mm
		\makebox[\textwidth][c]{\begin{tabular}[t]{lcr}\hline\hline
				left & center & right \\\hline
				L1   & C1     & R1    \\
				L2   & C2     & R2    \\\hline
			\end{tabular}}
	\end{minipage}
\end{figure}

\begin{figure*}[tb]
	\setbox0\vbox{\large
		\hbox{\|\begin{figure*}[t]|}
				\hbox{\quad \|<|図本体の指定\|>|}
				\hbox{\|\caption{<|和文見出し\|>}|}
				\hbox{\|\label{| $\ldots$ \|}|}
				\hbox{\|\end{figure*}|}}
	\centerline{\fbox{\hbox to.9\textwidth{\hss\box0\hss}}}
	\caption{2段幅の図}
	\label{fig:double}
\end{figure*}


また紙面スペースの節約のために,
1つの \|figure|(または \|table|)環境の中に複数の図表を並べて表示したい場合には,
\figref{fig:left} と \tabref{tab:right} のように個々の図表と各々の \|\caption|
を \|minipage| 環境に入れることで実現できる.
なお図と表が混在する場合,
\|minipage| 環境の中で\|\CaptionType{figure}| あるいは \|\CaptionType|
\|{table}| を指定すれば,
外側の環境が \|figure| であっても \|table| であっても指定された見出しが得られる.



2段の幅にまたがる図は,
\figref{fig:double} の形式で指定する.
位置の指定は \|t| しか使えない.



図の中身では本文と違い,
どのような大きさのフォントを使用しても構わない(\figref{fig:double} 参照).
また図の中身として,encapsulate されたPostScriptファイル(いわゆるEPSファイル)を読み込むこともできる.
読み込みのためには,プリアンブルで
%
\begin{quote}
	\|\usepackage{graphicx}|
\end{quote}
%
を行った上で,
\|\includegraphics| コマンドを図を埋め込む箇所に置き,
その引数にファイル名(など)を指定する.




%4.5
\subsection{表}

表の罫線はなるべく少なくするのが,仕上がりをすっきりさせるコツである.
罫線をつける場合には,
一番上の罫線には二重線を使い,左右の端には縦の罫線をつけない (\tabref{tab:example}).
表中のフォントサイズのデフォルトは\|\footnotesize|である.


また,表の上に和文と英文の双方の見出しを
\|\caption| で指定する.
表の参照は \|\tabref{<|ラベル\|>}| を用いて行なう.

\begin{table}[tb]
	\caption{表の例}
	\label{tab:example}
	\hbox to\hsize{\hfil
		\begin{tabular}{l|lll}\hline\hline
			     & column1  & column2  & column3  \\\hline
			row1 & item 1,1 & item 2,1 & ---      \\
			row2 & ---      & item 2,2 & item 3,2 \\
			row3 & item 1,3 & item 2,3 & item 3,3 \\
			row4 & item 1,4 & item 2,4 & item 3,4 \\\hline
		\end{tabular}\hfil}
\end{table}




%4.6
\subsection{参考文献・謝辞}

%4.6.1
\subsubsection{参考文献の参照}

本文中で参考文献を参照する場合には\|\cite|を使用する.
参照されたラベルは自動的にソートされ,
\|[]|でそれぞれ区切られる.
%
\begin{musequote}
	文献 \|\cite{okumura, Goossens94}| は\LaTeX の総合的な解説書である.
\end{musequote}
%
と書くと;
%
\begin{musequote}
	文献\cite{okumura, Goossens94}は\LaTeX の総合的な解説書である.
\end{musequote}
%
が得られる.

%4.6.2
\subsubsection{参考文献リスト}
参考文献リストはZoteroを用いて任意のフォルダに収集し,BetterBibTeXを使って\|.bib|ファイルを生成する.
TeX文書内では,本文の一番最後に
\begin{musequote}
	\|\bibliographystyle{muselabunsrt} |\\
	\|\bibliography{muselab-sample} | \\
	※muselab-sample.bib のこと
\end{musequote}
を記入する.

なお,製版用のファイル群には\verb+.bib+ファイルではなく\verb+.bbl+ファイルが必要となる.
\verb+.bbl+ファイルは\verb+BiBTeX+を実行すると自動生成されるので,普段は気にしなくても良い.
ただし,文字コードなど,何らかの理由で表示が困難な場合は\verb+.bbl+ファイルを直接編集してもよい\footnote{BiBTeXを実行すると上書きされるので注意.}.


%4.6.3
\subsubsection{謝辞}

謝辞がある場合には,
参考文献リストの直前に置き,
\|acknowledgment|環境の中に入れる.



%5
\section{論文内容に関する指針}
\label{sec:contents}
ここからは内容のお話.
\subsection{卒論?研究?}
そもそも,卒論だの研究だのとは,いったい何をするものなのか.4年生になるまで,あるいは研究室・ゼミに配属されるまでの間,それとなく話には聞くものの,いざ自分がそれをする立場になったら,あらためて「あれ?」と思い至ることであろう.

卒業論文とは,大学学部に在籍する学生が,「学士」という学位(=大卒という学歴)を得るための最大関門となる「卒業研究」の一貫で,学術の作法に\ruby{則}{のっと}って執筆する文章作品である\footnote{大学によっては,卒業論文を書かない部門もある.その代わり,芸術系学部によく見られる「卒業演奏会」や「作品制作」展等での発表,何らかの形での「卒業試験」が課されている.}.

卒業研究は,大学を拠点として行われる数ある学術研究の一つに属する.大学において学術研究を行うのは(1) 教員(教授,准教授,講師,助教),(2) 学生(大学院生と一部の学部生),(3) 研究のために特別に雇用された専門研究員である.学術研究には,その規模の違いこそあれ,必ず(1)〜(3)の誰かが「主たる遂行者」---要はその研究に一番深く関わり,直接的に実施する人となる.またの名を,その研究のプロジェクトリーダー,最高責任者などと呼ぶ.研究の成果を示す代表的なものが学術論文であり,その研究の全て,あるいは一部分について著述される.研究の規模が大きくなれば,その研究に関する論文は複数作成しうる.各論文においては,その著述内容に関することを最も直接的に行なった人物が“\ruby{第一著者}{ファースト・オーサー}”となる.「卒業」を冠している卒業研究の場合,主たる遂行者は例外なく学部4年生の本人であり,もちろん卒業論文の第一著者も本人,そして単著(論文の著者が本人だけ)として執筆される.

\subsection{何を書くの?}


卒論を含めた全ての学術論文に共通で課せられている仕事は,著者が研究を通じて得た,なんらかの発見・知見・知識を「\textbf{ことば(文章)によって説明する}」ことである.

\begin{musequote}
	はじめにことば(ロゴス)ありき.ことばは神と共にあり,ことばは神であった.
	\from{ヨハネによる福音書1章1節}
\end{musequote}

\begin{description}
	\item[【Why】] \textbf{なぜこの研究をするのか}.
	      当該分野における背景や,その文脈での本研究の位置付け,その研究を実施する意義・必要性などを説明する.(背景)
	\item[【What & How to do】] \textbf{何ができるようになりたくて/知りたくて,そのために何をするのか}.
	      本研究が目指すところ(目標),達成しようとすること(目的),そのために具体的に行うこと(手段)を述べる.
	\item[【How was it?】] \textbf{「やってみた」らどうなった/どうだったのか.}
	      調査や実験を通じて得られた「結果」がどういうものか,その結果から著者が判断・解釈したことは何か(考察),当初の目的がどのように達成されたか(あるいはされなかったか)(結論)を書く.
\end{description}

\subsection{どこに書くの?}
これらについて,論文ではそれぞれ「簡潔に記述する」箇所と「詳細に記述する」箇所がある.「詳細に」書くのはもちろん論文本体である.一方の「簡潔に」書くのは,「要旨(ようし)(abstract(アブストラクト))」と呼ばれる,論文の全体像を,200〜1,000字程度(A4版1ページ以内相当)に圧縮した文章である.ここでいう圧縮とは,言うなれば「時短」ならぬ「字短」である.内容の骨格はそのままに,文字数を可能な限り減らすものである.

詳細文(本文):
\begin{musequote}
	私は(前作のポピンズを演じた)ジュリー・アンドリュースのようには歌えないし,彼女の声も持っていない.だから,2人の歌の先生に指導してもらって,メリー・ポピンズの個性あふれる声を作り上げた.
	\from{文献\cite{natalie} より,筆者一部改変}
\end{musequote}

簡潔文1:
\begin{musequote}
	私はジュリー・アンドリュースのようには歌えないので,2人の歌の先生に指導を受けた.
\end{musequote}
簡潔文2:
\begin{musequote}
	私は2人の歌の先生に指導を受け,メリー・ポピンズの個性あふれる声を作り上げた.
\end{musequote}

論文で書くべき項目とその記述場所について,詳細・簡潔に書くとしたらどこに書くべきか.その典型パターンを\tabref{tab:ronbunmap}に示す.
%なお,これまでに提出されてきた卒業生6期分の卒業論文は全て本町合同研究室に保管されてあり,随時閲覧可能である.


\begin{table*}[tb]
	\centering
	\caption{論文で書くべき項目と記述場所}
	\includegraphics[width=\hsize]{../fig/ronbunmap.pdf}
	\label{tab:ronbunmap}
\end{table*}

目標,目的,手段の違いについては様々な例えがあるが,大学生に最も身近な例を挙げると以下のようになる.
\begin{description}
	\item[目標:] 将来的に叶えたいこと.実際(今回の卒論)に関しては実現できない可能性があって良い.
	      「大学を卒業する」
	\item[目的:] 目標を実現するために課したミッション.達成することを基本とする.
	      「卒業論文を提出する」
	\item[手段:] 目的のために直接的・間接的に筆者が実施すること.たいていの場合複数あり,「研究課題」として自身でその要点を整理することになる.
\end{description}

\begin{verse}
	課題1「○○に関する調査・実験を行う」\\
	「研究テーマを決める」\\
	※研究題目を決めることと同義\\
	「実験・調査の計画を立てる」\\
	「実験・調査を実施する」\\
	「実験・調査結果を整理する」\\

	課題2「卒業論文を執筆する」\\
	「研究題目(論文タイトル)を決める」\\
	「論文の構成を組み立てる」\\
	「論文各章を執筆する」\\
	「卒業論文を印刷・製本する」\\
	「卒業論文を事務室に提出する」
\end{verse}

\section{章別のヒント}
\subsection{はじめに/序論}

卒業論文の規模で言えば,第1章は,下記の項目に対応する6つの文を書ければ9割終了する.

分野のツカミ:研究テーマの話をするにあたっての入口

(1)の中で起きているもう少し詳細な状況説明

(2)の文脈において,何かしら起こっている問題,疑問,議論,不都合

(3)がなぜダメなことなのか,(3)に対して著者はどうしたい(世の中を変えたい?)のか

(4)そこで本研究(著者)は何をするのか.

(5)以降,この論文の中で具体的に何の話を書いていくのか(各章の説明)

この例文を,まさに本稿の第1章にて示している.

\vspace{5pt}
\begin{verse}
	\small{
	[a] 福知山公立大学情報学部では,4年次に,卒業研究(以下,卒研と呼ぶ)において,卒業論文もしくは卒業制作(以下,特に断りがない限りは両方を合わせて「卒論」と呼ぶ)を実施し,審査を経て,単位取得並びに卒業(学位取得)を目指す.

	[b] 卒論の執筆にあたっては,おおむね秋ごろに学科から配布される執筆ガイドラインに沿って,卒研生が各自でWordあるいは\LaTeX  を用いて進めることが原則である.

	[c]ただ,現実問題として,(1)ほぼ全ての卒研生にとって,十数ページ以上に及ぶ長大な文章を,簡潔かつ論理的に記述していくこと自体が初めての経験となる上に,(2)完成版にはある程度の見栄えの良さも求められるため,Wordの各種機能についてもそれなりに通じなければならない.

	[d]個人の能力・努力だけで(1)も(2)もこなすのは大変なハードワークである.効率的に執筆作業を手助けするための執筆の参考となる見本書の整備が求められる.

	[e]本稿では,主に本研究室の在学生を対象として,\LaTeX のフォーマットを整えつつ論文を執筆するための基本的な手順を解説する.

	[f]以下,第2章では,卒業論文の全体的な構成と最小限の文章例について解説する.第3章では,Wordのスタイル機能を用いて文書の視覚的な体裁を整える方法について示す.第4章では,論文執筆のための卒業研究実施についての心構えについて述べる.
	}
\end{verse}
\vspace{5pt}

[a]〜[f]の全6文で構成されており,それぞれ下記の役割をもつ.
\begin{itemize}
	\item[[a]] 研究背景の入口その1.どこの世界で何が行われているかの前振りを示している.
	\item[[b]]  研究背景の入口その2.[a]で述べた「卒業をめざす」ために具体的に行われることや基本ルールについて紹介している.
	\item[[c]] 「ただ」によって,[b]で何か現実的な問題が起こっていることを指摘している.ここでは(1)(2)と箇条書き型で二つ挙げているが,一文の字数としてはやや長い(1文が3行以上になってきたら注意が必要).一文をできるだけ短くするなら,下記のように2文に分けた方が好ましい.
	\item ...
	      \begin{musequote}
		      しかし実際には,ほぼ全ての卒研生にとって,(1)十数ページ以上に及ぶ長大な文章を,簡潔かつ論理的に記述していくこと自体が初めての経験となる.さらに,(2)完成版にはある程度の見栄えの良さも求められるため,Wordの各種機能についてもそれなりに通じなければならない.
	      \end{musequote}
\end{itemize}

\subsection{先行研究・関連研究}
上級生から引き継いだ研究テーマであったり,研究上の思想や技術基盤の部分で自分の研究に直結してくる既存研究を「先行研究」と呼ぶ.
そこまでではないにせよ,自分の研究遂行にあたって一部でも参考になった既存研究や書籍,記事などは「関連研究」として,随時本文中において引用し,
自分の論理立てに役立てていく.
特に,似たような発想,技術アプローチ,求める出力結果が出ている物については必ず取り挙げて,「自分の研究はそれらとどこが違うか」を明確にする.
(執筆が本格化したら,口頭ででも補足する.)

ひたすらに引用する必要があるので,参考文献は卒論クラスなら軽く20件は超える.\ref{sec:refcheck}節も参照のこと.

\subsection{設計}
ゼミ等にて随時説明.
\subsection{実装}
ゼミ等にて随時説明.
\subsection{分析・評価}
ゼミ等にて随時説明.
\subsection{考察}
ゼミ等にて随時説明.
\subsection{結論}
結論の章は,本論で述べてきたことを総まとめ=おさらいするためにある.はっきり言って,それ以外の事柄は書かなくて良い.というか書くな.

\vspace{5pt}
\begin{quote}
	\footnotesize{
		(1)本研究は,〜〜〜ことを目的に,〜〜を行ってきた.
		(2)そのために〜〜という仮説を立て,第2章で****について述べた.
		(3)それを踏まえて,第3章では具体的に〜〜〜〜という調査(または実験)を行った.
		(4)その結果は〜〜〜で,〜〜〜〜なことがわかった(第4章),
		(5)その上で,第5章で〜〜〜について考察した.
	}
\end{quote}
\vspace{5pt}


上記のように順番に淡々と書き連ねていく.ここで,\underline{前章までに記述していないことは書いてはいけない}.最後の最後で初めてするような話(主に,結果や考察,著者の意見など)は,

とってつけたような余計なネタか,

苦労して組み立てたそれまでの論の流れをぶち壊すか\\
のどちらかでしかない.結論のつもりでどうしても書きたいことは,前章までにそれを論じる適切な箇所を設け,そこに組み込み直すのが正しい.

ただし,きちんとおさらいできたことが前提で,その上で最終章にのみ付け足しで書いても良い例外が2つある.今回の研究で扱わなかった「今後の〔課題〕と〔展望〕」である.



\begin{musequote}
	本論文の調査は,◯◯を調べるという目的で行ったものであるが,この調査方法を用いると,□□□の分析にも適用できると思われる.これについては今後の課題とする.
\end{musequote}
\begin{musequote}
	今回〜〜〜を調査してみて,〜〜〜の分野では今後こんなことが主流になっていくであろうことが予想される.筆者の思いとしては,今後とも今回〜〜〜に関する調査を続けつつ,***の動向について注視していきたい.
\end{musequote}
なお,自分の感想を述べる祭,単に「楽しかった,難しかった」は,語彙が少なすぎて小学生低学年の感想文レベルであることは肝に銘じる.また,「自身の知識不足を痛感した」系の文面もよく見られるが,それは基本的に年齢関係なく“当たり前”である(だから学び,研究するのだ)から記述不要である.著者個人の能力云々は,学術の観点ではどうでもよいことであり,読者の関心は,徹底的に,研究題目となったテーマにある.それでもどうしても書きたければ,次項の「謝辞」に回すこと.



\subsection{時制と章立ての関係}
序章から結論までの内容は,

(1)これまでの研究の流れを整理する「過去形」の文章と,

(2)今回計画し,実施した「現在形→過去形」の文章\\
で成り立っている.つまり,\emph{ほぼ全章にわたって,「未来」または「現在形→未来形」につながる文章を書く余地はどこにもない}.こと学術の世界において,未来の話は,過去と現在の状況を確認した上で初めて語れるようになるものだからである.そこで,一通りのおさらいが終わった最後の最後に,研究をやってみてあらためて題目の問いについて「いま」思うことや,今回扱わなかった領域に関する派生課題,今後の予想などに触れておく.そうすることで,将来,著者自身あるいはこの論文を読んだ人が,この研究を未来につなげていくことができるようになる.文例を二つ挙げておく.



\subsection{謝辞}
謝辞は,本論執筆を含めて本研究を実施するにあたり,お世話になった方々に謝意を示すものである.なんだかんだと,研究・論文執筆はまったく一人で完遂することはほぼ不可能である.また,卒業論文は,一般に公開されるべき学術資料であり,一種の出版物でもある.人によっては一生に一度の「名前の残る著作」にもなりうる\cite{Shirai13}.したがって,何らかの形で助言やサポートをもらった人物や団体をここに挙げて,まとめてお礼を書き記しておくことは彼ら協力者への礼儀として望ましい.ただし,謝辞そのものは研究自体には直接の関係はなく,論文審査の対象にもならない.なので,どのように書くか(全く書かないことも含めて)は完全に著者個人の自由に任されている.その上で,書きたいと思う場合は,以下を読み進めてほしい.

謝辞で取り挙げる人物や団体の代表例としては,指導教員やゼミメンバー,実験等に関わった外部協力者らである.直接的に研究内容と関連が高く,かつ目上の人物から,というのが通例だとも言われる\cite{Tap-biz17}が,絶対というわけではない.自身で「ああこの人には本当にお世話になった」と,自然に頭の下がる思いをした人から順番に挙げていくのでも良い.ただし,あまりにプライベートすぎる人物については,もしかして十数年後に読み返すようなことがあった時のことを多少意識しておくことを勧める(止めはしないが,統計的に,今現在の彼氏彼女と十数年後も付き合いがある人はそう多くないであろう…).

\begin{musequote}
	本研究の実施にあたり,どこそこの誰々に**に関する多大な助言をいただいた.**の実験において,**の某店に実験環境を提供していただいた.◯◯氏には***の面でサポートをいただいた.ここに感謝の意を表する.
\end{musequote}

なお,論文全体の中でも謝辞は一番すんなりと書きやすい項目でもある.「とりあえず文書ファイルを用意して,執筆へのモチベーションをアップさせる」効果もあるので,何なら真っ先に書くことを勧めるサイトもある\cite{Nakata09}.


\subsection{参考文献のチェックリスト}
\label{sec:refcheck}

\begin{itemize}
	\item[$\Box$] 参考文献は10件以上必要(分野によっては20件以上,30件以上
	      という意見もある).
	\item[$\Box$] 十分な参考文献は新規性の主張に欠かせない.
	\item[$\Box$] 適切な文献が引用されておらず,その数も適切ではないのは再
	      考を要する.
	\item[$\Box$] 日本人によるしかるべき論文を引用することで日本人研究コミュ
	      ニティの発展につながる.
	\item[$\Box$] 参考文献は自分のものばかりではだめ.
\end{itemize}



\section{読み合わせと添削}
ある程度論文が書けてきたら,まずはゼミ内の学生(学年関係なし)同士で読み合わせを行うと良い.
自分では気づかない細かい文言や,話の組み立て順,図表や文献番号の参照などについてのフィードバックが得られる.

\subsection{自分の論文を読み返す}
\emph{すべての基本}.目で見て確認することはもちろんであるが,ある程度まとまった文章が書けてきたら,当該部分を\emph{普段の喋る速度で音読する}ことを強く勧める.
音に出して読んでみると,誤字脱字を発見したり,「書いた自分が理解できない」「なんだか言葉がつっかかる」「どうしても読み間違いをする」といった,自分自身の言葉癖とは明らかに異なっている箇所を見つけやすくなる.そういった箇所を洗い出し,文章を何度も練り上げていく.

また,完成版の提出に至る最後の過程として,完成版を再度印刷し,再び音読して最後確認を行う.この時は,深呼吸しながら,普段よりさらにゆっくり読み上げる.
ここに至って,今まで漏れてきた誤字脱字,直前までの修正によって意図せず表示崩れが起こっていたりしないかを確認する.

\subsection{自分の論文を読んでもらう}
見て欲しい部分を印刷し,紙面で読んでもらうことを推奨する.
PDFとして画面で見てもらうのも良いが,できれば紙メディアの方が,気になったところを印付けしやすい.

渡す際に「お願いします」,フィードバックが返ってきたら「ありがとうございます」を必ず伝えること.
また,相手が口頭で説明する必要があるようであれば,優先的に聞きに行くこと.

もらったフィードバックを読み,明らかに修正が必要なものについては確実に反映させる.
一方,すべての指摘をそのまま鵜呑みにしすぎないように注意する.
自分の書き方が悪かったために,相手が内容を誤解してしまったという可能性も十分にある.
フィードバックは「自分の書いた内容が正しく伝わっているか」を確認するためのもので,「無条件に従えば良い」ものではないことに注意しよう.

\subsection{他の人の論文を読む}
自分の論文チェックを快く引き受けてもらうためにも,誰かから論文チェックを頼まれたら積極的に見てあげよう.
同じ学生同士,立場と持ち時間は同じなので,助け合いはとても大事である.

PDFで受け取ってPC画面で見るのも良いが,できれば印刷して紙面で読むことを推奨する(必須ではない).
読み方としては,色付きペンを用いて「すんなり読めなかった部分」にどんどん印をつけていく.
下線や丸印など,気になった箇所をマーキングしていく.
そのうえで,もし書けるのであれば「なぜ気になったか」「自分がどう読解したか」「具体的にこうした方がよい」と思ったコメントなどを余白部分に書き加えていく.
印をつけることが重要なので,これらのアドバイスは無理に書かなくてもよい.
書いて説明するのが難しい場合は,本人を呼び出して,対面やビデオ会議ツールなどを用いて口頭で説明しよう.

まったく印がつかなかった場合のみ,「特に気づかなかった」というコメントを本人に伝えよう.

%6
\section{おわりに}

本稿では,A4縦型2段組用に整型したスタイルファイルを用いた論文のフォー
マット方法と,卒業論文を念頭においた論文の書き方を示した.
内容的にまだ不十分の部分が多いため,意見,要望等はゼミ等にて随時出してください.




\begin{acknowledgment}
	本稿の執筆にあたり,参考文献に挙げた方々のWebサイト,スライド,各種資料を大いに参考にさせていただいた.
\end{acknowledgment}

