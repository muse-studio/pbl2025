\documentclass[dvipdfmx,line_length=48zw,column_gap=2zw,number_of_lines=60,baselineskip=12pt]{jlreq}
\jlreqsetup{itemization_beforeafter_space=0pt}
\makeatletter
\RenewBlockHeading{section}{1}{font={\jlreq@keepbaselineskip{\normalsize\sffamily\gtfamily}},indent=0pt,lines=1}
\RenewBlockHeading{subsection}{2}{font={\jlreq@keepbaselineskip{\normalsize\sffamily\gtfamily}},indent=0pt,lines=1}
\RenewBlockHeading{subsubsection}{3}{font={\jlreq@keepbaselineskip{\normalsize\sffamily\gtfamily}},indent=0pt,lines=1}
\makeatother
\pagestyle{empty}
% ↑この上の部分は変更しない!
% (LuaTeXを使う場合は \documentclass[dvipdfmx,... の dvipdfmx, を消す)
%%%%%%%%%%%%%%%%%%%%%%%%%%%

%%↓必要なパッケージを追加してください.
\makeatletter
\providecommand{\jlreq@keepbaselineskip}{}
\makeatother
%  ↑みなてぃ仕様

\usepackage{mathtools,amssymb}
\usepackage{latexsym}
\usepackage{newtxtext,newtxmath}
\usepackage[utf8]{inputenc} 	%波ダッシュ(〜)を表記できるようにする

\usepackage{muselabyoko} % ← muselabyoko.sty を読み込む

% \newcommand{\red}[1]{\textcolor{red}{#1}}
% \newcommand{\cut}[1]{\textcolor{red}{\sout{\textcolor{black}{#1}}}}
%%%%%%%%%%%%%%%%%%%%%%%%%%%

\begin{document}
\twocolumn[
    {\Large
            \begin{center}
                %%↓タイトルを入力してください
                % 卒業論文のタイトルを書く
自動演奏と映像の同期による音楽表現の拡張 
                %%%%%%%%%%%%%%%%%%%%%%%%%%%
            \end{center}
        }
    \vspace{12pt}
    \begin{flushright}
        福知山公立大学情報学部 32345030 金丸陽香 \\
        湊蒼志
    \end{flushright}
    \vspace{1\Cvs}
]

%% ここから本文を書く
% !TEX root = _yoko.tex

% ==========================================
% プロジェクト予稿(1P)の本文
% ==========================================

%1
\section{はじめに}

管楽器演奏においてヴィブラートは音楽表現を豊かにする重要な要素である一方,その習得は容易ではない.
ヴィブラートの揺れ方には多様なバリエーションが存在し,
どのような揺れ方がどのような表現につながるのかを具体的に理解することが難しい.
特に教育現場では,模倣や口頭説明に依存する指導が中心となり,
学習者が自ら比較・試行しながら理解するための手がかりが十分に与えられていない.

一方,電子音楽や歌声合成の分野では,ヴィブラートは音高変動として数理的に定義され,
精密な再現や最適化の対象として扱われてきた.
しかし,これらの研究は必ずしも演奏者の理解や学習を直接の目的としたものではない.

そこで本研究では,ヴィブラートを「正解を当てるべき音響現象」や「最適化すべき対象」としてではなく,
演奏者,特に初心者が違いを理解し,学習するための認知的対象として捉える。
本稿では楽器演奏におけるヴィブラート表現に着目し,その実態を整理した上で,
ヴィブラートを音響的に厳密に再現することよりも,演奏者が理解・学習可能で,
かつコンピュータ上で扱いやすい表現モデルの構築を目指す.

なお,本研究でいう「初心者」とは,
管楽器の基礎的な演奏経験および楽譜読解能力は有しているものの,
ヴィブラートを意図的かつ安定して用いる経験が十分でない演奏者を指す.

%2
\section{ヴィブラートの定義と研究の立場}

ニューグローブ音楽辞典\cite{Niyugurovu}において,ヴィブラート(vibrato)とは,
「表現性を強めるため,多少とも急速かつ微細に音高を変動させることをいう.」と記載されている.
この定義から,ヴィブラートは音の高さ(音高)を中心とした周期的な変動によって生じる効果であり,
主として声楽および多くの旋律楽器において用いられる表現技法である.
音高変動の幅や周期は楽器や演奏様式によって異なり,歴史的にもその使用法や評価は一様ではない。
すなわち,ヴィブラートは単一の正解を持つ固定的な表現ではなく,多様な形態を許容する音楽的表現である。

本研究ではこの点に着目し,ヴィブラートを物理的に正確に再現する対象としてではなく,
演奏者が違いを把握しやすい形に抽象化された表現として扱う.
そのため,音高変動をセント(cents)単位で記述し,変動幅や周期,
形状といった要素を独立に操作可能な表現モデルを構築する.

%3
\section{提案システムの概要}

提案システムは,ヴィブラートを初心者が理解・学習するための認知的対象として提示することを目的とする.
音高の周期的変動として定義されるヴィブラートを,変動幅や周期といった要素に分解し,
演奏者がそれらを操作・比較できるように設計した.
演奏者は異なるヴィブラート表現を聴き比べることで,言語化しにくい揺れ方の違いを感覚的に把握できる.
本システムは,音響現象の精密な再現を目的とするものではなく,
演奏者の理解・学習を促進するための表現モデルを与える点に本研究の特徴がある.


%4
\section{おわりに}

本研究では,ヴィブラートに関し調査し,演奏者の理解・学習するための認知的対象として捉え,
その表現モデルの構築を目指した.
ヴィブラートを正解を当てるべき音響現象や最適化の対象としてではなく,
演奏者が違いを比較しながら把握するための表現として扱い,そのための支援システムを提案した.
提案手法により,揺れ方の違いを視覚的に提示することで,
初心者がヴィブラート表現を直感的に理解するための手がかりを与えられる一例を示した.
今後は,対象楽器の拡張や教育現場での利用を通じて,
本表現モデルの有効性についてさらなる検討を行う予定である.




% 参考文献
\bibliographystyle{muselabunsrt} % bstファイルの名前
\bibliography{muselab-sample} % .bibファイルの名前(Zoteroからエクスポートして作る)

\end{document}
