% !TEX root = _yoko.tex

% ==========================================
% プロジェクト予稿(1P)の本文
% ==========================================

% 1.はじめに
\section{はじめに}
音楽の演奏では,楽譜の正確な再現に加え,演奏者が強弱やテンポに変化を加えることで,聴き手に感動を与える表現が生まれる.

この演奏表現の探求にはDAW\cite{DAW}や楽譜作成ソフトウェアが用いられるが,それぞれ演奏者の視点から見た課題が存在する.
DAWはピアノロール操作が主流で五線譜からの直感的操作に,楽譜作成ソフトウェアは再生表現が均一的で多様な解釈の比較・検討に,それぞれ課題を抱えている.

この課題は特に,吹奏楽やオーケストラに所属し,楽譜の読解に慣れ親しんだ上で,自らの演奏表現を深く探求したいと考える楽器演奏者などにとって切実である.

そこで本研究では,五線譜の視認性とDAWの編集自由度の長所を組み合わせ,演奏者が五線譜上で直感的に多様な演奏表現を編集し,比較できるシステムの開発を行った.

% 2.先行研究の整理と本研究への応用
\section{先行研究の整理と本研究への応用}

% 2.1 比較対象となる既存技術
\subsection{比較対象となる既存技術}
演奏表現を探求する既存技術として,DAW(Digital Audio Workstation)\cite{DAW}や楽譜作成ソフトウェアといったデジタルツールが挙げられる.

DAWは音響編集の自由度が高い一方,そのインターフェースはピアノロールが主流であり,五線譜に慣れ親しんだ演奏者が曲の流れや意図を感覚的に把握しながら操作するには適していない.
一方,楽譜作成ソフトウェアは五線譜の視認性には優れるが,楽譜を書くことに特化しているため,再生される演奏表現は均一的になりがちで,多様な解釈を手軽に比較・検討する創造的な用途には不十分である.

本研究は,これらの長所を融合し,双方の課題を解決するシステムの開発を目的とする.

% 2.2 設計の参考とした学術研究
\subsection{設計の参考とした学術研究}
本システムの設計においては複数の学術研究を参考にした.

表現プリセットのパラメータ設計は,音楽のエネルギー構造に着目した保科理論\cite{Hoshina}に基づいている.
また,ユーザの意図を汲み取るという基本概念は「Mixtract」\cite{Mixtract}を,フレーズ頂点を指定する機能は橋田らの研究\cite{Apex}を応用したものである.

% 3. システム概要
\section{システム概要}
本研究で開発した演奏表現のパラメータ制御システムは,Webブラウザ上で動作する.
本システムは,ユーザがアップロードした楽譜(MusicXML)と演奏(MIDI)データに対し,以下の機能を持つ.

\begin{enumerate}
    \item \textbf{五線譜上での直感的なフレーズ指定:}ユーザはGUI上の五線譜の音符を直接クリックし,「開始」「頂点」「終了」の3点を指定することで,フレーズを直感的に指定できる.
    \item \textbf{表現プリセットによるパラメータ制御:}発想標語に対応した表現プリセットを選択するだけで,フレーズ内の強弱やテンポのパラメータが自動的に計算・適用される.
    \item \textbf{比較・視聴機能:}生成された音源は,加工前の音源と,加工後の単一パート・全体のパートの奏法で比較・視聴でき,表現による変化を確認できる.
\end{enumerate}

% 4. 適用例
\section{適用例}
% --スペースがあればここに書く
本システムの有効性を示すため,~~を対象に,演奏表情の付与を行った.
元のMIDIデータは~~であったが,本システムを用いてフレーズを指定し,~~の表現プリセットを適用した.
その結果,~~.

% 5. おわりに
\section{おわりに}
本研究では,五線譜上で直感的に演奏表現を編集し,比較できるシステムを提案・実装した.
実際に本システムを用いて楽曲に表現を適用したところ,~~.
今後の展望として,~~が挙げられる.