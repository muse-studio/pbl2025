% !TEX root = _main.tex
% ========================================
% 卒業論文 本文
% ========================================

%1
\section{はじめに:3000字}

\begin{enumerate}
    \item 研究背景(現代の物語表現は映像やテキストに依存している)
    \item 背景を踏まえた上での研究目的(音だけの物語にはどのような表現上の特徴や可能性、限界があるのか)
    \item 研究の意義(音だけの物語は、物語研究にどのような視点を与えられるのか)
    \item 論文の構成の説明
\end{enumerate}
%2
\section{先行研究:5000字}
\begin{enumerate}
    \item 問題提起
    \item この研究をするまでの背景
    \item 背景を踏まえた上での説明
    \item 論文の構成の説明
\end{enumerate}
\subsection{物語論}
物語とは何かを分析するためのアプローチとして,プロップ(1968)やラボフ(1967)に代表される構造主義的アプローチが存在している.

プロップ(1968)はロシアの100の民話を分析し,民話の人物たちの多様な行動(例:「主人公が出発する」「敵が主人公を騙す」「主人公が試練に勝利する」)の中に,31種類の「機能」と呼ばれる不変の行動単位を見出した.一方,ラボフ(1967)はニューヨークの日常会話で語られる個人的な体験談を分析し,口承による個人経験談には,
\begin{enumerate}
    \item 抽象(話の要約),
    \item 指向(時・場所・人物などの状況設定),
    \item 出来事(核心となる事件の展開),
    \item 評価(話の「オチ」や語る意義を示す部分),
    \item 結果(事件の結末),
    \item 締めくくり(現在への結びつけによる終結宣言)
\end{enumerate}
という6つの構造的要素が典型的に順序立てて現れることを明らかにした.

これらは,物語を1つの構造物と見なし,それを構成する普遍的な要素とその配列における規則を発見しようとするアプローチであり,多様な物語表現の背後に形式的な共通パターンを抽出したという点で重要な進展をもたらすものであった.

しかしながらこのような構造主義的アプローチには主に次のような限界が指摘できる.

第一に,物語を形式的な完成度に依存して評価する傾向があり,物語を静的なものとしてしか捉えられていないこと,第二に,物語を静的な構造物として分析の対象とすることにより、物語が実際に語られる場面や社会的な文脈が看過されがちであること.第三に,特にラボフのモデルは典型的で首尾一貫した経験談を想定しており,相手の相づちや質問によって中断され,話の方向性を修正されるという過程を通じて複数人で作られていくという日常的な語りの特徴を捉えきれないことである.結果としてこのアプローチは,物語が語り手や聞き手の人生観や世界観にいかに影響を与え,意味を生成するかという,動的で機能的な側面を十分に説明することが難しい.
\vskip\baselineskip
こうした構造主義の限界を考慮した上で,物語を語る行為そのものの場面やそれが聞き手に与える影響という事項を含めた物語のより本質的な意味を探ることを目的とする理論を考案したのが,シフ(2012)である.シフは,物語研究の焦点を名詞の「語り」から動詞の「語る」へと移すことを提唱している.

 シフはこの語ることの最も基本的な機能は「現在化」にあるとした。現在化とは,人生への理解が、現在の状況において,また物語の語り手と聞き手の相互行為を通じて,
\begin{enumerate}
    \item 宣言的に経験に存在感を与え,
    \item 時間的に過去・現在・未来に意味の連続性を構築し,
    \item 空間的(社会的)に他者と世界の理解を共創造する,
\end{enumerate}
という三側面を持つ包括的な行為である。

\section{作品の説明:10000字}

\begin{enumerate}
    \item 作品コンセプト
    \item あらすじの説明
    \item 使用した音の種類
    \item 制作プロセスの説明(そうくんのツールで作ったEMDAをここで利用)
    \item 直面した課題点
    \item 部分的な成果(12/20の発表会とそのフィードバックの説明)
\end{enumerate}

\section{まとめ:2000字}

\begin{enumerate}
    \item 研究で明らかになったこと(音のみの物語の可能性と限界点)
    \item 今後の課題
    \item 研究の意義を再確認した締め
\end{enumerate}