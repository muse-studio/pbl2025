% !TEX root = _main.tex
% ========================================
% 卒業論文 本文
% ========================================
% -1.はじめに-
\section{はじめに}
% 研究背景
音楽の演奏において,聴き手に豊かな感動を届けるためには,楽譜に記された音符や記号を正確に再現するだけでは不十分である.
演奏者自身が楽曲を深く解釈し,強弱(ダイナミクス)やテンポの緩急に繊細な変化を加えることで,聴き手に感動を与える表現が生まれる.
この演奏表現を探求する過程において,近年のデジタルツールは強力な支援となりうるが,既存のツールには演奏者の視点から見ていくつかの課題が存在する.

現在の音楽制作において主流であるDAW(Digital Audio Workstation)\cite{DAW}は,ピアノロール画面を用いて音量やタイミングを数値レベルで緻密に編集できる強力なツールである.
しかし,そのインターフェースはトラックベースの音響制作に最適化されており,普段から五線譜に慣れ親しんだ吹奏楽やオーケストラの演奏者が,楽譜全体の構造やフレーズのまとまり,和声の流れといった音楽的文脈を直感的に読み取りながら表現を編集するには不向きである.

一方で,Dorico\cite{Dorico}やSibelius\cite{Sibelius}に代表される楽譜ソフトウェアは,近年の製品では高品質なサンプリング音源を用いた再生も可能となっている.
しかし,主な目的はあくまで正確な記譜にある.
そのため,「Cantabile(歌うように)」や「Dolce(甘く)」といった演奏指示(発想標語)が再生音に与える影響は,ソフトウェアにあらかじめ定義された標準的な解釈が適用されることが多い.
もし演奏者が,フレーズの頂点の位置や抑揚のカーブなどを自身の解釈に合わせて微調整しようとしても,その操作は直感的な五線譜上では完結しない.
結果として,ピアノロール画面やグラフエディタを用いたパラメータの編集作業が求められ,フレーズに対して複数の表現を比較・検討するといった創造的な試行錯誤を支援する機能は十分とはいえない.

このように,演奏者が慣れ親しんだ「五線譜」というインターフェースの上で,DAWのように柔軟かつ具体的な音響編集を直感的に行えるツールは不足しており,両者の間には明確なギャップが存在する.
筆者自身も吹奏楽の演奏経験の中で,楽譜に記された演奏指示の解釈に迷い,それが具体的にどのような音の変化として表現されるのかを試行錯誤する際に,適切な支援ツールが存在しないことに課題を感じてきた.

% 研究目的
そこで本研究では,楽譜ソフトウェアの持つ優れた視認性(五線譜インターフェース)と,DAWの持つ柔軟な音響編集能力の長所を組み合わせることを目的とし,Webブラウザ上で動作する「楽譜上のフレーズに基づく演奏表情のパラメータ制御システム」を構築した.

本研究の主な対象は,DAWよりもパート譜での練習が中心となる管楽器奏者や,演奏表現を深く探求したいと考えている学生・アマチュア奏者である.
本システムは,抽象的な発想標語が音量やテンポといった物理的な演奏パラメータとしてどう表現されるのかを可視化・聴取可能にすることで,演奏者が自身の音楽的解釈を具体的かつ主体的に構築するプロセスを支援するものである.

% -2.関連研究-
\section{関連研究} \label{sec:Background}
本章では,既存の音楽制作環境や楽譜ソフトウェア,および演奏表情付けに関する先行研究について整理し,それらの課題と本研究の位置づけについて述べる.

% -2.1 既存の音楽制作環境-
\subsection{既存の音楽制作環境}

% -2.1.1 DAW(Digital Audio Workstation)-
\subsubsection{DAW(Digital Audio Workstation)}
PCを用いた音楽制作において,最も一般的かつ高機能なツールとしてDAW(Digital Audio Workstation)\cite{DAW}が挙げられる.
代表的なソフトウェアには,Cubase\cite{Cubase}やLogic Pro\cite{LogicPro}などがある.
これらのソフトウェアでは,MIDIデータの音の高さ,長さ,強弱(Velocity),演奏の表情や抑揚に関するControl Change(CC)などのパラメータを数値レベルで緻密に編集することが可能である.

% 図1
\begin{figure}
	\centering
	\includegraphics[width=\hsize]{../fig/Cubase.png}
	\caption{DAWの画面例(Cubase)}
	\label{fig:Cubase}
\end{figure}

しかし,その編集インターフェースの多くは「ピアノロール」と呼ばれる,縦軸に音高,横軸に時間をとったグラフ形式が採用されている(\figref{fig:Cubase}).
これは,音響的な編集作業には適しているが,五線譜に慣れ親しんだ演奏者にとっては,フレーズのまとまりや和声の流れといった音楽的な文脈を直感的に把握しにくいという課題がある.
また,クレッシェンドやデクレッシェンドといった抑揚をつけるためには,オートメーションレーンと呼ばれる領域にマウスで曲線を書き込む操作が必要となり,演奏表現の探求においては直感的とは言い難い.

% -2.1.2 楽譜ソフトウェア-
\subsubsection{楽譜作成ソフトウェア}
五線譜を用いた記譜に特化したツールとして,MuseScore\cite{MuseScore}やDorico\cite{Dorico},Sibelius\cite{Sibelius}などが広く普及している.
これらのソフトウェアは,綺麗な楽譜を作成することを主な目的としているが,演奏表現の再生機能に関しては,本研究の目的である「演奏者が主体的に表現を探求する」という観点から見ると以下の課題がある.

% MuseScore
まず挙げられるのが,楽譜上の発想標語が十分に音に反映されない点である.
例えば,最新のMuseScore 4においては,高品質な音源(MuseSounds)が搭載されているものの,楽譜上に「Cantabile」などの発想標語をテキストとして配置しても,それはあくまで視覚的な文字情報として扱われ,再生される音には変化がないことが多い.
音量やテンポに変化を与えるためには,クレッシェンドなどの強弱記号を1つずつ配置するか,個々の音符の設定画面を開いて数値を手動で入力する必要があり,直感的に様々な表現を試すことは難しい.

% Dorico
MuseScoreのようなソフトウェアに加え,Doricoなどの高機能なソフトにおいても,操作の複雑さから五線譜の持つ直感性が十分に活かされていないという課題が見られる.

% 図2
\begin{figure}
	\centering
	\includegraphics[width=\hsize]{../fig/Dorico.png}
	\caption{楽譜ソフトウェアの画面例(Dorico)}
	\label{fig:Dorico}
\end{figure}

Doricoなどのソフトウェアでは,楽譜の下部に「キーエディタ」と呼ばれるピアノロール画面を表示し,そこでDAWと同様にマウスで曲線を書き込むことで詳細な表現付けが可能となっている(\figref{fig:Dorico}).
しかし,これは実質的に五線譜のインターフェースから離れた操作であり,DAWにおける描画作業と本質的に変わらない.

% Sibelius
さらに,処理の中身が分かりにくく,細かい指定ができない点も課題である.
Sibeliusには,「Espressivo機能」や「Rubato機能」といった再生機能が存在し,これらを使うことで人間らしい揺らぎを加えることができる.
しかし,これらの機能は楽曲全体に対して一律に適用されるため,「この特定のフレーズだけを溜めて演奏したい」といった細かい意図を反映させることは難しい.
また,それらの設定によって具体的に音量が上がったのか,テンポが遅くなったのかといった中身がユーザには見えないため,演奏者がパラメータと聴こえ方の関係を学ぶことは困難である.

このように,五線譜という見やすいインターフェースを維持したまま,フレーズ単位で直感的かつ中身の見えるパラメータ制御を行える環境は,既存の楽譜ソフトウェアでは実現されていない.

% -2.1.3 高品質なAI演奏生成ツール-
\subsubsection{高品質なAI演奏生成ツール}
近年では,AIを用いて楽譜情報から人間らしい演奏を自動生成する技術も実用化されている.

% Note Performer
Note Performer\cite{NotePerformer}は,SibeliusやDoricoなどの楽譜作成ソフトウェアに組み込んで使用するAI演奏エンジンである.
これは,楽譜上の音符や基本的な記号を読み取り,前後の文脈を考慮して,音の立ち上がりや減衰,拍の強弱(Velocity)などを自動的に補正する機能を持つ.
これにより,ユーザが細かい調整を行わなくても機械的ではない自然な演奏を生成できる点は大きな利点である.
しかし,その処理プロセスはブラックボックス化されており,ユーザが特定のフレーズだけを極端に強調したいなどの発想標語を用いた独自の解釈を反映させようとしても,AIの自動補正が優先され,意図通りの変化が得られない場合がある.

% ACE Studio(AIバイオリン)
ACE Studio\cite{ACEStudio}は,AIを用いた歌声合成技術を楽器の演奏生成に応用した音楽制作プラットフォームである.
ACE StudioのAIバイオリンは,MIDIデータを入力するだけで,AIが最適なアーティキュレーション(奏法)や感情表現を自動的に付加し,高品質でリアルな演奏を生成することができる.
例えば,MIDIノートの長さを短くするだけでスタッカート気味の奏法に変化するなど,AIによる解釈の精度は非常に高い.

しかし,これらはあくまで「自動化」による品質向上に重点が置かれている.
現状では,Expressionなどの細かいニュアンスを手動で制御する機能は限定的であり,演奏者が意図的に「フレーズ内で最も強調したい音の位置を変えたい」といった繊細な表現の試行錯誤を行うことは困難である.

% 図3
\begin{figure}
	\centering
	\includegraphics[width=\hsize]{../fig/ACEStudio.png}
	\caption{AI演奏生成ツールの画面例(ACE Studio)}
	\label{fig:ACEStudio}
\end{figure}

また,その操作インターフェースは\figref{fig:ACEStudio}のようなピアノロール形式であり,五線譜を用いて音楽構造を捉えながら表現を探求する本研究のアプローチとは,対象とするユーザ層や利用目的が異なるといえる.

これらの技術は最終的な成果物のクオリティを効率よく高める点では非常に優れている.
しかし,生成プロセスがブラックボックス化されているため,演奏者がどのようなパラメータ操作が音の変化に繋がるのかを理解することは難しい.
そのため,ユーザはAIが生成した演奏を受動的に受け入れる形での利用に留まりやすい.

% -2.2 演奏表情付けに関する先行研究 -
\subsection{保科理論:音楽構造とエネルギー運動}
本研究では,システムにおける演奏表現の考え方として,保科洋による「保科理論」\cite{Hoshina}を参考にした.
保科理論は,楽譜上の音符の並びを,物理的なエネルギーの動きに例えて解説したものである.
以下に,本システムの設計において特に重要となる「音楽構造の階層化」と「頂点(重心)の概念」について述べる.

% -2.2.1 音楽構造の階層モデル:グループとフレーズ -
\subsubsection{音楽構造の階層モデル:グループとフレーズ}

% 図4
\begin{figure}
	\centering
	\includegraphics[width=\hsize]{../fig/hoshina_phrasing.png}
	\caption[保科理論におけるフレーズ構造]{保科理論におけるフレーズ構造(文献\cite{Hoshina}より抜粋)}
	\label{fig:phrasing}
\end{figure}

保科理論では,楽譜上の音符の羅列を,エネルギーのまとまりとして階層的に捉える(\figref{fig:phrasing}).
その最小単位がグループである.
グループとは,複数の音を1つの意味のあるまとまりとして捉えたものであり,言語における「単語」に相当する.
楽譜上の音符を適切にグルーピング(分節化)することで,演奏に意味が生まれる.

さらに,複数のグループが連なることで,より大きなまとまりである「フレーズ」が形成される.
これは言語における「文章」に相当する.
楽曲は,\figref{fig:phrasing}のように,小さなグループが集まって大きなフレーズを作るという階層構造になっている.

% -2.2.2 頂点(重心)-
\subsubsection{頂点(重心)}
保科理論では,「独立したひとつのグループには,必ず1つの頂点が存在する」と定義している.
頂点(重心)とは,グループの中で最もエネルギーが集中する音符のことを指す.
演奏表現は,この頂点を中心として以下のような山型のカーブを描くことで生成される.

\begin{description}
	\item[(1) アナクルーズ]
		グループの開始点から頂点に向かって,エネルギーを徐々に高めていく過程(クレッシェンド)
	\item[(2) 頂点]
		エネルギーが最大化する点
	\item[(3) デジナンス]
		頂点を越えた後,エネルギーを減衰させていく過程(ディミヌエンド)
\end{description}

どの音が頂点となるかは,音符の並び(旋律線),和声進行,拍節(強拍・弱拍),アーティキュレーションなどの音楽的要素によって決定される.
本研究では,「グループごとに1つの頂点を持ち,そこに向かって強弱が変化する」という原則を,システムの制御モデル(Expression値の線形補間)として採用している.

% -2.3 Mixtract:演奏表現のデザイン環境 -
\subsection{Mixtract:演奏表現のデザイン環境}

橋田らは,前節で述べた保科理論の考え方を計算機上で実現し,ユーザが演奏表現を自由に設計できるシステム「Mixtract」を提案している\cite{Mixtract}.

% 図5
\begin{figure}[tb]
	\centering
	\includegraphics[width=\hsize]{../fig/Mixtract.png}
	\caption[演奏表現デザイン環境「Mixtract」]{演奏表現のデザイン環境(文献\cite{Mixtract}より抜粋)}
	\label{fig:Mixtract}
\end{figure}

\figref{fig:Mixtract}にMixtractの概要を示す.

このシステムの特徴は,AIが勝手に全ての演奏を決めてしまうのではなく,ユーザ自身が「どこがフレーズか」「どこが頂点か」を決定し,それに基づいて演奏を生成する点にある.

具体的な処理の流れは,システムが認知的音楽理論であるGTTM\cite{GTTM}を用いて,楽曲のフレーズ構造(階層構造)を自動解析して提示する.
同時に,保科理論に基づきフレーズ内での頂点(重心)を自動推定する.
Mixtractでは,各音符が持つエネルギー値を算出し,それを音量やテンポなどの表現要素に変換することで,保科理論における山型の表現を実現している.

ユーザは提示された構造や頂点を確認し,必要に応じて自分の解釈に合わせてフレーズの範囲や階層を修正する.
決定されたフレーズに対し,「テンポ(速さ)」「ダイナミクス(強弱)」などの表情カーブを割り当てることで,抑揚ある演奏が生成される.
ユーザは生成された演奏を聴取し,納得がいかなければ再度設定を調整する.
この「設定→聴取→修正」というサイクルを繰り返すことで,ユーザの頭の中にあるイメージを具体的な演奏として再現することができる.

ユーザが主体となって試行錯誤するという設計思想は,本研究が目指すシステムにおいて非常に重要な指針となっている.
しかし,Mixtractにおける強弱の制御は,主にVelocity(打鍵の強さ)を対象として実装されている.
これはピアノなどの打鍵楽器には有効であるが,管楽器のように音を伸ばしている最中に音量を膨らませること(Expression制御)が重要な楽器に対しては,その表現力を十分に活かしきれないという課題がある.
本研究では,Mixtractのユーザ主導のデザインおよびエネルギー値(頂点)に基づく表現生成という方向性を継承しつつ,制御対象をVelocityではなくExpression(CC2)に置き換えることで,管楽器に適した演奏表現システムを構築する.

% -2.4 フレーズ頂点の推定手法 -
\subsection{フレーズ頂点の推定手法}
保科理論を計算機システム上で実現するためには,システムが楽譜の中から自動的に頂点(重心)を特定する,あるいはユーザの頂点決定を支援する仕組みが必要となる.
これに対し,橋田らは聴取実験に基づいたグループと頂点の推定モデル\cite{Apex_group}や,スラー境界情報に基づく手法\cite{Apex_phrase}を提案している.

% -2.4.1 グループと頂点の推定 -
\subsubsection{グループと頂点の推定}
橋田らのグループと頂点の推定に関する研究\cite{Apex_group}では,ベートーヴェンのピアノソナタ「悲愴」などを題材に,人間の演奏家がどこを頂点として捉えているかを分析し,頂点となりやすい音符の条件をルール化している.
提案されたモデルは,以下の音楽的特徴を持つ音符に対し,頂点らしさを表すポイント(スコア)を加算していく「投票(Voting)方式」を採用している.

\begin{description}
	\item[(1) 音高]
		フレーズ内で相対的に音高が高い音は,エネルギーが高く頂点になりやすい.
	\item[(2) 音価]
		周囲の音符よりも長い音価を持つ音は,強調される傾向にある.
	\item[(3) 跳躍進行]
		音程が大きく跳躍した(特に上行跳躍した)到達音は,強いエネルギーを持つ.
	\item[(4) 和声的緊張感]
		非和声音(倚音や掛留音)や,和声進行上で緊張度の高い音(ドミナントなど)は頂点となる可能性が高い.
	\item[(5) 特定の音型]
		「ジェットコースター音型」と呼ばれる,下行してから上行するような旋律の変曲点が頂点となる場合がある.
\end{description}

% -2.4.2 スラー境界に基づく頂点推定 - 
\subsubsection{スラー境界に基づく頂点推定}
橋田らのスラー境界に基づく頂点推定の研究\cite{Apex_phrase}では,単純な音高や音価の比較だけでなく,特定の音型パターンがグルーピングや頂点位置に与える影響についても論じられている.
その代表例が「バウンド分割」である.

バウンド分割とは,1拍以下の短い音価に分割された音群の後に,長い音価の音が続くパターンのことを指す.
一般的に,短い音符の連続の後に現れる長い音符はエネルギーを受け止めやすく,頂点になりやすい性質がある.
しかし,短い音群が拍節の強拍(小節の頭など)に現れる場合は,その短い音群自体にアクセントが生じ,頂点となる場合がある.
先行研究では,このような「バウンド分割」や前述の「ジェットコースター音型(下行後の上行)」といった特徴的なパターンを検出することで,頂点の判定精度を向上させている.

これらの先行研究は,和声分析やスラー情報を含む多角的な情報を用いることで,高い精度(約80%以上)での頂点推定を実現している.
しかし,これらのモデルをそのまま管楽器のパート譜(単旋律)に適用しようとした場合,伴奏パートの和声情報が欠落していたり,計算が複雑になりすぎたりするという課題がある.
これらの背景から本研究では,グルーピング(フレーズ範囲の決定)はユーザの音楽的解釈を尊重し直感的な操作に委ねる一方,範囲内での頂点推定においては,単旋律からでも判断可能な「バウンド分割」などの音型ルールを限定的に採用するアプローチをとる.

% -3. 演奏表現支援システムの設計 -
\section{演奏表現支援システムの設計} \label{sec:Design}
本章では,第\ref{sec:Background}章で述べた課題を解決し,管楽器奏者が五線譜上で直感的に演奏表現を探求できるシステムの設計方針について述べる.

% -3.1 設計の基本方針 -
\subsection{設計の基本方針}
本システムは,以下の3点を基本方針として設計する.

\begin{enumerate}
	\item \textbf{五線譜インターフェースでの操作}\\
		DAWのようなピアノロールではなく,演奏者が日常的に使用する五線譜を操作の基盤とする.
		また,楽譜作成ソフトウェアのような見やすさを保ちつつ,演奏表現の編集に特化したシンプルなインターフェースを提供する.
	\item \textbf{フレーズ単位でのパラメータ制御}\\
		音符1つ1つを編集するのではなく,音楽的な意味のまとまりである「フレーズ」を選択し,それに対して一括で表現を適用する方式をとる.
		これにより,音楽的な文脈を保ったままの試行錯誤が可能となる.
	\item \textbf{聴き比べによるフィードバック}\\
		選択した発想標語やパラメータ設定が,実際の音にどのように反映されるかを即座に聴取できる環境を構築する.
		数値の変化だけではなく,聴覚的な変化を確認することで,演奏者の理解を深める.	  
\end{enumerate}

% -3.2 演奏制御パラメータの設計 -
\subsection{演奏制御パラメータの設計}
本システムでは,演奏表現を構成する要素として,強弱と時間の2点に着目する.
特に管楽器や弦楽器のような持続音を制御できる楽器の特性を考慮し,以下のパラメータを制御対象とする.

% -3.2.1 Expression(CC2)による強弱制御
\subsubsection{Expression(CC2)による強弱制御}
ピアノなどの打鍵楽器では,音の強さは打鍵の瞬間の速度(Velocity)で決定され,発音後に音量を変化させることはできない.
しかし,管楽器は息の圧力(ブレスコントロール)によって,音の持続中に音量を自由に増減させることができる.
実際,浅野による先行研究では,この持続音の特性に着目し,フレーズの頂点を基準としてExpression値を線形補間する手法が提案されている\cite{LongTone}.
第\ref{sec:Background}章で述べた保科理論における頂点に向かうクレッシェンドや,頂点を越えた後のディミヌエンドを表現するためには,Velocityだけでは不十分である.
MIDIのコントロールチェンジの2番(以下,CC2とする)は,拍よりさらに細かいTick単位で音量の制御が可能である.

そのため,本システムではCC2を用いて,フレーズ内の音量を連続的に変化させる(線形補間する)手法を採用する.


% -3.2.2 Onsetによる発音時刻の制御 -
\subsubsection{Onsetによる発音時刻の制御}
発想標語には,強弱だけでなく速度の変化も含まれる.例えば,「Maestoso(堂々と)」であれば,単に音が大きいだけでなく,1音1音を十分に保って演奏するような,わずかな遅れ(溜め)が生じることが一般的である.
本システムでは,BPM(楽曲全体のテンポ)を変更するのではなく,各音符の発音時刻(Onset)をミリ秒単位で前後にずらすことで,局所的なテンポの揺らぎを表現するものとする.

% -3.3 発想標語プリセットの作成 -
\subsection{発想標語プリセットの作成}
ユーザがパラメータを1から設定する負担を軽減するため,代表的な発想標語に対応するパラメータの組み合わせをプリセットとして実装した.

プリセットの値(CC2の増減量,Onsetの遅延量)の決定にあたっては,楽器経験のある学生4名の協力を得て予備調査を行った.
調査では,被験者に本システムを使用させ,各発想標語のイメージに合致するパラメータ値を初期値として採用した.
これにより,個人の主観に偏り過ぎない,ある程度客観的な妥当性のあるプリセットを提供することを目指した.

本システムでは,「Cantabile(歌うように)」といった旋律的な表現から,「Appassionato(情熱的に)」といった激しい表現,「Tranquillo(静かに)」といった穏やかな表現まで,多様な音楽的性格を比較できるよう,代表的な10種類の発想標語を選定し,プロトタイプとして実装した.
これらのプリセットの値の決定にあたっては,上記の調査結果に基づき\tabref{tab:table1}に示す値を定義した.

% 表1
\begin{table}[tb]
    \centering
    \caption{実装した発想標語プリセットとパラメータ設定値}
    \label{tab:table1}
    \begin{tabular}{l|l|r|r|r} \hline
        \textbf{発想標語} & \textbf{意味} & \textbf{Base} & \textbf{Peak} & \textbf{Onset} \\
         & & \textbf{CC2} & \textbf{CC2} & \textbf{(ms)} \\ \hline \hline
        Cantabile & 歌うように & +15 & +35 & +20 \\
        Dolce & 甘く,柔らかく & -25 & +10 & +15 \\
        Maestoso & 荘厳に,堂々と & +20 & +50 & +40 \\
        Appassionato & 情熱的に & +25 & +60 & -30 \\
        Con brio & 生き生きと & +15 & +40 & -40 \\
        Leggiero & 軽く,軽快に & -20 & +5 & -30 \\
        Tranquillo & 静かに,穏やかに & -35 & +5 & +30 \\
        Risoluto & 決然と,きっぱりと & +20 & +45 & 0 \\
        Sostenuto & 音を十分に保って & +10 & +20 & +50 \\
        Marcato & はっきりと & +15 & +65 & 0 \\ \hline           
    \end{tabular}
\end{table}

各パラメータの役割と設定の意図は以下の通りである.

\begin{description}
	\item[Base CC2とPeak CC2(強弱の制御):]
		  これらは,選択されたフレーズ範囲における元の演奏データに含まれるCC2値の平均値を基本とし,それに対する増減量を表す.
	\begin{itemize}
		\item \textbf{Base CC2:}フレーズの開始点と終了点での音量レベルを決定する.
		\item \textbf{Peak CC2:}フレーズの頂点における音量レベルを決定する.
	\end{itemize}
例えば,「Appassionato(情熱的に)や「Maestoso(堂々と)」は,Peakの値をBaseよりも大幅に高く設定することで,頂点に向かう急激なクレッシェンドを生み出し,力強い音楽表現を実現している.
また,「Dolce(甘く)」や「Tranquillo(穏やかに)」は,BaseとPeakの差を小さく,全体的に低い値に設定することで,起伏の少ない平穏な強弱変化としている.

\item[Onset(時間的な揺らぎ):]
	 各音符の発音タイミングを,楽譜上の正規の位置からミリ秒単位でずらすパラメータである.これにより,音楽表現における時間的な揺らぎを再現している.
	 「Maestoso(堂々と)」のように重厚感が求められる標語には正の値(遅延)を与えて溜めを作り,「Con brio(生き生きと)」のように前進するエネルギーが必要な標語には負の値(前倒し)を与えて疾走感を表現している.
\end{description}

% -3.4 頂点推定ルールの選定 -
\subsection{頂点推定ルールの選定} \label{sec:ApexRule}
フレーズ内の頂点を決定する支援機能として,自動推定機能を設計する.
第\ref{sec:Background}章で述べた先行研究\cite{Apex_group}のモデルを参考にしつつ,本システムでは伴奏や和声情報を持たない単旋律のパート譜でも機能するよう,以下の4つの観点からルールを選定した.
詳細なルールと配点を\tabref{tab:table2}に示す.

\begin{enumerate}
	\item \textbf{音価:}長い音符は強調されやすい.
	\item \textbf{音高:}高い音符はエネルギーが高い.
	\item \textbf{進行:}跳躍進行の到達音や,特定の音型(バウンド分割など)に着目する.
	\item \textbf{位置:}フレーズの開始・終了音は頂点になりにくい.
\end{enumerate}

% 表2
\begin{table}[tb]
	\centering
	\caption{本システムで採用した頂点推定ルールと評価ポイント}
	\includegraphics[width=\hsize]{../fig/Apex_Rule.png}
	\label{tab:table2}		
\end{table}

% ※上の表は以下の部分をコンパイルしてスクショしました(Apex_Rule.png)
% 画像の大きさどのくらいがいいのかが分かりません…

% % 表2
% \begin{table*}[tb]
%   \caption{本システムで採用した頂点推定ルールと評価ポイント}
%   \label{tab:table2}
%   \large
%   \begin{tabular}{l|l|l|c|c|c} \hline
%     \multirow{2}{*}{\textbf{分類}} & \multirow{2}{*}{\textbf{対象音}} & \multirow{2}{*}{\textbf{条件}} & \multicolumn{3}{c}{\textbf{評価ポイント}} \\ \cline{4-6}
%      & & & \textbf{先行音} & \textbf{対象音} & \textbf{後続音} \\ \hline \hline
%     \multirow{3}{*}{音価} & 隣接する2音の第1音 & 後続音より短い & - & 0 & 1 \\
%      & & 後続音より長い & - & 1 & 0 \\ \cline{2-6}
%      & 同一音価の連続する音群の第一音 &  & - & 1 & - \\ \hline
%     \multirow{3}{*}{音高} & 隣接する2音の第1音 & 後続音より低い & - & 0 & 1 \\
%      & & 後続音より高い & - & 1 & 0 \\ \cline{2-6}
%      & 同一音価が連続する音群の第2音以降 &  & - & {発音順/音符数} & - \\ \hline
%     \multirow{5}{*}{進行} & \multirow{5}{*}{\shortstack[l]{進行到達音\\(隣接4音の第3音)}} & 1) 上行-上行-下行 & 0 & 1 & 0 \\
%      & & 2) 下行-上行-下行 & 2 & 1 & 0 \\
%      & & 3) 上行-下行-上行 & 1 & 2 & 1 \\
%      & & 4) 下行-下行-上行 & 0 & 2 & 1 \\
%      & & 5) 1)2)で基本音長0.25秒以下 & 1 & 0 & 0 \\ \hline
%     \multirow{6}{*}{グループ} & 開始音 &  & - & 1 & - \\ 
% 	& 終了音 &  & - & -1 & - \\ \cline{2-6}
%      & \multirow{2}{*}{最長音} & バウンド分割を含む & - & 1 & - \\
%      & & バウンド分割を含まない & - & 2 & - \\ \cline{2-6}
%      & \multirow{2}{*}{最高音} & 音価の中央値より長く0.25秒以上 & - & 2 & - \\
%      & & その他 & - & 1 & - \\ \cline{2-6}
%      & \multirow{2}{*}{\shortstack[l]{最大の上行跳躍音}} & 音価の中央値より長く0.25秒以上 & 0 & 2 & - \\
%      & & その他 & 2 & 0 & - \\ \hline
%   \end{tabular}
% \end{table*}

% -4. システムの実装 -
\section{システムの実装}
本章では,第\ref{sec:Design}章の設計に基づき実装した「楽譜上のフレーズに基づく演奏表情のパラメータ制御システム」の詳細について述べる.

% -4.1 システム概要 -
\subsection{システム概要}
本システムは,フロントエンドにHTML,CSS,JavaScript,バックエンドにPython(Flask\cite{Flask})を用いて,ローカル環境で動作するWebアプリケーションとして構築した.
システム全体の処理の流れを\figref{fig:System_flow}に示す.

% 図6
\begin{figure*}[tb]
	\centering
	\includegraphics[width=\hsize]{../fig/System_flow.png}
	\caption{システムの全体構成と処理の流れ}
	\label{fig:System_flow}
\end{figure*}

ユーザは,WebブラウザからMusicXML(楽譜データ)とMIDI(演奏データ)をアップロードする.
システムはこれらを解析し,ブラウザ上に楽譜を表示するためのデータと,音符と時間を対応付けるマッピングデータを生成する.
ユーザは楽譜上でフレーズと頂点を指定し,発想標語プリセットを選択する.
サーバー側では,その指示に基づいてMIDIデータを加工し,FluidSynth\cite{FluidSynth}を用いてWAV形式の音源を生成する.
また,加工されたMIDIデータをユーザのローカル環境に保存する.

% -4.2 開発環境と使用したライブラリ -
\subsection{開発環境と使用したライブラリ}
本システムの実装に使用した主な言語・ライブラリは以下の通りである.

\begin{itemize}
	\item \textbf{言語:}Python 3.12, JavaScript, HTML5, CSS3
	\item \textbf{Webフレームワーク:}Flask
	\item \textbf{楽譜解析・描画:}
	\begin{itemize}
		\item \textbf{music21}\cite{music21}\textbf{:}MusicXMLの解析,パート抽出を行う.
		\item \textbf{xml2abc}\cite{xml2abc}\textbf{:}MusicXMLをABC記譜法に変換する.
		\item \textbf{abcjs}\cite{abcjs}\textbf{:}ブラウザ上でABC記譜法を表示し,クリック操作に対応した楽譜表示を実現する.
	\end{itemize}
	\item \textbf{MIDI処理・音声合成:}
	\begin{itemize}
		\item \textbf{mido}\cite{mido}\textbf{:}MIDIメッセージの解析・生成・編集を行う.
		\item \textbf{FluidSynth}\textbf{:}加工したMIDIデータを高品質なSoundFontを用いてWAV音声に変換する.
	\end{itemize}
\end{itemize}

% - 4.3 実装の詳細 -
\subsection{実装の詳細}

% -4.3.1 楽譜データと演奏データの対応付け
\subsubsection{楽譜データと演奏データの対応付け}
本システムにおける最大の技術的課題の1つは,楽譜上の音符(MusicXML)と演奏データの音符(MIDI)を正確に対応付けることである.
MusicXMLは小節や拍といった音楽的な単位で時間を管理するのに対し,MIDIはTick(時間分解能)という絶対的な数値で管理するため,両者の時間軸の定義が異なる.

そこで本システムでは,サーバ側での解析処理(\texttt{app.py},\texttt{midi\_processor.py})において,MusicXMLのテンポ情報をもとに各音符の発音時刻をTickに換算し,MIDIイベントと対応させる処理を実装した.
具体的には,音符のindex,Pitch,発音時刻(msおよびTick)を記録した音符対応表(NoteMap)をJSON形式で生成する.
これにより,次項で述べるユーザ操作と内部処理の連携を実現している.

% -4.3.2 楽譜表示と表情指示の反映 -
\subsubsection{楽譜表示と表情指示の反映} \label{sec:Rendering}
ユーザが五線譜上でフレーズを指定し,演奏表情(発想標語)を指示するためには,ブラウザ上のGUIとシステム内部のデータ構造が同期している必要がある.
本システムでは,五線譜の描画に\texttt{abcjs}ライブラリを採用し,以下の仕組みを実装した.

\noindent \textbf{(1) 操作位置の特定}

\texttt{abcjs}は,MusicXMLから変換されたABC記譜データをSVG形式で描画する際,各音符の要素(DOM)に対してクリックイベントを検知する機能を持つ.
ユーザが画面上の任意の音符をクリックすると,システムはその音符のindex番号とTickを取得する.
このindexとTick情報から前項で生成したNoteMapを参照することで,クリックされた箇所が「楽曲全体の何Tick目にあるのか」を即座に特定する仕組みを構築した.

\noindent \textbf{(2) 表情指示の可視化}

ユーザがフレーズ範囲と頂点を決定し,プリセットから発想標語(例:Cantabile(歌うように))を選択すると,システムはその指示を楽譜上に視覚的に反映させる.
具体的には,SVG要素を操作し,選択されたフレーズ範囲に緑色のハイライトを,強調表示した音符(開始:赤,終了:青,頂点:緑(候補は黄色))を描画する.
また,フレーズの開始位置上部に発想標語のテキストを追記する.
これにより,ユーザはどの範囲に・どのような演奏表情の指示を与えたかを一目で把握することが可能となる.

\noindent \textbf{(3) 操作履歴の管理}

これらの指示内容は,操作履歴としてリスト構造で管理される.
各履歴には,「フレーズ範囲」「選択された頂点」「適用されたプリセットのパラメータ」が記録されており,これに基づきMIDI加工が行われる.
また,この履歴構造を持つことで,操作の取り消し(Undo)や,やり直し(Redo)といった試行錯誤を支援する機能を実現している.


% -4.3.3 頂点候補の計算 - 
\subsubsection{頂点候補の計算}
サーバ側の\texttt{app.py}では,ユーザが指定したフレーズ範囲内の全ての音符に対し,第\ref{sec:Design}章で定めた頂点推定ルール(\tabref{tab:table2})に基づき,スコア計算を行う.
各ルール(音価の長さ,跳躍の有無など)に合致するごとに所定のポイントを加算し,最終的にスコアが最も高い音符を「頂点候補」として抽出する.
抽出された候補はフロントエンドに送信され,画面上の楽譜において該当する音符が強調表示される.

% -4.3.4 MIDIデータの加工処理
\subsubsection{MIDIデータの加工処理} \label{sec:Processing}
確定したフレーズ範囲と頂点情報,および選択されたプリセットに基づき,MIDIデータを加工する.

\noindent \textbf{(1) Expression(CC2)の線形補間} 

本システムでは,フレーズの頂点に向かって音量を増加させ,頂点通過後は音量を減少させることで,自然な抑揚を表現する.
現在の時刻(Tick)を$t$,フレーズの開始時刻を$t_{start}$,頂点の時刻を$t_{peak}$,終了時刻を$t_{end}$とする.
また,それぞれの時点でのCC2目標値(音量)を$V_{base}$(開始・終了値),$V_{peak}$(頂点値)とする.

このとき,時刻$t$における CC2値$E(t)$は,以下の区分線形関数として定義される.

\begin{equation}
	E(t) =
  	\begin{cases}
    	V_{base} + (V_{peak} - V_{base}) \cdot \frac{t - t_{start}}{t_{peak} - t_{start}} \\
    	\hfill (t_{start} \le t \le t_{peak}) \\
    	V_{peak} + (V_{base} - V_{peak}) \cdot \frac{t - t_{peak}}{t_{end} - t_{peak}} \\
    	\hfill (t_{peak} < t \le t_{end})
  	\end{cases}
\end{equation}

ここで,目標値$V_{base}$および$V_{peak}$は,元の演奏データから算出されたフレーズ範囲の平均CC2値$\overline{CC_{orig}}$と,選択された発想標語プリセットのパラメータ($P_{base},P_{peak}$)を用いて次の式のように決定される.

\begin{equation}
	\begin{aligned}
		V_{base} &= \overline{CC_{orig}} + P_{base} \\
		V_{peak} &= \overline{CC_{orig}} + P_{peak}
	\end{aligned}
\end{equation}

なお,算出された$E(t)$はMIDI規格に基づき,$0 \le E(t) \le 127$ の範囲に制限される.
この計算を1Tick単位で行うことで,滑らかな強弱変化を実現している.

\noindent \textbf{(2) Onsetの調整}

各音符の発音時刻(Onset)に対し,プリセットのパラメータ(\texttt{onset(ms)})に応じたオフセットを加算する.
本システムでは,単純に一律の遅延を与えるのではなく,フレーズの進行(経過拍数)に比例して変化量を増大させる線形モデルを採用している.

元の発音時刻(Tick)を$t$,補正後の発音時刻を$t'$とし,フレーズ開始点からの経過拍数を$\Delta B(t)$とするとき,制御式は以下のように定義される.

\begin{equation}
	t' = t + \alpha \cdot P_{onset} \cdot \Delta B(t)
\end{equation}

ここで,$P_{onset}$は1拍あたりの変化量(\texttt{onset(ms)}に対応)を表し,$\alpha$はミリ秒からTickへの変換係数である.
この式により,フレーズが進むにつれて遅延量(または前倒し量)が蓄積されるため,$P_{onset}$が正の場合は「重厚感(溜め)」,負の場合は「切迫感(走り)」といった効果が強調される.
そのため,楽曲データのテンポ設定(BPM)自体を変更せずに,体感上では局所的にテンポが揺らいでいるかのような表現を実現している.

% -4.4 UIと操作手順 -
\subsection{UIと操作手順}
ユーザが直感的に試行錯誤できるように,画面構成は左側の「操作パネル」と右側の「楽譜表示エリア」に分割した(\figref{fig:UI_main}).

% 図7
\begin{figure}[tb]
	\centering
	\includegraphics[width=\hsize]{../fig/UI_main.png}
	\caption{システム画面(左:操作パネル,右:楽譜表示エリア)}
	\label{fig:UI_main}
\end{figure}

本システムの操作手順を以下に示す.
\begin{enumerate}
	\item \textbf{ファイルの読み込みとパート選択}\\
		操作パネルからMusicXMLとMIDIファイルをアップロードし,編集したいパート(例:Trumpet)を選択すると,右側に五線譜が描画される.
		この際,原曲の再生も可能である.
	\item  \textbf{フレーズと頂点の指定}\\
		楽譜表示エリアにて,ユーザはフレーズの「開始音」と「終了音」を順にクリックする.
		すると,第\ref{sec:Rendering}項で述べた仕組みにより,システムが推定した頂点候補が黄色でハイライトされる(\figref{fig:UI_phrase}).
		ユーザは候補の中から自身の解釈に合う音符(あるいは任意の音符)をクリックし,頂点として確定させる.
	\item \textbf{発想標語の適用と調整}\\
		操作パネルのプルダウンメニューから「Cantabile」などの発想標語を選択する.
		「楽譜に反映」をクリックすると,楽譜上の該当箇所に発想標語のテキストが追記される.
		また,楽譜上の発想標語のテキストをクリックするとパラメータの設定値を確認することができる(\figref{fig:UI_notation}).
		さらに,パラメータ調整パネル(\figref{fig:UI_parameter})を展開することで,プリセット値(Base,Peak,Onset)をスライダーや数値入力で調整することも可能である.
	\item \textbf{音源生成と履歴管理}\\
		「音源を生成」をクリックすると,第\ref{sec:Processing}項の処理を経てWAVファイルが生成される.
		完了後,再生プレーヤーが表示され,加工前の演奏と加工後の演奏を聴き比べることができる.
		なお,納得がいかない場合は「一つ前に戻る(Undo)」ボタンで直前の状態に復帰できるほか,加工結果をMIDIファイルやPDF(楽譜)として保存することも可能である.
\end{enumerate}

% 図8
\begin{figure}[tb]
	\centering
	\includegraphics[width=\hsize]{../fig/UI_phrase.png}
	\caption{フレーズ範囲と頂点候補の表示}
	\label{fig:UI_phrase}
\end{figure}

% 図9
\begin{figure}[tb]
	\centering
	\includegraphics[width=\hsize]{../fig/UI_notation.png}
	\caption{プリセットを楽譜に反映}
	\label{fig:UI_notation}
\end{figure}

% 図10
\begin{figure}[tb]
	\centering
	\includegraphics[width=\hsize]{../fig/UI_parameter.png}
	\caption{パラメータ調整パネル(左部分)}
	\label{fig:UI_parameter}
\end{figure}

% -5. 評価 -
\section{評価} 

% -5.1 適用事例 -
\subsection{適用事例} 
本節では,実装したシステムを用いて実際の楽曲に演奏表現を付与した事例を示し,生成されたパラメータの変化とシステムの有効性について考察する.

% -5.1.1 同一フレーズに対するパラメータ変化の検証 -
\subsubsection{同一フレーズに対するパラメータ変化の検証} \label{sec:Evaluation}
まず,同一のフレーズに対して異なる発想標語を適用し,生成されるパラメータ(CC2およびOnset)の違いを定量的に検証する.

% (1)検証の概要
\noindent \textbf{(1) 検証の概要}

検証には,以下の楽曲を用いた.
なお,本検証ではMutopia Projectで公開されている楽譜データ\cite{Symphony9}を元に,メロディーパート(冒頭~18小節)を抽出し,Oboeパートとして編集したMusicXMLとMIDIデータを使用した.

\begin{itemize}
	\item \textbf{楽曲:}交響曲第9番「新世界より」第2楽章(A. Dvorak 作曲)
	\item \textbf{パート:}Oboe(原曲のEnglish Hornを読み替え)
	\item \textbf{対象フレーズ:}抜粋データの最後4小節(15小節~18小節)
\end{itemize}

% 図11
\begin{figure}[tb]
	\centering
	\includegraphics[width=\hsize]{../fig/Symphony9_phrase.png}
	\caption{対象フレーズ(15小節~18小節)}
	\label{fig:Symphony9}
\end{figure}

対象フレーズの楽譜を(\figref{fig:Symphony9})に示す.
このフレーズに対し,性格の異なる3つの発想標語「Cantabile(歌うように)」「Maestoso(堂々と)」「Con brio(生き生きと)」をそれぞれ適用した.
なお,フレーズの頂点は推定結果により,17小節4拍目と18小節目2拍目の音符が候補として表示されたが,前者の音符を選択した(\figref{fig:Symphony9_apex}).

% 図12
\begin{figure}[tb]
	\centering
	\includegraphics[width=\hsize]{../fig/Symphony9_apex.png}
	\caption{対象フレーズの範囲指定と頂点設定}
	\label{fig:Symphony9_apex}
\end{figure}

% (2)CC2値(強弱)の変化
\noindent \textbf{(2) CC2値(強弱)の変化}

元の音源のCC2値を\figref{fig:Symphony9_cc2}に,各発想標語を適用した際のCC2値の変化を\figref{fig:Symphony9_cc2_result}に示す.

% 図13
\begin{figure}[tb]
	\centering
	\includegraphics[width=\hsize]{../fig/Symphony9_cc2.png}
	\caption{適用前のCC2値の推移}
	\label{fig:Symphony9_cc2}
\end{figure}

% 図14
\begin{figure}[tb]
	\centering
	\includegraphics[width=\hsize]{../fig/Symphony9_cc2_result.png}
	\caption{各発想標語を適用した際のCC2値の比較}
	\label{fig:Symphony9_cc2_result}
\end{figure}

\figref{fig:Symphony9_cc2_result}より,適用した発想標語によって生成される強弱のカーブが大きく異なることが分かる.
cantabile(紫線)では変化幅が+35と中程度であり,旋律を自然に歌うような滑らかな変化となっている.
一方,Maestoso(緑線)では,Peak CC2が+50と高く設定されているため,頂点に向かって急激な音量の盛り上がりが確認できる.
Con brio(オレンジ線)も変化幅は大きいが,後述するOnset(発音時刻の制御)の違いにより,Maestosoとは全く異なる印象を与える.

% (3)Onset(発音時刻)の比較
\noindent \textbf{(3) Onset(発音時刻)の比較}

各発想標語を適用した際のメロディラインの発音時刻(Onset)と音高の推移を加工前のデータと比較したグラフを\figref{fig:Symphony9_onset}に示す.
横軸は時間(Tick),縦軸は音高(MIDIノート番号)を表す.
加工後の各色の線が,加工前の灰色線に対して時間軸上でどのように変化したかを見ることで,時間的な揺らぎを確認することができる.
加工後の線が灰色線よりも右側にプロットされている箇所は演奏が遅延している(溜めがある)ことを示し,左側にプロットされている箇所は前倒しになっている(走っている)ことを示す.

% 図15
\begin{figure}[tb]
	\centering
	\includegraphics[width=\hsize]{../fig/Symphony9_onset.png}
	\caption{発想標語によるOnset(発音時刻)の比較}
	\label{fig:Symphony9_onset}
\end{figure}

\figref{fig:Symphony9_onset}より,発想標語の持つ性格が時間的な揺らぎとして明確に可視化されていることがわかる.
Maestoso(オレンジ線)は,プリセット値(+40ms)に基づき,全体として元の演奏(灰色線)よりも大幅に右側へ推移している.
これは,堂々とした表現を意図した強い「溜め」が生成されていることを示している.
Cantabile(青線)は,プリセット値(+20ms)により,Maestosoよりも緩やかに右側へ推移しており,自然に歌うような穏やかな時間的揺らぎが表現されている.
一方,Con brio(緑線)は,プリセット値(-40ms)が負であるため,元の演奏よりも左側へ推移している.
これは生き生きとした演奏を意図した「前倒し」の効果であり,疾走感が生まれていることがわかる.

これらの時間的なずれは,フレーズが進むにつれて大きくなる傾向が見られる.
これは,第\ref{sec:Processing}項で述べた,経過拍数に比例して変化量を累積させる線形モデルが意図通りに機能していることを示している.

以上の結果から,本システムはExpression(CC2)による強弱制御だけでなく,Onsetによる時間制御においても,発想標語の持つニュアンスをパラメータとして適切に反映できているといえる.

% -5.1.2 楽曲全体を通じた演奏デザインの実践 -
\subsubsection{楽曲全体を通じた演奏デザインの実践}
第\ref{sec:Evaluation}項では単一フレーズにおけるパラメータ変化を検証したが,実際の楽曲演奏においては,曲の展開や繰り返しに応じて表情を変化させることが求められる.
そこで,本節では「G線上のアリア」\cite{Air}の楽曲全体に対し,場面ごとに異なる発想標語を適用することで,物語性のある演奏デザインを試みた.

\noindent \textbf{構成の意図と適用パラメータ}

本楽曲(全36小節)を音楽的な流れに基づいて4つのセクションに分け,主旋律であるFluteパートに以下のようにプリセットを適用した.
なお,本検証に用いたデータは,Mutopiaプロジェクトで公開されているものをもとに,筆者が本検証用に編集したものである.

\begin{enumerate}
	\item \textbf{前半(1~12小節):}
	\begin{itemize}
		\item \textbf{1~6小節:Cantabile(歌うように)}
		\begin{itemize}
			\item 意図:冒頭の主題を滑らかに歌わせる.
			\item 頂点:3小節1拍目
		\end{itemize}
		\item \textbf{7~12小節:Dolce(甘く)}
		\begin{itemize}
			\item 意図:反復部分でより繊細な表現とする.
			\item 頂点:9小節1拍目
		\end{itemize}
	\end{itemize}
	\item \textbf{中間(13~24小節):}
	\begin{itemize}
		\item \textbf{13~18小節:Appassionato(情熱的に)}
		\begin{itemize}
			\item 意図:音域が上昇し感情が高まる部分とする.
			\item 頂点:17小節3拍目
		\end{itemize}
		\item \textbf{19~24小節:Sostenuto(音を十分に保って)}
		\begin{itemize}
			\item 意図:クライマックスに向け,音を十分に保ちながらエネルギーを溜める.
			\item 頂点:21小節2.75拍目
		\end{itemize}
	\end{itemize}
	\item \textbf{クライマックス(25〜30小節):}
  	\begin{itemize}
    	\item \textbf{25~30小節:Maestoso(堂々と)}
    	\begin{itemize}
      		\item \textbf{意図:} 楽曲の頂点として重厚感を表現する.
      		\item \textbf{頂点:} 26小節1.5拍目
    	\end{itemize}
	\end{itemize}
	\item \textbf{終結(31~36小節)}
  	\begin{itemize}
    	\item \textbf{31~36小節:Tranquillo(静かに)}
    	\begin{itemize}
      		\item \textbf{意図:} 曲の終わりに向かって静けさを表現する.
      		\item \textbf{頂点:} 34小節2.75拍目
    	\end{itemize}
	\end{itemize}
\end{enumerate}

本システムで上記の表情指示を適用した結果,生成された楽譜を\figref{fig:Air_score}に示す.
図中の発想標語およびフレーズ範囲のハイライト,頂点を示す緑色のマーカーは,全て本システムの機能によって自動的に描画されたものである.

また,一連の操作によって実際に生成された演奏音源(加工前・加工後)を,GitHub\cite{GitHub}にて公開している.
加工後の音源では,冒頭の「Cantabile」による滑らかな主題提示から,中間部の「Appassionato」による情熱的な盛り上がり,そして終結部の「Tranquillo」による静かな収束へと,楽曲の物語性が強調された演奏表現が実現されていることを聴覚的に確認できる.

このように,ユーザが楽曲全体の構造を解釈し,セクションごとに異なる表情をデザインすることで,一本の旋律から多様な音楽表現を生成できることが本システムの大きな特徴である.

% 図16
\begin{figure}[tb]
	\centering
	\includegraphics[width=\hsize]{../fig/Air_Score_flute.png}
	\caption{「G線上のアリア」に対する演奏表現の適用}
	\label{fig:Air_score}
\end{figure}

% -5.2 頂点推定における和声情報の重要性 - 
\subsection{頂点推定における和声情報の重要性} \label{sec:Harmonies}
本研究で実装した頂点推定機能は,単旋律のパート譜からも音楽的な頂点を推定できることを目指したものである.
しかし,音楽表現における頂点は,旋律の動きだけでなく,楽曲全体の和声進行によっても大きく影響される.

そこで本節では,本システムの頂点推定機能の有効性と,和声情報を考慮しないことによる限界を明らかにすることを目的とする.
分析対象として,和声的に豊かな表現を持つベートーヴェンのピアノソナタ第8番「悲愴」第2楽章\cite{Sonata8}とエルガーの「愛の挨拶」\cite{Salutdamour}を選定した.
それぞれの楽曲から特徴的な2つのフレーズを抽出し,(1)単旋律ルールのみを適用した本システムの出力結果と,(2)伴奏パートの和声情報を加味した音楽理論的な分析結果を比較・考察する.

% -5.2.1 分析事例1:ベートーヴェン「悲愴」第2楽章 -
\subsubsection{分析事例1:ベートーヴェン「悲愴」第2楽章}

\noindent \textbf{(1)フレーズ前半(1小節~4小節)}

まず,楽曲冒頭の主題を構成する1小節目から4小節目までを分析対象とする.
このフレーズを本システムに入力したところ,\figref{fig:sonata8_phrase1}に示すように,2小節目のE$\flat$5が頂点候補として提示された.

% 図17
\begin{figure}[tb]
	\centering
	\includegraphics[width=\hsize]{../fig/sonta8_phrase1.png}
	\caption{「悲愴」第2楽章(1~4小節)における頂点(青:システム,赤:和声情報あり)}
	\label{fig:sonata8_phrase1}
\end{figure}

本システムがこのE$\flat$5を頂点と判断したのは,ルールに基づき,フレーズ内で最も音価が長い点や,大きな上行跳躍の到達点であるという特徴を評価した結果である.
和声的に見ても,このE$\flat$5はフレーズ前半における1つの到達点として機能しており,旋律の自然な山と和声的な区切りが一致している.
そのため,この場合では単旋律のルールのみでも音楽的に妥当な頂点を推定できることが確認された.

\noindent \textbf{(2)フレーズ後半(5小節~8小節)}

続いて,前半から直接繋がる5小節目から8小節目までを分析する.
\figref{fig:sonata8_phrase2}の青い矢印で示すように,本システムは,複数のルールが複合的に適用された5小節目のD$\flat$5を頂点として提示した.

% 図18
\begin{figure}[tb]
	\centering
	\includegraphics[width=\hsize]{../fig/sonta8_phrase2.png}
	\caption{「悲愴」第2楽章(5~8小節)における頂点(青:システム,赤:和声情報あり)}
	\label{fig:sonata8_phrase2}
\end{figure}

しかし,このフレーズは8小節目で完全終止(全終止)する構成をとっており,音楽の構造上,最もエネルギーが高まるのは終止直前のドミナント(V\textsuperscript{7})である.
この強い緊張を生み出しているのが,\figref{fig:sonata8_phrase2}の赤い矢印で示す7小節目のG4である.
この音は,楽曲の終止感を決定づけるエネルギーの頂点であり,音高や音価では目立たない音であるが,和声的には極めて重要な役割を担っている.

% -5.2.2 分析事例2:エルガー「愛の挨拶」 -
\subsubsection{分析事例2:エルガー「愛の挨拶」}

\noindent \textbf{(1)主題冒頭(3小節~6小節)}

まず,楽曲を象徴する主題の冒頭部分(3~6小節)を分析対象とする.
\figref{fig:sault_phrase1}の青い矢印で示すように,本システムの単旋律ルールは,フレーズ内で最も音高が高く,上行跳躍によって到達する5小節目のA5を頂点として提示した.

% 図19
\begin{figure}[tb]
	\centering
	\includegraphics[width=\hsize]{../fig/sault_phrase1.png}
	\caption{「愛の挨拶」(3~6小節)における頂点(青:システム,赤:和声情報あり)}
	\label{fig:sault_phrase1}
\end{figure}

しかし,このフレーズの音楽的特徴を決定づけているのは,和声的な観点から見ると4小節目のF$\sharp$5である.
\figref{fig:sault_phrase1}の赤い矢印で示したこの音は,背景の和音に対する非和声音(倚音)として機能し,一時的に強い和声的な緊張を生み出した後に解決する.
この和声的な緊張の頂点は,音高や音価では目立たないため,単旋律ルールでは捉えることが困難であった.

\noindent \textbf{(2)クライマックス部分(11小節~14小節)}

一方で,主題が1度目のクライマックスを迎える11~14小節では,単旋律ルールと和声分析の結果が一致した.
\figref{fig:sault_phrase2}に示すように,システムは主題全体の最高音である13小節目のC$\sharp$6を頂点として正しく推定した.

% 図20
\begin{figure}[tb]
	\centering
	\includegraphics[width=\hsize]{../fig/sault_phrase2.png}
	\caption{「愛の挨拶」(11~14小節)における頂点(青:システム,赤:和声情報あり)}
	\label{fig:sault_phrase2}
\end{figure}

和声的に見ても,このC$\sharp$6は,楽曲のドミナント(V)に向かうエネルギーが強いセカンダリー・ドミナント(V\textsuperscript{7} of V)の上に乗っている.
この箇所では,旋律の頂点と和声的な緊張の頂点が一致しており,このような場合では本システムの単旋律ルールが有効に機能することが示された.

% -5.3 演奏音源に対する表情付けの定性的な再構築 -
\subsection{演奏音源に対する表情付けの定性的な再構築}

これまでの評価では,ユーザが自身の音楽的解釈に基づいて新たな演奏表現をデザインする事例を示した.
本節では,システムの応用的な有効性を検証するため,既存の演奏音源が持つ表現のニュアンスを,本システムを用いて再構築する試みについて述べる.
これは,演奏者がプロの演奏を聴き,その表現を模倣・分析する「耳コピ」に近いプロセスを,本システムがどの程度支援できるかを評価するものである.

% -5.3.1 目的と方法 -
\subsubsection{目的と方法}
本検証では,ユーザが目標とする演奏音源を聴き,その音楽的ニュアンスを本システム上で再構築するプロセスを通じて,本システムが演奏の分析・学習ツールとしてどの程度有効に機能するかを検証する.

検証には,G線上のアリアの冒頭6小節を対象とした(\figref{fig:Air_Score}).
この範囲は,楽曲の最も象徴的な主題を含み,1つの音楽的なまとまりとして完結しているため,詳細な表現分析を行うのに適している.

% 図21
\begin{figure}[tb]
	\centering
	\includegraphics[width=\hsize]{../fig/Air_Score.png}
	\caption{「G線上のアリア」冒頭6小節の楽譜}
	\label{fig:Air_Score}
\end{figure}

目標音源としては,Youtubeにて公開されている演奏\cite{AirYoutube}を参考とした.

具体的な手順は以下の通りである.

\begin{enumerate}
	\item \textbf{目標音源の聴取と分析:} 対象フレーズ(1~6小節)を繰り返し聴き,強弱のカーブやテンポの揺らぎといった表現上の特徴を,\figref{fig:Air_Score}に示すセグメントに分けて聴き取る.
	\item \textbf{システムによる再構築:} 聴き取った音楽的特徴を再現するため,セグメントごとにフレーズを指定し,プリセットの選択とパラメータの微調整を行う.
		また,主旋律(Fluteパートで代用)にのみ加工を施した.
	\item \textbf{結果の可視化と比較:} 再構築後のMIDIデータからCC2(強弱)およびOnset(発音時刻)のパラメータを抽出し,加工前のデータと比較・可視化することで,聴覚的な印象の変化を定量的に評価する.
\end{enumerate}

なお,本検証で生成した加工前および再構築後の演奏音源(MIDI)は,Github\cite{GitHub}にて公開している.

% -5.3.2 目標音源の分析とパラメータ調整 -
\subsubsection{目標音源の分析とパラメータ調整}

目標音源の聴取に基づき,冒頭6小節を音楽的な文脈から5つの短いフレーズの連なりとして解釈し,それぞれに対してパラメータ調整を行った.
各フレーズの音楽的な意図と,最終的に設定したパラメータ値を\tabref{tab:Air_params}に示す.
この調整により,単一のプリセットを適用するだけでは得られない,目標音源が持つ複雑で変化に富んだ音楽表現の再現を試みた.

% 表3
\begin{table*}[t]
    \centering
    \caption{目標音源の分析と再構築に用いたパラメータ設定}
    \label{tab:Air_params}
    \begin{tabular}{l|l|c|c|c|c} \hline
        \textbf{対象範囲} & \textbf{音楽的な意図・分析} & \textbf{プリセット(参考)} & \textbf{Base CC2} & \textbf{Peak CC2} & \textbf{Onset (ms)} \\ \hline \hline
        1〜2小節 & \shortstack[l]{穏やかに始まり,頂点で大きく盛り上がる.\\ ゆったりとしたテンポ感} & Maestoso & -30 & +50 & +120 \\ \hline
        3小節   & \shortstack[l]{頂点を過ぎるとすっと息を抜くように静かに入る\\ フレーズの終わりは消え入るように} & Tranquillo & 0 & +5 & +50 \\ \hline
        4小節   & \shortstack[l]{次の盛り上がりに向かう\\ 少しだけエネルギーを溜め,後半は落ち着く} & Sostenuto & -20 & -5 & +80 \\ \hline
        5小節   & \shortstack[l]{フレーズ全体のクライマックス\\ 推進力は失わない} & Appassionato & +25 & +60 & +30 \\ \hline
        6小節   & \shortstack[l]{クライマックス後の余韻\\ 落ち着きを取り戻し,大きくテンポを落として終結} & Sostenuto & +10 & +20 & +120 \\ \hline
    \end{tabular}
\end{table*}

% -5.3.3 再構築結果の比較と評価 -
\subsubsection{再構築結果の比較と評価}
\tabref{tab:Air_params}のパラメータ調整によって,再構築されたメロディラインと,加工前のメロディラインの発音時刻と音高の推移を比較したグラフを\figref{fig:Air_onset}に示す.
横軸は時間(Tick),縦軸は音高(MIDIノート番号)を表す.
加工後の青線が加工前の灰色線よりも右側にプロットされている箇所は,演奏が遅延している(溜めがある)ことを示し,両者を結ぶ赤色の矢印はその「ずれ」の大きさを示している.

% 図22
\begin{figure}[tb]
	\centering
	\includegraphics[width=\hsize]{../fig/Air_onset.png}
	\caption{加工前後のメロディラインにおける発音時刻(Onset)の比較}
	\label{fig:Air_onset}
\end{figure}

このグラフから,今回の再構築における時間表現の変化を明確に読み取ることができる.
グラフ全体を通して青線が灰色線の右側を推移している点は,目標音源が持つ全体的にゆったりとしたテンポ感を再現するために,一貫して正のOnset値(遅延)を適用した結果である.
しかし,この時間的なずらし方(テンポの揺らし方)は一様ではない.
\tabref{tab:Air_params}に示したように,音楽的文脈に応じてOnsetパラメータの値は大きく変化している.
例えば,楽曲のクライマックスである5小節目では,音楽的な推進力を失わないようにOnset値を+30msに抑えている.
それに対し,荘厳な導入(1~2小節)や,落ち着いた終結(6小節)では,+120msという大きな値を設定することで,「溜め」を生み出している.
このように,グラフに示された全体的な遅延は,音楽の物語性に応じて局所的に制御されたパラメータの集積であり,聴いた印象という定性的な分析を,本システムが定量的な数値として効果的に反映できていることを示している.

また,\figref{fig:Air_cc2}に示す強弱の変化と合わせることで,本システムの有効性がより明確になる.
1~2小節の大きなクレッシェンドと時間的な溜め,3小節目の静寂,そして5小節目のダイナミックな頂点と前進するエネルギーという,強弱と時間の両側面からの表情付けが,目標音源の持つ豊かな音楽性の再現に不可欠であった.

% 図23
\begin{figure}[tb]
	\centering
	\includegraphics[width=\hsize]{../fig/Air_cc2.png}
	\caption{再構築後のCC2値の推移}
	\label{fig:Air_cc2}
\end{figure}

以上の結果から,本システムは新たな演奏表現をデザインするだけでなく,ユーザが目標とする演奏を聴き,その音楽的ニュアンスをパラメータレベルで分析・再構築するための支援ツールとしても有効に機能することが示された.
これにより,演奏者が自身の表現を探求する上での実践的な手段を提供できると考えられる.

% - 6.考察 -
\section{考察}
本章では,本システムの有効性と,実装した頂点推定機能の評価について考察を行う.

% -6.1 発想標語と演奏パラメータの対応付け -
\subsection{発想標語と演奏パラメータの対応付け}
第\ref{sec:Evaluation}項の適用事例において,同一フレーズに対し,適用する発想標語を切り替えることで,生成される演奏表現が大きく変化することが示された.
これは,本システムが定義したプリセットが,各発想標語の持つ音楽的ニュアンスをCC2(強弱)とOnset(時間)という物理パラメータとして適切にモデル化できていることを示している.

特に,CC2による強弱表現だけでなく,時間軸上の微細な制御(Onset)を組み合わせた点が,豊かな演奏表現を実現する上で効果的であった.
例えば,「Maestoso」ではExpression(CC2)の大きな山に加え,Onsetの遅延が重厚感を生み出し,「Con brio」ではOnsetの前倒しが疾走感を生み出すなど,発想標語の持つ多面的な性格を再現することができた.
これは,Velocity制御中心のシステムでは表現が困難であった,管楽器などのロングトーンの表現の幅を広げる上で有効である.

また,本システムではフレーズの頂点をユーザが指定する設計とした.
これにより,同じ発想標語を適用する場合でも,頂点をフレーズの前半に置くか後半に置くかで強弱の山の形が変化し,演奏のニュアンスをさらに細かく制御することができた.
これは,音楽表現が「グループごとに一つの頂点を持ち,その頂点に向かって強弱が増減する」という,保科理論の基本原則を実践的に検証するものであったといえる.

% -6.2 頂点推定機能の評価と課題 -
\subsection{頂点推定機能の評価と課題} \label{sec:ApexIssue}
楽曲を表情豊かに演奏するためには,フレーズの構造,特にその頂点をどこに置くかを理解することが重要である.
しかし,楽器の演奏経験が浅い学習者にとって,楽譜から自力で頂点を読み解くことは必ずしも容易ではない.
本システムの頂点推定機能は,このような楽曲解釈を行う際の指針を提供することを目的として実装した.

第\ref{sec:Harmonies}節で行った分析に基づき,本研究で実装した単旋律に基づく頂点推定機能の有効性と課題が明らかになった.
\tabref{tab:apex_summary}に結果をまとめる.

% 表4
\begin{table}[tb]
    \centering
	\fontsize{7pt}{10pt}\selectfont
    \caption{頂点推定における単旋律ルールと和声分析の比較}
    \label{tab:apex_summary}
    \begin{tabular}{l|l|c|c} \hline
        \textbf{楽曲} & \textbf{フレーズ} & \textbf{推定結果} & \textbf{不一致の要因} \\ \hline \hline
        \multirow{2}{*}{悲愴} & 1~4小節  & 一致  & -- \\ \cline{2-4}
                                    & 5~8小節  & 不一致 & カデンツ(終止形)の緊張 \\ \hline
        \multirow{2}{*}{愛の挨拶}  & 3~6小節  & 不一致 & 倚音(非和声音)による緊張 \\ \cline{2-4}
                                    & 11~14小節 & 一致   & -- \\ \hline
    \end{tabular}
\end{table}

分析の結果,旋律の動きと和声の構造が一致している場合(「悲愴」1~4小節,「愛の挨拶」11~14小節)においては,本システムは単旋律ルールのみでも音楽的に妥当な頂点を推定できることが確認された.
しかし,カデンツ(終止形)のような強い音楽的緊張を持つフレーズ(「悲愴」5~8小節)や,倚音に代表される非和声音がもたらす和声的な緊張(「愛の挨拶」3~6小節)が用いられる場合,単旋律の情報だけでは音楽の構造的な関係性を捉え切れない限界があることが明確になった.

これは,本アルゴリズムが和声的な文脈を考慮していないことが原因だと考えられる.
例えば,音高や音価では目立たない音であっても,強拍に置かれた非和声音(倚音など)が和声的な緊張感から頂点として判断されることがある.
現在のシステムは単旋律の情報のみを扱っているため,このような和声進行に伴う音楽的な重みを判断することが困難である.

本システムの頂点推定は,完全な正解を提示するものではなく,あくまでユーザの解釈を支援するための機能である.
そのため,システムが提示した頂点候補を参考にしつつも,最終的な決定権はユーザに委ね,ワンクリックでの選択や候補以外の音符の指定を可能にした.
このようなUI設計は,演奏者による主体的な解釈の構築を支援するという本研究の目的に合致したアプローチであるといえる.

% -6.3 持続音楽器を考慮した階層的フレーズ表現 -
\subsection{持続音楽器を考慮した階層的フレーズ表現}
保科理論で示されるように,音楽はグループやフレーズといった階層的な構造を持つ.
この階層性は演奏表現においても同様であり,例えば「楽曲全体をある性格で演奏しつつ,その中の特定の区間のみを異なるニュアンスで際立たせる」といった,入れ子構造の表現意図を演奏者は持つことがある.
本システムでは,ユーザによる複数回の加工を行うことで,この階層的な表現意図を実現することが可能である.
具体的には,一度表現付けを行ったMIDIデータを保存し,それを新たな入力データとして再度読み込ませることで,ベースとなる表現の上に,より局所的な表現の再加工を行うことができる.

本節では,「G線上のアリア」の冒頭6小節の主旋律パート(\figref{fig:Air_Phrase_flute})を例に,ベースとなる大きな表現の流れの中に,性格の異なる複数の局所的な表現を段階的に組み込むことで,より多層的で物語性のある演奏デザインを試みる.

% 図24
\begin{figure}[tb]
	\centering
	\includegraphics[width=\hsize]{../fig/Air_Phrase_flute.png}
	\caption{「G線上のアリア」冒頭6小節の主旋律パート}
	\label{fig:Air_Phrase_flute}
\end{figure}

% -6.3.1 適用手順と音楽的意図 -
\subsubsection{適用手順と音楽的意図}
本検証では,楽曲の表現意図を段階的に構築するプロセスとして,以下の手順でデータの再加工を行った.

\begin{enumerate}
	\item \textbf{ベース表現の適用:}
		まず,フレーズ全体(1~6小節)に対し,繊細で柔らかなベース表現として「Dolce」を適用し,そのMIDIデータを一度書き出す.
	\item \textbf{局所的表現の再加工:}
		次に,書き出したMIDIデータを再度システムに読み込ませ,以下の2つの表現の再加工を行った.
		\begin{itemize}
			\item 3~4小節:Dolceの静けさからクライマックスへ繋ぐため,「Cantabile」を適用した.
			\item 5小節:フレーズのクライマックスである5小節目に対し,情熱的な高まりを表現するため「Appassionato」を適用した.
		\end{itemize}	
\end{enumerate}

% -6.3.2 CC2パラメータの変化と考察 -
\subsubsection{CC2パラメータの変化と考察}
上記のプロセスによって生成されたCC2値の推移を段階ごとに比較したグラフを\figref{fig:Air_cc2_result}に示す.
グラフ中の緑線は加工前のCC2値,オレンジ線はベースとなる「Dolce」を適用した後のCC2値,青線は最終的な階層的表現を適用した後のCC2値の推移をそれぞれ表している.

% 図25
\begin{figure}[tb]
	\centering
	\includegraphics[width=\hsize]{../fig/Air_cc2_result.png}
	\caption{階層的表現の適用によるCC2値の変化の比較}
	\label{fig:Air_cc2_result}
\end{figure}

\figref{fig:Air_cc2_result}は,本アプローチの有効性を明確に示している.
加工前の平坦なCC2値(緑線)に対し,まずフレーズ全体に「Dolce」を適用することで,全体が抑えられた緩やかな山型のカーブ(オレンジ線)が形成される.
次に,このオレンジ線を新たな基準値として局所的な加工が行われる.
最終的な出力である青線を見ると,ベースラインの上に3~4小節目の「Cantabile」による緩やかな盛り上がりが現れ,その流れを引き継ぐ形で,5小節目には「Appassionato」によるダイナミックなピークが形成されていることがわかる.

ここでは,後から適用された「Cantabile」や「Appassionato」のカーブが,元のMIDIデータのCC2値を基準にするのではなく,先行して適用された「Dolce」のカーブの値を新たな基準として再計算・生成されている.
これにより,大きなフレーズの流れを汲みつつ部分的な性格を変化させるという,音楽的に自然な階層性がパラメータレベルで実現されている.

この結果は,本システムがユーザによる段階的な加工プロセスを通じて,単一のプリセットでは実現できない,多層的な楽曲の表現意図を構築できることを示している.
演奏者は,大きなフレーズの全体像(Dolce)を描いた後,その中の特定の区間に対して「少し歌わせる(Cantabile)」「情熱的に盛り上げる(Appassionato)」といった,複数の異なるレベルの表情付けを施すことができる.
これは,楽曲の物語性を深く解釈し,それを具体的なパラメータとして具現化する創造的なプロセスであり,本システムの有効性を強く示すものである.

% -6.4 今後の課題と展望 -
\subsection{今後の課題と展望}

本システムの実用性をさらに高め,より高度な演奏デザインを可能にするためには,機能面および応用面においていくつかの課題が挙げられる.

まず,頂点推定機能の高度化である.
第\ref{sec:ApexIssue}節の考察で述べたように,現在の頂点推定は単旋律の音価や音高のみに依存しており,和声的な文脈を捉えきれていない.
実際には,フレーズの頂点はメロディだけでなく,伴奏パートとの響きの関係性(協和・不協和)からも判断される.
そのため,この和声情報を無視している現状の仕様が,システムの解釈のずれを生む一因となっていた.
今後は,簡易的な和声分析機能をバックエンドに導入し,和声情報を含めたスコア計算を行うことで,より音楽の文脈に合った頂点候補を提示できるようアルゴリズムを改良する必要がある.

次に,より複雑な記譜情報への対応である.
現在のシステムでは,リピート記号などの楽譜の繰り返し記号を解釈できず,1回目の演奏にしか表現が適用されないという制限がある.
しかし,実際の演奏では1回目より2回目の方を盛り上げたいなど,繰り返しの中で表現を変化させることが行われる.
今後は,これらの繰り返し記号をシステムが認識し,ユーザが1回目と2回目で異なる発想標語を適用するといった,より実践的な演奏デザインを可能にするための機能拡張が必要である.

さらなる発展の可能性として,アンサンブル演奏への応用が挙げられる.
本システムは現在,1つのパート(単旋律)を編集することに特化しているが,吹奏楽やオーケストラといった合奏での音楽表現は,単一パートだけでは完結しない.
将来的には,複数のパート(スコア全体)を同時に表示・編集し,例えば,「主旋律のパートの頂点に合わせて,伴奏のパートの音量を少し下げる」といったパート間の相互作用を考慮した表現付けを支援する機能へと拡張することが期待される.
これにより,個々の奏者の視点だけでなく,合奏全体を俯瞰する指揮者のような視点での演奏デザインを探求できるようになるだろう.

% 7. おわりに
\section{おわりに}
本研究では,五線譜を見ながら直感的に演奏表現をデザインできる環境の構築を目的とし,Webブラウザ上で動作する支援システムを開発した.
管楽器などの持続音楽器の特性を考慮し,CC2とOnsetを組み合わせたパラメータ制御により,発想標語の持つニュアンスを反映した豊かな演奏表現を生成できることを確認した.

本研究の成果は,演奏者が自身の音楽的解釈を具体的な音として試行錯誤するプロセスを支援し,五線譜インターフェースとDAWの音響編集能力との間のギャップを埋める一つのアプローチを提示した点に意義がある.

一方で考察で述べたように,頂点推定における和声情報の欠如や,複雑な記譜への未対応といった課題も明らかになった.
今後はこれらの課題に取り組み,将来的にはアンサンブル演奏へ応用することで,本システムをより実践的な支援ツールへと発展させていくことが期待される.

本研究で開発したシステムが,多くの演奏者にとって,自身の音楽表現の可能性を探るための有効なツールとなることを期待する.

% - 謝辞 -
\begin{acknowledgment}
本研究を進めるにあたり,指導教員の橋田光代准教授には大変お世話になりました.
研究テーマに悩んでいた時期からいつも親身に相談に乗っていただき,本研究の完成まで温かくご指導くださいましたことに,心より感謝申し上げます.

また,研究室の4年生の皆さんには,日々の生活の中で数えきれないほど助けていただきました.
研究で行き詰まった時に話を聞いてくれたことや,何気ない雑談の時間が大きな励みになりました.
皆さんと過ごした時間がなければ,この研究を乗り越えることはできなかったと思います.

また,本システムの評価にご協力いただいた1年生や2年生の皆様にも,心より感謝申し上げます.
本当にありがとうございました.
\end{acknowledgment}
