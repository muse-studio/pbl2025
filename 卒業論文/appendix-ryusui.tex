\section{読み合わせ時の主なチェックリスト}

下記のほか,作文技術に関する 文献\cite{book1, book2, book3}のような書籍も参考になる.


\subsection{書き方の基本}

\begin{itemize}
	\item[$\Box$] 研究の新規性,有用性,信頼性が読者に伝わるように記述する.
	\item[$\Box$] 読み手に,読みやすい文章を心がける(内容が前後する,背景・課題の設定が不明瞭などは読者にとって負担).
	\item[$\Box$] 解決すべき問題が汎用化(一般的に記述)されていないのは再考を要する(XX大学の問題という記述に終始).あるいは,(単に「作りました」だけで)解決すべき問題そのものの記述がないのは再考を要する.
	\item[$\Box$] 結論が明確に記されていない,または,範囲,限界,問題点などの指摘が適切ではない,または,結論が内容にそったものではないものは再考を要する.
	\item[$\Box$] 科学技術論文として不適当な表現や,分かりにくい表現があるのは再考を要する.
	\item[$\Box$] 極端な口語体や,長文の連続などは再考を要する.
	\item[$\Box$] 章,節のたて方,全体の構成等が適切でない文章は再考を要する.
	\item[$\Box$] 文中の文脈から推測しないと内容の把握が困難な論文にしない.
	\item[$\Box$] 説明に飛躍した点があり,仮説等の説明が十分ではないのは再考を要する.
	\item[$\Box$] 説明に冗長な点,逆に簡単すぎる点があるのは再考を要する.
	\item[$\Box$] 未定義語を減らす.
\end{itemize}


%5.2
\subsection{新規性と有効性を明確に示す}

\begin{itemize}
	\item[$\Box$] 在来研究との関連,研究の動機,ねらい等が明確に説明されていないのは再考を要する.
	\item[$\Box$] 既知/公知の技術が何であって,何を新しいアイデアとして提案しているのかが書かれていないのは再考を要する.
	\item[$\Box$] 十分な参考文献は新規性の主張に欠かせない.
	\item[$\Box$] 提案内容の説明が,概念的または抽象的な水準に終始していて,読者が提案内容を理解できない(それだけで新規性が感じられないもの)のは再考を要する.
	\item[$\Box$] 論文で提案した方法の有効性の主張がない,またはきわめて貧弱なのは再考を要する.
\end{itemize}

%5.3
\subsection{書き方に関する具体的な注意}

\begin{itemize}
	\item[$\Box$] 和文標題が内容を適切に表現していないのは再考を要する.
	\item[$\Box$] 英文標題が内容を適切に表現していない,または英語として適切でないのは再考を要する.
	\item[$\Box$] アブストラクトが主旨を適切に表現していない,または英文が適切ではないのは再考を要する.
	\item[$\Box$] 記号・略号等が周知のものでなく,または,用語が適切でなく,または,図・表の説明が適当ではないのは再考を要する.
	\item[$\Box$] 個人的あるいは非常に小さなグループ/企業だけで通用するような用語が特別な説明もなしに多用されているのは再考を要する.
	\item[$\Box$] 図表自体は十分に明確ではない,または誤りがあるのは再考を要する.
	\item[$\Box$] 図表が鮮明ではないのは再考を要する.
	\item[$\Box$] 図表が大きさ,縮尺の指定が適切でないのは再考を要する.
\end{itemize}


%5.5
\subsection{適切な引用}

\begin{itemize}
	\item[$\Box$] 他の論文とまったく同じ図表を引用の明示なしに利用することは禁止.
	\item[$\Box$] 既発表の論文等との間に重複があるのは再考を要する.
\end{itemize}


%5.7
\subsection{その他}

\begin{itemize}
	\item[$\Box$] 投稿前にチェックリストの各項目を満たしているか,必ず確認する.
\end{itemize}


\section{おまけの雑感}
日本の大学生で,卒論を書くのは全体の8割ほどらしい\footnote{どこかのサイトに書いてあったが、どれだったかな…} .それだけいるなら,卒論の書き方に関する資料などは世の中(インターネット,書籍)いくらでも転がっているものである.なので,今さらわざわざ筆者がこんなものを書かなくても良いのではないだろうか.しかし思い返せば,筆者は音楽学部の出身で,26歳で博士進学するまでおよそ論文指導などというものは受けたこともない.ついでに,進学先には同居する院生が他におらず,そもそも現役の博士課程学生の日常,思考,“(ある意味での)常識”に触れる機会がなかった(出身大でも博士課程はなかった).それでもとりあえず,書店に行って「卒論の書き方」本などを読んでみるし,一応理解したとは思うのだが(文章を書くこと自体は決して嫌いではなかった,はず),いざ自分のテーマで書こうとすると,それらの「サンプル」はやっぱり抽象的だし,自分の場合に当てはめようとしてもどう書いていけばいいか結局わからない.先行研究の文章を参考にせよと言われるも,結局は本文コピペになってしまう.…という悩みすら,思い切って指導教員に言ってみると「論文読んでない(足らない)から」「国語力,語彙力がない」「小学生レベル」とまあおよそ滅多斬りにされて打ちのめされて終わってきたのである....

せめてもう少し自分のテーマに近く,かつ実例に近いサンプルが欲しいものである.誰か作ってほしい.今の世ならchatGPTにお世話になろうか.




