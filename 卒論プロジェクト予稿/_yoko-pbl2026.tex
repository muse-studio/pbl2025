\documentclass[dvipdfmx,line_length=48zw,column_gap=2zw,number_of_lines=60,baselineskip=12pt]{jlreq}
\jlreqsetup{itemization_beforeafter_space=0pt}
\makeatletter
\RenewBlockHeading{section}{1}{font={\jlreq@keepbaselineskip{\normalsize\sffamily\gtfamily}},indent=0pt,lines=1}
\RenewBlockHeading{subsection}{2}{font={\jlreq@keepbaselineskip{\normalsize\sffamily\gtfamily}},indent=0pt,lines=1}
\RenewBlockHeading{subsubsection}{3}{font={\jlreq@keepbaselineskip{\normalsize\sffamily\gtfamily}},indent=0pt,lines=1}
\makeatother
\pagestyle{empty}
% ↑この上の部分は変更しない!
% (LuaTeXを使う場合は \documentclass[dvipdfmx,... の dvipdfmx, を消す)
%%%%%%%%%%%%%%%%%%%%%%%%%%%

%%↓必要なパッケージを追加してください.
\usepackage{mathtools,amssymb}
\usepackage{latexsym}
\usepackage{newtxtext,newtxmath}
\usepackage[utf8]{inputenc} 	%波ダッシュ(〜)を表記できるようにする
%%%%%%%%%%%%%%%%%%%%%%%%%%%

\begin{document}
\twocolumn[
    {\Large
            \begin{center}
                %%↓タイトルを入力してください
                % 卒業論文のタイトルを書く
%EMDA簡略化の為のツール作成
% EMDA分析を支援する構造可視化ツールの設計と実装
EMDA分析簡略化のためのインタラクティブ編集ツールの設計

%→ 「なぜ作ったか」「何を支援するか」が前面に

                %%%%%%%%%%%%%%%%%%%%%%%%%%%
            \end{center}
        }
    \vspace{12pt}
    \begin{flushright}
        福知山公立大学情報学部 32345030 金丸陽香 \\
        指導教員 橋田光代
    \end{flushright}
    \vspace{1\Cvs}
]

%% ここから本文を書く
\input{muselab-yoko-sample-pbl2026}


% 参考文献
<<<<<<< HEAD:muselab-pblproject-template-main/_yoko-pbl2026.tex
\bibliographystyle{muselabunsrt} % bstファイルの名前
\bibliography{2026sankou} % .bibファイルの名前(Zoteroからエクスポートして作る)
=======
\bibliographystyle{../muselabunsrt} % bstファイルの名前
\bibliography{../muselab-sample} % .bibファイルの名前(Zoteroからエクスポートして作る)
>>>>>>> ce78ee9a86767c2bb557e2f6b8c327a82047aa91:卒論プロジェクト予稿/_yoko-pbl2026.tex

\end{document}
