\documentclass{article}
\usepackage[utf8]{inputenc}
\title{卒論メモ}
\date{}
\begin{document}
    \maketitle
    \noindent
    ◯第1場面

    \noindent
    ・最初の場面では、鳥たちが何やら騒いでいる(3人称視点)→1人称での鳥の視点、という形にした
    \vskip\baselineskip
    \noindent
    ・遠くからウサギがやってきて、主人公の様子を匂いを嗅いで確認する
    \vskip\baselineskip
    \noindent
    ・パニックになっている感じを出すため、ピッチをあげて鳥の鳴き声を甲高い声にした

    \vskip\baselineskip
    \noindent
    ・3人称の場面の1番最初は、ざわついた感じ、混沌とした感じを出すために、4種類の鳥の声を使った

    \vskip\baselineskip
    \noindent
    ・3人称視点のときはリバーブを多めにかけた

    \vskip\baselineskip
    \noindent
    ・1人称では、激しい風の音を追加し、すぐ近くで風を切っている音を感じられるようにして1人称視点であることがわかるようにした
    ・3人称から1人称の切り替えでは、同じ嵐という状況の中で、

    \vskip\baselineskip
    \noindent
    ・一人称視点のときはリバーブを小さく、そして風を切る音を加えてより主人公の鳥が感じているであろう音を再現するようにした

    \vskip\baselineskip
    \noindent
    ・主人公の鳥が仲間を探して鳴く→少し先で仲間がこちらに向かって鳴く→主人公が仲間のところに向かって速度をあげて向かう→障害物に衝突

    \vskip\baselineskip
    \noindent
    ・鳥が障害物にぶつかり墜落する場面では、鳥が墜落する様子を強調するために、鳥の声にディレイを加えた
    
    \vskip\baselineskip
    \noindent
    ◯第2場面

    \noindent
    ・地上に落ちたところの場面で、まだ雨が降っている

    \vskip\baselineskip
    \noindent
    ・遠くからウサギがやってきて、主人公の様子を匂いを嗅いで確認する

    \vskip\baselineskip
    \noindent
    ・遠くからウサギがやってきて、主人公の様子を匂いを嗅いで確認する



\end{document}