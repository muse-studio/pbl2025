% !TEX root = _yoko.tex

% ==========================================
% プロジェクト予稿(1P)の本文
% ==========================================

%1
\section{はじめに}

管楽器演奏においてヴィブラートは音楽表現を豊かにする重要な要素である一方,その習得は容易ではない。
どのような揺れ方がどのような表現につながるのかを具体的に理解することが難しい。
特に教育現場では,模倣や口頭説明に依存する指導が中心となり,
学習者が自ら比較・試行しながら理解するための手がかりが十分に与えられていない。

一方,電子音楽や歌声合成の分野では,ヴィブラートは音高変動として数理的に定義され,
精密な再現や最適化の対象として扱われてきた。
しかし,これらの研究は必ずしも演奏者の理解や学習を直接の目的としたものではない。

そこで本研究では,ヴィブラートを「正解を当てるべき音響現象」や「最適化すべき対象」としてではなく,
演奏者,特に初心者が違いを理解し,学習するための認知的対象として捉える。
本研究では楽器演奏におけるヴィブラート表現に着目し,その実態を整理分析するとともに,
ヴィブラートを音響的に厳密に再現することよりも,演奏者が理解・学習が可能で,
かつコンピュータ上で扱いやすい表現モデルの構築を目指す.

なお,本研究でいう「初心者」とは,
管楽器の基礎的な演奏経験および楽譜読解能力は有しているものの,
ヴィブラートを意図的かつ安定して用いる経験が十分でない演奏者を指す.

%2
\section{ヴィブラートの定義と本研究の位置づけ}

ニューグローブ音楽辞典\cite{Niyugurovu}において,ヴィブラート(vibrato)とは,
「表現性を強めるため,多少とも急速かつ微細に音高を変動させることをいう.」と記載されている.
この定義から,ヴィブラートは音の高さ(音高)を中心とした周期的な変動によって生じる効果であり,
主として声楽および多くの旋律楽器において用いられる表現技法である.
音高変動の幅や周期は楽器や演奏様式によって異なり,歴史的にもその使用法や評価は一様ではない。
すなわち,ヴィブラートは単一の正解を持つ固定的な表現ではなく,多様な形態を許容する音楽的表現である。

本研究ではこの点に着目し,ヴィブラートを物理的に正確に再現する対象としてではなく,
演奏者が違いを把握しやすい形に抽象化された表現として扱う.
そのため,音高変動をセント(cents)単位で記述し,変動幅や周期,
形状といった要素を独立に操作可能な表現モデルを構築する.

%3
\section{提案システムの概要}



%4
\section{おわりに}

本研究では,ヴィブラートに関し調査し,演奏者の理解・学習するための認知的対象として捉え,
その表現モデルの構築を目指した.


