% !TEX root = _main.tex
\begin{table*}[b]
    \caption{抽出した特徴量一覧}
    \label{tab:features}
    \centering
    % \renewcommand{\arraystretch}{2.1} % ← 1.5倍に広げる(1.2〜1.8くらいで調整)
    \begin{adjustbox}{width=\linewidth} % adjustboxで幅を100%に指定
        \begin{tabular}{lllll}\hline
            特徴量              & 意味               & どの性質を表すか         &  & \\\hline\hline
            measures         & 小節数              & 曲の長さ             &  & \\
            avg\_chord\_size & 平均和音サイズ(同時発音数)   & 和声の厚み            &  & \\
            seventh\_ratio   & 7th和音の割合         & 和声の複雑さ           &  & \\
            pitch\_mean      & 平均音高             & 全体の音域の中心         &  & \\
            pitch\_std       & 音高の標準偏差          & メロディの動きの大きさ      &  & \\
            pitch\_range     & 最高音−最低音          & 音域の広さ            &  & \\
            leap\_rate       & 跳躍進行の割合          & メロディの飛び方の多さ      &  & \\
            smoothness       & 音程差の逆数的な指標       & 滑らかさ             &  & \\
            up\_rate         & 上行音程の割合          & メロディの方向性(上昇か下降か) &  & \\
            \\
            \shortstack{dur\_whole/half/quarter/                        \\eighth/sixteenth} & 各音価の出現比率         & リズムの構成(長音中心か短音中心か) &  &  \\
            vel\_mean        & 平均ベロシティ(強さ)      & 曲の平均的な音量         &  & \\
            vel\_std         & ベロシティの変動         & 強弱表現の豊かさ         &  & \\
            prog\_unique     & ユニークな和音進行数       & ハーモニーの多様性        &  & \\
            rn\_seq\_len     & ローマ数字解析による進行列の長さ & 和声進行の規模          &  & \\\hline
        \end{tabular}
    \end{adjustbox}
\end{table*}