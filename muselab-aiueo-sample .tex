% !TEX root = _main.tex
% ========================================
% 卒業論文 本文
% ========================================
\section{はじめに}
%1.1
\subsection{研究背景}
舞台演出における照明は、音楽雰囲気を“見える形”に変えてくれる存在であり、観客の感情にダイレクトに作用する重要な要素だといえる。今のコンサートやライブパフォーマンスでは、照明はもはや脇役ではなく、演奏と一体となって「体験」をつくりあげる中心的な要素になっている。特にポピュラー音楽やクラシック音楽の演奏においては、曲の構造や演奏者の動き、さらに即興的な表現までもが照明と結びつくことで、観客は深い没入感を味わうことができる。

従来の舞台照明は、オペレーターが手動で操作するか、事前に仕込んだキュー(cue)に従うのが一般的だった。この方式は大規模公演では安定する一方で、演奏中に生まれる即興的な変化には柔軟に対応しにくい。その結果、音楽のダイナミクスと照明演出との間にズレが生じることもある。また、小規模なライブや個人のパフォーマンスでは、照明オペレーターを雇うコストや人手の問題から、十分な演出ができない場合も少なくない。

こうした課題を背景に、最近では音響信号処理やコンピュータ性能の進歩によって、音声の特徴を解析し、照明や映像を自動でコントロールする研究が進んでいる。たとえば、音の強さを示すRMS(Root Mean Square)、音の変化を捉えるZCR(Zero Crossing Rate)、音色の特徴を表すMFCC(Mel Frequency Cepstral Coefficients)、そしてテンポ推定などの指標は、曲の勢いやリズムを数値化するのに有効だと知られている。これらを用いれば、演奏に合わせたリアルタイムの照明制御も可能になる。

さらに、機械学習の発展によって、音響特徴量と視覚的な演出との関係をモデル化できるようになった。これまでは照明デザイナーの経験に頼って決められていたルールを、データに基づいて獲得できるようになり、より柔軟で高精度な演出が実現可能になっている。こうした背景から、本研究では音響信号処理と機械学習を組み合わせた、新しい舞台演出システムを提案する。
%2
\section{GUIシステムの設計}


%2.1
\subsection{システム全体の概要}

本研究では、読み込んだ映像に対するシーン分
割をもとに、効果音やBGMを簡単に付加できるGUI
システムを構築した。以下に、ユーザが行う操作と
システムが実行する処理を区別して説明する。


%2.1.1
\subsubsection{ユーザが行う操作}
\begin{enumerate}
	\item[(1)]動画ファイルのアップロード
\end{enumerate}

ユーザは、無音または既存の音声が含まれた動画ファイルをアップロードする。 

\begin{enumerate}
	\item[(2)]シーンの分割確認
\end{enumerate}

シーン分割後のプレビューを確認し、シーンの内容をチェックする。 

\begin{enumerate}
	\item[(3)]効果音の検索と選択
\end{enumerate}

キーワードを入力して効果音を検索し、表示されたリストから音源を選択する。検索結果を試聴し、挿入する効果音を決定する。

\begin{enumerate}
	\item[(4)]効果音の配置
\end{enumerate}

分割された各シーンに対し、効果音を選択操作で配置する。

\begin{enumerate}
	\item[(5)]プレビューの確認
\end{enumerate}

編集内容をプレビューで確認し、必要に応じて配置を調整する。 

\begin{enumerate}
	\item[(6)]動画のエクスポート
\end{enumerate}

完成した動画をエクスポートボタンを押して保存する。 


%2.1.2
\subsubsection{システムが行う処理}
\begin{enumerate}
	\item[(1)]動画の読み込みとシーン分割 
\end{enumerate}

アップロードされた動画を解析し、映像フレー
ムの特徴量を基に自動的にシーンを分割する。各
シーンの開始時刻、終了時刻を記録し、リストとし
てGUIに表示する。 

\begin{enumerate}
	\item[(2)]シーン分割情報のHTML出力
\end{enumerate}

シーンごとの開始時刻や終了時刻などの情報をHTML形式で出力する。

\begin{enumerate}
	\item[(3)]既存音声の無音化
\end{enumerate}
 
動画に既存の音声トラックがある場合、FFmpegを用いて音声トラックを無効化し、映像を無音化
する。

\begin{enumerate}
	\item[(4)]効果音データベースの検索 
\end{enumerate}

ユーザが入力したキーワードに基づき、事前に
用意されたCSV形式の効果音データベースを検索
し、一致する効果音のリストを表示する。 

\begin{enumerate}
	\item[(5)]効果音の再生 
\end{enumerate}

ユーザがリスト内の効果音をクリックすると、
その音源をPygameを用いて再生する。 

\begin{enumerate}
	\item[(6)]編集結果のプレビュー生成 
\end{enumerate}

効果音の配置情報を基に、編集内容を反映した
動画のプレビューを作成、再生する。

\begin{enumerate}
	\item[(7)]編集結果の動画エクスポート  
\end{enumerate}

効果音を挿入した後の動画をFFmpegを用いてエ
クスポートする。

%2.2
\subsection{使用した技術}
システムの実装には主に以下の3つの技術を用
いた(表1)。 
\begin{table}[htbp]
	\centering
	\begin{tabular}{|p{4.19cm}|p{3.29cm}|}
	  \hline
	  プログラミング言語 & Python\cite{python}  \\
	  \hline
	  GUI構築ライブラリ &Tkinter\cite{tkinter} \\
	  \hline
	  映像処理ツール & FFmpeg\cite{ffmpeg}   \\
	  \hline
	\end{tabular}
	\caption{システム実装に使用した技術}
 \end{table}

%2.3
\subsection{実装の詳細}
本節では、GUIシステムの具体的な実装プログラ
ムコードについて順に解説する。 

%2.3.1
\subsubsection{使用ライブラリのインポート}
\begin{figure}[H] 
 \includegraphics[width=0.48\textwidth]{1.png}  % ← 画像ファイル名(拡張子つき)
 \caption{使用ライブラリのインポートプログラム}
\end{figure}
\begin{enumerate}
	\item[(1)]pandas  
\end{enumerate}

効果音データベースの管理と検索を行うために
使用した。CSV形式で保存された効果音データを読
み込み、検索機能を実現するために利用する。

\begin{enumerate}
	\item[(2)]tkinter
\end{enumerate}

GUIの構築に使用した。ユーザーが直感的に操作できるインターフェースを提供する。

\begin{enumerate}
	\item[(3)]threading
\end{enumerate}

複数の処理を同時に実行し、システム全体の応
答性を向上させるために利用している。 

\begin{enumerate}
	\item[(4)]pygame
\end{enumerate}

効果音の再生機能を実装するために使用した。
音声の再生を簡便に行えるライブラリであり、リ
アルタイムで音声を再生し、ユーザーが選択した
効果音を確認できる。 

\begin{enumerate}
	\item[(5)]subprocess
\end{enumerate}

映像処理にFFmpegを呼び出すために使用した。FFmpegを利用することで、動画の編集や音声の挿
入が可能となる。 

\begin{enumerate}
	\item[(6)]re(正規表現ライブラリ) 
\end{enumerate}

映像処理のログからシーンの開始時刻を抽出す
るために利用している。

\begin{enumerate}
	\item[(7)]webbrowser 
\end{enumerate}

Pythonプログラム内からユーザーの既定のウェ
ブブラウザを開き、指定したURLを表示するため
に利用している。

%2.3.2
\subsubsection{GUIのレイアウト設計}
\begin{figure}[H] 
	\includegraphics[width=0.48\textwidth]{2.png}  % ← 画像ファイル名(拡張子つき)
	\caption{GUIのレイアウト設計プログラム }
   \end{figure}
本GUIシステムでは、主要な操作パネルやデータ表示領域を分離するために5つのFrameが設置
されている。 

\begin{enumerate}
	\item[(1)]Frame1 
\end{enumerate}

X座標80px、Y座標35pxの位置に幅1380px、高 
さ100pxの領域が設置されている。システムのタ
イトルや基本情報の表示領域として利用される。

\begin{enumerate}
	\item[(2)]Frame2 
\end{enumerate}

X座標80px、Y座標200pxの位置に幅250px、高
さ330pxの領域が設置されている。動画のアップ
ロードや再生、シーン分割などを行う領域である。

\begin{enumerate}
	\item[(3)]Frame3
\end{enumerate}

X座標400px、Y座標200pxの位置に幅300px、
高さ550pxの領域が設置されている。シーンリス
トの表示やシーン動画の再生、シーンに対する音
付けを行う領域である。

\begin{enumerate}
	\item[(4)]Frame4
\end{enumerate}

X座標780px、Y座標200pxの位置に幅300px、
高さ550pxの領域が設置されている。効果音の検
索や再生を行う領域である。

\begin{enumerate}
	\item[(5)]Frame5
\end{enumerate}

X座標1160px、Y座標200pxの位置に幅300px、
高さ550pxの領域が設置されている。効果音付き
のシーンの再生や、音付けが完了した動画のエク
スポートなどを行う領域である。

%2.3.3
\subsubsection{動画ファイルのアップロード}
\begin{figure}[H] 
	\includegraphics[width=0.48\textwidth]{3.png}  % ← 画像ファイル名(拡張子つき)
	\caption{動画ファイルのアップロードプログラム}
   \end{figure}
\begin{enumerate}
	\item[(1)]ファイル選択ダイアログを通じて動画ファイルを選択する。
\end{enumerate}

\begin{enumerate}
	\item[(2)]ファイルが正常に選択された場合、アップロード成功のメッセージを通知する。 
\end{enumerate}

%2.3.4
\subsubsection{動画の無音化}
\begin{figure}[H] 
	\includegraphics[width=0.48\textwidth]{4.png}  % ← 画像ファイル名(拡張子つき)
	\caption{動画の無音化プログラム}
   \end{figure}
\begin{enumerate}
	\item[(1)]動画をアップロードしていない場合に警告を
	表示する。動画の無音化処理は計算負荷が高く、実
	行中にGUIがフリーズする可能性がある。そのた
	め、バックグラウンドで処理を実行するよう設計
	した。
\end{enumerate}

\begin{enumerate}
	\item[(2)]FFmpegを用いて入力動画ファイルから音声ト
	ラックを除去し、新しい無音化済み動画ファイル
	を生成する。処理が正常に完了した場合、成功メッ
	セージを表示する。 
\end{enumerate}

%2.3.5
\subsubsection{動画の再生}
\begin{figure}[H] 
	\includegraphics[width=0.48\textwidth]{5.png}  % ← 画像ファイル名(拡張子つき)
	\caption{動画の再生プログラム}
   \end{figure}
\begin{enumerate}
	\item[(1)]動画ファイルがアップロードされているかを
	確認する。アップロードされていない場合、警告を
	表示し、処理を中断する。
\end{enumerate}
\begin{enumerate}
	\item[(2)]os.startfileメソッドを用いて、システムの
	既定のメディアプレーヤーで動画ファイルを再生
	することにした。元々はGUI内にプレビュー画面
	を作成する予定であったが、GUIがフリーズしてし
	まうため、メディアプレーヤーで再生する手法を
	採用した。動画が正常に再生された場合は、再生の
	開始を通知する。 
\end{enumerate}

%2.3.6
\subsubsection{シーンの分割}
\begin{figure}[H] 
	\includegraphics[width=0.48\textwidth]{6.png}  % ← 画像ファイル名(拡張子つき)
	\caption{シーンの分割プログラム}
   \end{figure}
\begin{enumerate}
	\item[(1)]動画ファイルが選択されているかを確認した
	後、内部的なシーン分割処理を呼び出す。
\end{enumerate}

\begin{enumerate}
	\item[(2)]シーン間の変化が0.7以上の閾値を超える場
	合に、フレームを抽出する条件を示す。このフィル
	タにより、視覚的な大きな変化(カットなど)を特
	定可能となる。
\end{enumerate}

\begin{enumerate}
	\item[(3)]FFmpegの出力から、シーン切り替わり時刻を
	正規表現で抽出する。抽出された情報はシーンリ
	ストボックスに追加され、ユーザーが確認可能な
	形式で表示される。
\end{enumerate}

%2.3.7
\subsubsection{シーン動画の保存}
\begin{figure}[H] 
	\includegraphics[width=0.48\textwidth]{7.png}  % ← 画像ファイル名(拡張子つき)
	\caption{シーン動画の保存プログラム}
   \end{figure}
\begin{enumerate}
	\item[(1)]シーンが分割されていない場合、警告メッ
	セージを表示する。シーンが分割されていれば、次
	に保存先のフォルダを選択するダイアログが表示
	される。そして、保存先フォルダを選択させ、フォ
	ルダが選択されなかった場合は、処理を中断する。
\end{enumerate}

\begin{enumerate}
	\item[(2)]シーンの開始時刻を保存し、それをもとに、各
	シーンの終了時刻を計算する。各シーンごとに、計
	算された時間範囲で動画を切り出して保存する。
	保存処理中にエラーが発生した場合、エラーメッ
	セージを表示する。エラーが発生しなければ、保存
	完了のメッセージが表示される。
\end{enumerate}

\begin{enumerate}
	\item[(3)]各シーンは、FFmpegを使用して切り出される。
	コマンドには、入力ファイル、開始時刻、終了時刻、
	および出力ファイルの情報が含まれる。 
	生成したFFmpegコマンドを実行し、シーン動画の
	保存処理を実行する。 
\end{enumerate}

%2.3.8
\subsubsection{シーン画像の保存}
\begin{figure}[H] 
	\includegraphics[width=0.48\textwidth]{8.png}  % ← 画像ファイル名(拡張子つき)
	\caption{シーン画像の保存プログラム}
   \end{figure}
\begin{enumerate}
	\item[(1)]各シーンの開始時刻を利用して、その時刻で
	動画から1フレーム(静止画像)を抽出する。
\end{enumerate}

\begin{enumerate}
	\item[(2)]実行中にエラーが発生した場合、エラーメッ
	セージを通知する。すべてのシーン画像が保存さ
	れた後、「完了」のメッセージを表示する。 
\end{enumerate}

%2.3.9
\subsubsection{シーン情報のHTML形式での表示}
\begin{figure}[H] 
	\includegraphics[width=0.48\textwidth]{9.png}  % ← 画像ファイル名(拡張子つき)
	\caption{シーン情報のHTML形式での表示プログラム}
   \end{figure}
\begin{enumerate}
	\item[(1)]シーン分割が実行されていないと、シーン情 
	報が表示できないため、シーン分割が実行されて 
	いるかどうかを確認する。シーンが分割されてい 
	ない場合、警告メッセージを表示する。
\end{enumerate}

\begin{enumerate}
	\item[(2)]シーン情報をHTMLで表示するために、HTML文 
	書の基本構造を作成する。ここでは、タイトル、 
	ヘッダー、表を設定し、シーン番号、開始時刻、終 
	了時刻、プレビュー画像を含む列を用意している。
\end{enumerate}

\begin{enumerate}
	\item[(3)]事前に計算された開始時刻と終了時刻を、HTMLの表に行として挿入する。各行には、対応す 
	るシーンの開始時刻、終了時刻、プレビュー画像が 
	表示される。
\end{enumerate}

\begin{enumerate}
	\item[(4)]完成したHTMLコンテンツは、ユーザーが指定 
	した場所に保存される。保存後、そのファイルをWebブラウザで自動的に開くため、ユーザーが生成 
	されたシーン情報を即座に確認できるようになる。 
\end{enumerate}

%2.3.10
\subsubsection{シーン選択および再生}
\begin{figure}[H] 
	\includegraphics[width=0.48\textwidth]{10.png}  % ← 画像ファイル名(拡張子つき)
	\caption{シーン選択および再生プログラム}
   \end{figure}
\begin{enumerate}
	\item[(1)]GUI上のシーンリストボックスから再生した 
	い1つシーンを選択する。  
\end{enumerate}

\begin{enumerate}
	\item[(2)]シーンのインデックスがリスト内に存在する
	場合、保存したシーン動画のリストから該当する
	シーンのパスを取得する。 
\end{enumerate}

\begin{enumerate}
	\item[(3)]選択されたシーン動画を再生する。
\end{enumerate}

%2.3.11
\subsubsection{複数のシーン選択および再生}
\begin{figure}[H] 
	\includegraphics[width=0.48\textwidth]{11.png}  % ← 画像ファイル名(拡張子つき)
	\caption{複数のシーン選択および再生プログラム}
   \end{figure}
\begin{enumerate}
	\item[(1)]GUI上でシーンリストボックスから再生した
	い複数のシーンを選択。
\end{enumerate}

\begin{enumerate}
	\item[(2)]選択されたシーンのパスを取得し、FFmpegで
	使用するリストファイルを動的に作成される。 
	リストファイルに基づきシーンを連結し、複数の
	シーンが統合された動画が生成される。
\end{enumerate}

\begin{enumerate}
	\item[(3)]統合された動画がデフォルトのメディアプ
	レーヤーで再生される。また、不要なファイルを残
	さないように、一時リストファイルを削除する。
\end{enumerate}

%2.3.12
\subsubsection{効果音の検索}
\begin{figure}[H] 
	\includegraphics[width=0.48\textwidth]{12.png}  % ← 画像ファイル名(拡張子つき)
	\caption{効果音の検索プログラム}
   \end{figure}

\begin{enumerate}
	\item[(1)]GUI内の検索ボックスからユーザーが入力し
	た検索キーワードを取得する。効果音ラボ\cite{se} 等か
	ら収集した約4000種類の音声ファイルを基に作成
	した効果音データベース(図14)のtag1列(その
	効果音に関連するキーワード 例:爆発音→爆発、
	斬撃音→剣)とtag2列(その効果音のオノマトペ)
	をもとに、キーワードに一致する効果音を抽出す
	る。 
\end{enumerate}
\begin{figure}[H] 
	\includegraphics[width=0.48\textwidth]{13.png}  % ← 画像ファイル名(拡張子つき)
	\caption{効果音データベース(抜粋)}
   \end{figure}

\begin{enumerate}
	\item[(2)]検索結果をGUI内の効果音リストボックスに
	表示するため、既存のリストを初期化した後、新た
	な検索結果を挿入する。
\end{enumerate}

%2.3.13
\subsubsection{効果音の再生}
\begin{figure}[H] 
	\includegraphics[width=0.48\textwidth]{14.png}  % ← 画像ファイル名(拡張子つき)
	\caption{効果音の再生プログラム}
   \end{figure}
\begin{enumerate}
	\item[(1)]効果音リストボックスでユーザーが選択した
	項目のインデックスを取得する。
\end{enumerate}
\begin{enumerate}
	\item[\textnormal{(2)}] 効果音データベースのfile\_path列(図14)を
	参照し、選択インデックスに対応する音素材の
	ファイルパスを抽出する。抽出されたファイルパ
	スをもとに、音素材が再生される。
\end{enumerate}

%2.3.14
\subsubsection{シーンへの効果音挿入}
\begin{figure}[H] 
	\includegraphics[width=0.48\textwidth]{15.png}  % ← 画像ファイル名(拡張子つき)
	\caption{シーンへの音声挿入プログラム}
   \end{figure}
\begin{enumerate}
	\item[(1)]ファイル選択ダイアログを使用して音声ファ 
	イル(MP3形式)を選択。この選択がなければ、処
	理は中断される。そして、シーンリストボックスか
	ら、音声を挿入したいシーンを選択する。もしシー
	ンが選択されていない場合、警告メッセージが表
	示される。 
\end{enumerate}

\begin{enumerate}
	\item[(2)]音声を開始するタイミング(秒単位)を入力。
	このオフセットにより、音声の開始位置が調整さ
	れる。各シーンに対して、指定されたタイミングで音声が挿入され、音声付きの新しいシーンファイルが生成される。音声付きのシーンファイルのパ
	スが追加され、音声ファイル名が登録される。そし
	て、効果音付きシーンがGUI内の効果音付きリス
	トボックスに表示される。 
\end{enumerate}

\begin{enumerate}
	\item[(3)]音声挿入は、FFmpegを利用して行われる。指
	定されたオフセット時間後に音声が開始されるよ
	うに設定される。映像の映像ストリームと音声ス
	トリームをマッピングし、最短のメディア長に合
	わせて出力される。
\end{enumerate}

%2.3.15
\subsubsection{複数シーンへの効果音挿入}
\begin{figure}[H] 
	\includegraphics[width=0.48\textwidth]{16.png}  % ← 画像ファイル名(拡張子つき)
	\caption{複数シーンへの音声挿入プログラム}
   \end{figure}
\begin{enumerate}
	\item[(1)]挿入する音声ファイルを選択する。選択が行われない場合、処理は中断される。シーンリスト
	ボックスから複数のシーンを選択することができ
	る。選択されたシーンが2つ以上でなければなら
	ず、2つ以上選択されていない場合、警告メッセー
	ジが表示され、処理は終了する。
\end{enumerate}

\begin{enumerate}
	\item[(2)]音声の開始タイミングを指定する。ユーザー
	に秒単位での開始時間を入力させる。
\end{enumerate}

\begin{enumerate}
	\item[(3)]音声を複数のシーンに挿入するため、まず
	シーンの結合が行われる。選択されたシーンのパ
	スはリスト化され、これらを1つの動画に統合す
	る。この際、シーンを結合した新たな動画ファイル
	が作成される。
\end{enumerate}

\begin{enumerate}
	\item[(4)]シーンの結合が完了した後、音声の挿入が行
	われる。音声ファイルの開始時間は、前述のユー
	ザー入力に基づいて決定される。音声は動画ファ
	イルに挿入される。ここでは、ffmpegコマンドが
	呼び出され、指定された開始タイミングで音声が
	挿入されるように設定される。また、音声のオフ
	セット時間を指定し、音声と映像の同期が取れる
	ようにする。音声の挿入が完了すると、音声を含む
	新たな動画ファイルが保存され、効果音付きシー
	ンリストに追加される。 
\end{enumerate}

%2.3.16
\subsubsection{効果音付きシーンの選択と再生}
\begin{figure}[H] 
	\includegraphics[width=0.48\textwidth]{17.png}  % ← 画像ファイル名(拡張子つき)
	\caption{音声付きシーンの選択と再生プログラム}
   \end{figure}
\begin{enumerate}
	\item[(1)]効果音付きシーンリストボックスから効果音
	付きシーンが選択されているかどうかを確認する。
	選択されたインデックスを取得し、選択が行われ
	ていない場合は、警告メッセージを表示する。
\end{enumerate}

\begin{enumerate}
	\item[(2)]選択されたインデックスに基づいて、効果音
	付きのシーンファイルのリストから該当するシーンのパスを取得する。もしインデックスが無効で
	あれば、"保存されたシーンが見つかりません"という警告メッセージが表示され、処理は終了する。
\end{enumerate}

\begin{enumerate}
	\item[(3)]有効なシーンのパスが見つかった場合、デ
	フォルトのメディアプレーヤーにてそのシーンを開く。再生中にエラーが発生した場合、エラーメッ
	セージが表示される。
\end{enumerate}

%2.3.17
\subsubsection{音付けされた動画のプレビュー再生}
\begin{figure}[H] 
	\includegraphics[width=0.48\textwidth]{18.png}  % ← 画像ファイル名(拡張子つき)
	\caption{音付けされた動画のプレビュー再生プログラム}
   \end{figure}

\begin{enumerate}
	\item[(1)]効果音付きのシーンが存在しているかどうか
	が確認される。シーンが存在していない場合、警告
	メッセージが表示され、処理が終了する。効果音付
	きのシーンが存在する場合、プレビュー作成と再
	生が別スレッドで非同期に実行される。
\end{enumerate}

\begin{enumerate}
	\item[(2)]音声付きのシーンファイルのリストに格納さ
	れたシーンファイルパスをシーン番号順にソート
	し、その後、FFmpegに渡すための入力ファイルリ
	ストを作成する。リストは一時的にテキストファ
	イルに書き込まれる。
\end{enumerate}

\begin{enumerate}
	\item[(3)]作成されたリストファイルを基に、複数の
	シーンを1つのプレビュー動画に統合する。FFmpegはリストファイル内のシーンを順番に結合し、て
	映像と音声を再エンコードせずにコピーする。
\end{enumerate}

\begin{enumerate}
	\item[(4)]統合が成功した後、生成されたプレビュー動
	画をデフォルトのメディアプレーヤーにて再生す
	る。プレビュー作成後、リストファイルは削除され
	る。
\end{enumerate}

%2.3.18
\subsubsection{音付け完了後のエクスポート}
\begin{figure}[H] 
	\includegraphics[width=0.48\textwidth]{19.png}  % ← 画像ファイル名(拡張子つき)
	\caption{音付け完了後のエクスポートプログラム}
   \end{figure}

\begin{enumerate}
	\item[(1)]効果音付きのシーンが存在しているかどうか
	が確認される。シーンが存在していない場合、警告
	メッセージが表示され、処理が終了する。
\end{enumerate}

\begin{enumerate}
	\item[(2)]音声付きのシーンファイルのリストに格納さ
	れたシーンファイルパスを昇順にソートし、その
	順序でFFmpegに渡すための入力ファイルリストを
	作成する。このリストは、一時的にテキストファイ
	ルに書き込まれる。リストファイルには、統合したいシーンのパスがfile '...'という形式で記載 
	される。
\end{enumerate}

\begin{enumerate}
	\item[(3)]作成されたリストファイルを基に、複数の
	シーンを1つのプレビュー動画に統合する。FFmpegはリストファイル内のシーンを順番に結合し、
	映像と音声を再エンコードせずにコピーする。
\end{enumerate}

\begin{enumerate}
	\item[(4)]統合処理中にエラーが発生した場合、適切な
	エラーメッセージをユーザーに表示する。そして、
	統合処理が完了した後、一時的に作成したリスト
	ファイルが削除される。
\end{enumerate}

%3
\section{GUIシステムの動作例}

ユーザがどのようにシステムを利用して動画に
音付を行うか、具体的な例を手順ごとにに紹介す
る。GUIシステムの全体像は以下(図20)のように
なっている。また、各機能の詳細については、各ボ
タンの右にある「?」ボタン(図21)をクリックするこ
とで確認可能である。
\begin{figure}[H] 
	\includegraphics[width=0.48\textwidth]{20.png}  % ← 画像ファイル名(拡張子つき)
	\caption{GUIシステムの全体像}
\end{figure}
\begin{figure}[H] 
	\includegraphics[width=0.48\textwidth]{21.png}  % ← 画像ファイル名(拡張子つき)
	\caption{?ボタン(黄色い枠線部分)}
\end{figure}

%3.1
\subsection{動画ファイルのアップロード}

まずは、音付け編集を行うための動画ファイル
をアップロードする(図22)。この動作例では、ワー
ルドトリガーの戦闘シーン\cite{world}の一場面を題材と
している。
\begin{figure}[H] 
	\includegraphics[width=0.48\textwidth]{22.png}  % ← 画像ファイル名(拡張子つき)
	\caption{動画ファイルのアップロード手順}
\end{figure}

\begin{enumerate}
	\item[(1)]「アップロード」ボタンをクリックすることで、 
	ファイル選択ダイアログが表示され、ユーザーは 
	動画ファイルを選択することができる(図22)。
\end{enumerate}

\begin{enumerate}
	\item[(2)]ユーザーは、ファイル選択ダイアログ内でMP4形式の動画ファイルを選択し、「開く」ボタンをク 
	リックする(図22)。
\end{enumerate}

\begin{enumerate}
	\item[(3)]ファイルが正常に選択されると、画面に「動画 
	が正常にアップロードされました。」というメッ 
	セージが表示される(図22)。
\end{enumerate}

%3.2
\subsection{アップロード動画の再生}

続いて、アップロードした動画を再生していく(図23)。
\begin{figure}[H] 
	\includegraphics[width=0.48\textwidth]{23.png}  % ← 画像ファイル名(拡張子つき)
	\caption{アップロード動画の再生手順}
\end{figure}

\begin{enumerate}
	\item[(1)]「再生」ボタンをクリックするとシステム既定の 
	メディアプレーヤーにて動画が再生される(図23)。
\end{enumerate}

\begin{enumerate}
	\item[(2)]再生が開始されると、「動画が再生されます」という情報メッセージが表示され、ユーザーに再生が成功したことが通知される。もし、再生中に何
	らかのエラーが発生した場合(例えば、指定されたファイルが存在しない、またはファイルが破損し
	ている場合)、エラーメッセージが表示され、再生
	に失敗した理由がユーザーに通知される(図23)。
\end{enumerate}

%3.3
\subsection{動画の無音化}

アップロードした動画にそのまま音付けを行うと、元の動画の音声と挿入する効果音が混ざって
しまうため、動画の元の音声を消す必要がある(図24)。また、動画の無音化処理は、FFmpegを用いた
エンコードが必要なため、処理に一定の時間を要
する。動画読み込み時に自動実行すると、処理に無
駄な時間がかかる可能性があるため、動画無音化
ボタンを設けた。
\begin{figure}[H] 
	\includegraphics[width=0.48\textwidth]{24.png}  % ← 画像ファイル名(拡張子つき)
	\caption{動画の無音化手順}
\end{figure}

\begin{enumerate}
	\item[(1)]「動画無音化」ボタンをクリックすると、指定
		された動画ファイルに対して音声トラックを削除
		する。出力ファイル名はデフォルトで
		output\_no\_audio.mp4 に設定される(図24)。
\end{enumerate}

\begin{enumerate}
	\item[(2)]無音化処理が正常に完了した場合、情報メッセージ「無音化された動画が作成されました:  
		output\_no\_audio.mp4」が表示される。これにより、 
		ユーザーは処理結果を確認できる。無音化処理に 
		失敗した場合、エラーメッセージが表示され、処理 
		失敗の原因をユーザーに通知する(図24)。
\end{enumerate}

\begin{enumerate}
	\item[(3)]そして最後に、作成された
	output\_no\_audio.mp4を再度「アップロード」ボタ
	ンにて読み込む。また、一度無音化した動画は再度
	無音化する必要はない(図24)。
\end{enumerate}

%3.4
\subsection{動画のシーン分割}

本GUIシステムはシーン分割(図25)をもとに音
付けしていく。
\begin{figure}[H] 
	\includegraphics[width=0.48\textwidth]{25.png}  % ← 画像ファイル名(拡張子つき)
	\caption{動画のシーン分割手順}
\end{figure}

\begin{enumerate}
	\item[(1)]「シーン分割」ボタンをクリックすると、アッ 
	プロードされた動画ファイルに対し、シーン分割 
	処理が開始される(図25)。
\end{enumerate}

\begin{enumerate}
	\item[(2)]シーンごとの情報(例:「シーン1: 開始時刻  
	0.00 秒」)がシーンリストボックスに順次表示され 
	る。動画ファイルが選択されていない場合や、 
	FFmpegコマンドの実行に失敗した場合、警告メッ 
	セージやエラーメッセージが表示され、ユーザーに原因を通知する(図25)。
\end{enumerate}

%3.5
\subsection{シーン動画の保存}

シーンの分割後、シーン動画を任意のフォルダに保存していく(図26)。
\begin{figure}[H] 
	\includegraphics[width=0.48\textwidth]{26.png}  % ← 画像ファイル名(拡張子つき)
	\caption{シーン動画の保存手順}
\end{figure}

\begin{enumerate}
	\item[(1)]「シーン保存」ボタンをクリックすると、フォルダ選択ダイアログが表示される(図26)。
\end{enumerate}

\begin{enumerate}
	\item[(2)]ユーザーは、保存先のフォルダを指定し、選択
	する(図26)。
\end{enumerate}

\begin{enumerate}
	\item[(3)]シーン動画の保存処理が実行され、選択した
	保存先フォルダ内に、シーン動画ファイルが保存
	される(図26)。
\end{enumerate}

\begin{enumerate}
	\item[(4)]シーン保存が正常に完了すると、「シーンの保
	存が完了しました」というメッセージが表示され
	る(図26)。 
\end{enumerate}

%3.6
\subsection{シーン情報のHTML出力}

シーン分割の視覚的な結果を出力する機能は、
ユーザーが各シーンの情報を確認する上で重要で 
ある。そのため、本GUIシステムでは、シーン分割
情報をHTML形式でエクスポートし、ブラウザでの
プレビューを提供する(図27)。
\begin{figure}[H] 
	\includegraphics[width=0.48\textwidth]{27.png}  % ← 画像ファイル名(拡張子つき)
	\caption{シーン情報のHTML出力手順}
\end{figure}

\begin{enumerate}
	\item[(1)]「シーン情報をHTMLで表示」ボタンをクリッ
	クすると、保存先を指定するダイアログが表示さ
	れる(図27)。
\end{enumerate}

\begin{enumerate}
	\item[(2)]ユーザーはHTMLファイルの保存場所を選択す 
	る。選択後、シーン情報を基に、開始時刻・終了時 
	刻・プレビュー画像を含むHTMLコンテンツが動的 
	に生成される(図27)。
\end{enumerate}

\begin{enumerate}
	\item[(3)]生成後、指定された場所にHTMLファイルが出 
	力される。ファイル名にはデフォルト拡張子とし 
	て.htmlが付加される(図27)。
\end{enumerate}

\begin{enumerate}
	\item[(4)]保存完了後、自動的にブラウザが起動し、生成
	されたHTMLファイルが開かれる。ユーザーは表形
	式でシーン情報を視覚的に確認できる(図27)。
\end{enumerate}

%3.7
\subsection{シーンの選択と再生}
シーン動画の保存後、シーン動画を再生し、各シーンの内容を確認する(図28)。
\begin{figure}[H] 
	\includegraphics[width=0.48\textwidth]{28.png}  % ← 画像ファイル名(拡張子つき)
	\caption{シーンの選択と再生手順}
\end{figure}
\begin{enumerate}
	\item[(1)]シーンリストボックスから再生したいシーン
	をクリックして選択する(図28)。
\end{enumerate}

\begin{enumerate}
	\item[(2)]「選択したシーンを再生」ボタンをクリックす
	る(図28)。
\end{enumerate}

\begin{enumerate}
	\item[(3)]デフォルトのメディアプレーヤーにてシーン
	動画が再生される(図28)。
\end{enumerate}

%3.8
\subsection{複数シーンの選択と再生}

本GUIシステムでは、1つのシーンだけでなく、
複数のシーンの再生も可能である(例:シーン4か
らシーン7まで再生したい場合、シーン4から7ま
で複数選択)(図29)。
\begin{figure}[H] 
	\includegraphics[width=0.48\textwidth]{29.png}  % ← 画像ファイル名(拡張子つき)
	\caption{複数シーンの選択と再生手順}
\end{figure}
\begin{enumerate}
	\item[(1)]シーンリストボックスから再生したいシーン
	をクリックして選択する(図29)。
\end{enumerate}

\begin{enumerate}
	\item[(2)]「選択したシーンを再生(複数)」ボタンをクリックする(図29)。
\end{enumerate}

\begin{enumerate}
	\item[(3)]デフォルトのメディアプレーヤーにて、選択
	した範囲のシーン動画が再生される(図29)。 
\end{enumerate}

%3.9
\subsection{効果音の検索および再生}

本GUIシステムでは、効果音リストから検索条
件に基づくフィルタリングと、選択された効果音
ファイルの即時再生機能を提供している(図30)、(図31)。

%3.9.1
\subsubsection{効果音の検索}
\begin{figure}[H] 
	\includegraphics[width=0.48\textwidth]{30.png}  % ← 画像ファイル名(拡張子つき)
	\caption{効果音の検索手順}
\end{figure}
\begin{enumerate}
	\item[(1)]ユーザーは検索ボックスに目的の効果音に関
	連するキーワードを入力する(図30)。
\end{enumerate}

\begin{enumerate}
	\item[(2)]入力内容が変更されるたびに効果音リストが
	リアルタイムで更新され、該当する効果音のみが
	リストボックスに表示される(図30)。
\end{enumerate}

%3.9.2
\subsubsection{効果音の再生}
\begin{figure}[H] 
	\includegraphics[width=0.48\textwidth]{31.png}  % ← 画像ファイル名(拡張子つき)
	\caption{効果音の再生手順}
\end{figure}

\begin{enumerate}
	\item[(1)]ユーザーは更新されたリストボックスの中か
	ら目的の効果音を選択する(図31)。
\end{enumerate}

\begin{enumerate}
	\item[(2)]選択した効果音をダブルクリックすることで、効果音がデフォルトのメディアプレイヤーで再生
	される(図31)。
\end{enumerate}

%3.10
\subsection{シーンへの効果音挿入}

シーンと効果音を再生した後、選択したシーン
に音声ファイルを挿入する。また、音声の開始タイ
ミングは設定可能である(図32)。 
\begin{figure}[H] 
	\includegraphics[width=0.48\textwidth]{32.png}  % ← 画像ファイル名(拡張子つき)
	\caption{シーンへの効果音挿入手順}
\end{figure}
\begin{enumerate}
	\item[(1)]ユーザーはシーンリストボックスから、効果
	音を挿入したいシーンを選択する。シーンが選択
	されていない場合、警告メッセージが表示される
	(図32)。 
\end{enumerate}

\begin{enumerate}
	\item[(2)]「音声を挿入(単一シーン)」ボタンをクリック
	する(図32)。
\end{enumerate}

\begin{enumerate}
	\item[(3)]音声ファイル選択用のファイルダイアログが
	表示され、ユーザーは目的の音声ファイル(MP3形
	式)を選択する(図32)。
\end{enumerate}

\begin{enumerate}
	\item[(4)]ダイアログボックスを通じて、ユーザーは音声開始タイミング(秒単位)を指定する(例:シーン
	開始 2 秒後に効果音挿入したい場合は、「2」と指
	定)(図32)。
\end{enumerate}

\begin{enumerate}
	\item[(5)]音声ファイルと選択されたシーンファイルを
	基に、音声挿入処理が実行され、挿入済みのシーン
	情報が効果音付きシーンリストに追加される(図
	32では、シーン3に「K.O.mp3」という音声ファイ
	ルがシーン開始2秒後に挿入されている)。
\end{enumerate}

%3.11
\subsection{複数シーンへの効果音挿入}

本GUIシステムでは、1つのシーンだけでなく、
複数シーンにまたいで、音声の挿入が可能である
(図33)。
\begin{figure}[H] 
	\includegraphics[width=0.48\textwidth]{33.png}  % ← 画像ファイル名(拡張子つき)
	\caption{複数シーンへの効果音挿入手順}
\end{figure}
\begin{enumerate}
	\item[(1)]ユーザーはシーンリストボックスから、音声
	を挿入したい複数シーンを選択する。シーンが選
	択されていない場合、警告メッセージが表示され
	る(図33)。
\end{enumerate}

\begin{enumerate}
	\item[(2)]「音声を挿入(複数シーン)」ボタンをクリック
	する(図33)。
\end{enumerate}

\begin{enumerate}
	\item[(3)]音声ファイル選択用のファイルダイアログが
	表示され、ユーザーは目的の音声ファイル(MP3形
	式)を選択する(図33)。
\end{enumerate}

\begin{enumerate}
	\item[(4)]ダイアログボックスを通じて、ユーザーは音
	声開始タイミング(秒単位)を指定する(図33)。
\end{enumerate}

\begin{enumerate}
	\item[(5)]音声ファイルと選択されたシーンファイルを
	基に、音声挿入処理が実行され、挿入済みのシーン
	情報が効果音付きシーンリストに追加される(図
	33では、シーン1から3にまたいで、「K.O.mp3」
	という音声ファイルがシーン開始2秒後に挿入さ
	れている)。 
\end{enumerate}

%3.12
\subsection{効果音付きシーンの再生}

本GUIシステムでは、ユーザーがシーンの再生
を通じて音声の配置やタイミングの確認を行える
ように、効果音が挿入されたシーンを選択して再
生する機能を提供している(図34)。
\begin{figure}[H] 
	\includegraphics[width=0.48\textwidth]{34.png}  % ← 画像ファイル名(拡張子つき)
	\caption{効果音付きシーンの再生手順}
\end{figure}
\begin{enumerate}
	\item[(1)]ユーザーは、効果音付きシーンリストボック
	スから、再生したい効果音付きシーンを選択する
	(図34)。
\end{enumerate}

\begin{enumerate}
	\item[(2)]「音声付きシーンを再生」ボタンをクリックする。シーンが選択されていない場合、警告メッセー
	ジが表示され、「再生する音声付きシーンを選択し
	てください」と通知される(図34)。
\end{enumerate}

\begin{enumerate}
	\item[(3)]選択されたシーンがデフォルトのメディアプ
	レーヤーで再生される(図34)。
\end{enumerate}

%3.13
\subsection{効果音付き動画のプレビュー再生}

本GUIシステムでは、ユーザーが効果音を挿入
した複数のシーンのプレビュー動画を生成・再生
する機能を提供する(図35)。この機能により、編
集後のシーンが連続した映像としてどのように表
示されるかを確認できる。
\begin{figure}[H] 
	\includegraphics[width=0.48\textwidth]{35.png}  % ← 画像ファイル名(拡張子つき)
	\caption{効果音付き動画のプレビュー再生手順}
\end{figure}
\begin{enumerate}
	\item[(1)]ユーザーは、音付けされた映像のプレビュー
	を開始するために「プレビュー」ボタンをクリック
	する。プレビュー対象のシーンが存在しない場合、
	警告メッセージが表示され、「プレビューするシー
	ンがありません」と通知される。この際、処理は中
	断される。プレビューボタンが押下されると、プレ
	ビュー動画の生成される(図35)。
\end{enumerate}

\begin{enumerate}
	\item[(2)]プレビュー動画の生成後、システムのデフォ
	ルトプレーヤーを使用して動画が再生される(図35)。
\end{enumerate}

%3.14
\subsection{効果音付き動画のエクスポート}

最後に、音付けが完了した動画をユーザー指定
のフォルダへMP4形式でエクスポートする(図36)。
\begin{figure}[H] 
	\includegraphics[width=0.48\textwidth]{36.png}  % ← 画像ファイル名(拡張子つき)
	\caption{効果音付き動画のエクスポート手順}
\end{figure}
\begin{enumerate}
	\item[(1)]「エクスポート」ボタンをクリックすると、
	ファイル保存ダイアログが表示される(図36)。
\end{enumerate}

\begin{enumerate}
	\item[(2)]ユーザーはエクスポート後の動画ファイルの
	保存先ディレクトリおよびファイル名を指定する。
	デフォルトのファイル形式はMP4である(図36)。
\end{enumerate}

\begin{enumerate}
	\item[(3)]エクスポート処理が正常に終了した場合、GUI
	画面に「成功」のメッセージが表示される。一方、
	エラーが発生した場合は「エラー」メッセージが表
	示され、問題の詳細が提示される(図36)。
\end{enumerate}

\begin{enumerate}
	\item[(4)]エクスポートされたファイルは、指定された
	ディレクトリに保存されていることを確認できる(図36)。
\end{enumerate}

%4
\section{GUIシステムの課題}

本GUIシステムには、主に以下の3つの課題が
存在する。

%4.1
\subsection{FFmpegの処理時間}

プレビュー動画を生成する際、複数のシーンを
結合するFFmpeg処理に時間がかかる場合があるこ
とである。特にシーン数が多い場合や個々のシー
ンファイルのサイズが大きい場合に、処理速度が
低下する傾向が見られた。

%4.2
\subsection{リストファイルのエラー}

2つ目は、プレビュー動画生成時に一時的に作成
されるリストファイルが、エラー発生時や予期せ
ぬ中断で削除されない場合があることである。 

%4.3
\subsection{複数の効果音の挿入}

本GUIシステムでは、1つのシーンに対して1つ
の効果音のみを挿入することができるが、音付け
編集の現場では複数の効果音を同時に挿入する場
合が多い。例えば、アクションシーンでは爆発音と
銃声、背景音として風の音などが同時に再生され
ることが求められる場合がある。しかし、現行シス
テムでは複数の効果音を1つのシーンに挿入する
機能が実装されていない。

%5
\section{まとめと今後の展望}

今年度は、無音にした映像に対する効果音の付
加作業を効率化するためのGUIシステムを開発し
た。このシステムは、動画ファイルの読み込み、
シーン分割、効果音検索、効果音再生、タイミング
を指定した効果音挿入、効果音付きシーンのプレ
ビュー再生、プレビューの生成、エクスポートを一
連のプロセスとして実現している。 

来年度以降は、「誰もが手軽に自身の音響のアイ
デアを映像へ反映できる環境」の実現を目指す。具体的には、シーンと効果音をタイムライン上で視
覚的に配置・調整できる機能の実装や、効果音のパ
ラメータ調整機能の追加を行いGUIをより直感的
にする。また、AIを活用し、映像の内容やシーン
の雰囲気を解析して、適切な効果音を自動で提案
する機能や、ユーザーが指定した条件(例:「雨音」、
「SF的なエネルギー音」)に基づいて新しい効果音
を生成する機能の実装などを目指す。このように、
音付け編集のハードルを下げるだけでなく、創造
的な映像制作の可能性を広げていきたいと考えて
いる。 