% !TEX root = _main.tex
% ========================================
% 関連研究・作品
% ========================================

%1
\section{関連研究・作品}


%2

\subsection{音による出来事の伝達}

従来,音響学や音響心理学における研究は,楽曲や楽器音といった対象を中心として発展してきた.

これに対してGaver(1993)は,生活の中にある日常音に着目し,人は音そのものではなく,音を生じさせた出来事を知覚していると論じた点で,出来事の知覚としての日常音を扱う新たな視点を提示した.

Gaverは,この点を説明するために,「音楽的聴取(musical listening)」と「日常的聴取(everyday listening)」を区別している.音楽的聴取とは,音の高さ,音の大きさ,音色といった音の聴覚的属性そのものに関心を向ける聴取の態度である.一方,日常的聴取とは,音そのものではなく,その音を生じさせた出来事や環境に関心を向ける聴取の態度である.

この区別の本質は,音の種類ではなく経験の様式にあり,同一の音であっても,どのような態度で聴取されるかによって,聴者にもたらされる経験は異なる.また,これまでの音響学的研究は音楽音を中心に発展してきた一方で,人の日常的な聴取経験の多くは,むしろ日常的聴取に属するものであるとされる.
