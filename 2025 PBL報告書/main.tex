\documentclass[submit,techrep,noauthor,dvipdfmx]{ipsj}

%====== PBL報告書 橋田ゼミ仕様 ============================
\usepackage{muselab-rep2025}
\usepackage{tocloft}
\setcounter{tocdepth}{3}
\addtocontents{toc}{\cftpagenumbersoff{subsubsection}}

\usepackage[dvipdfmx]{graphicx}
\usepackage{latexsym}
\usepackage{newtxtext,newtxmath}
\usepackage[utf8]{inputenc}
\usepackage{url}

\usepackage{float}
\usepackage{adjustbox}
\usepackage{multirow}
\usepackage{pdfpages}
\usepackage{url}

\usepackage{fancyvrb}
\renewcommand{\|}{\Verb|}
\newcommand{\vbarSafe}{\texttt{|}}

\usepackage[hang,small,bf]{caption}
\usepackage[subrefformat=parens]{subcaption}
\captionsetup{compatibility=false}

% 添削用(不要ならコメントアウト可)
\usepackage{muselab-correction}

% ========================================================

\begin{document}

\title{一年次の活動報告}
\author{32545064 中田 周佑}{Shusuke Nakata}{FUKU}[]

\maketitle

\thispagestyle{empty}
\clearpage
\addtocounter{page}{-2}

% ========================================================
% 本文
% ========================================================

\section{はじめに}
本報告書は,所属する橋田ゼミにおいて,
私が一年を通して行った活動について報告するものである.
私はピアノ演奏を通して音楽に親しんできたが,
演奏以外の側面から音楽芸術について学びたいと考え,
橋田ゼミへの所属を決めた.

一年次の活動としては,
未経験の楽器演奏への取り組みおよび美山音楽祭への参加を経験した.
本報告書では,これらの活動内容と,
それらを通して得られた学びについて述べる.


\section{楽器演奏}
\subsection{ギター選択について}
橋田ゼミでは,一年次の活動として,
未経験の楽器を選択し,その演奏に取り組むことになっている.
私は,ピアノとギターは,
多くの楽器に共通する基礎的な要素を学ぶことができる楽器であると考えている.
この考えから,本活動ではギターを選択し,
演奏に挑戦することにした.


\subsection{ギターについて}
ギターとは,弦楽器の一種であり,
指板上に張られた弦を指やピックを用いてはじくことで音を出す楽器である.
クラシック音楽,ジャズ,ロックなど,
さまざまな音楽ジャンルで広く使用されている.
本活動では,アコースティックギターを選択した.


\subsection{ギターの各部位の名称と役割}
ギターは,ネック,指板,ボディ,弦などの部位から構成されており,
それぞれが音程や音色に影響を与えている.
各部位の名称や役割を理解することは,
楽器の構造への理解を深め,
演奏技術の向上にもつながると考えられる.

% 図は後から挿入
\begin{figure}[H]
  \centering
  \includegraphics[width=0.7\linewidth]{R.jpg}
\end{figure}



\subsection{練習内容}
本活動では,指を用いて演奏するフィンガーピッキングという手法を用いた練習を行った.
フィンガーピッキングは,
一音一音を明確に鳴らすことができる演奏手法であり,
本活動ではソロギターに挑戦した.

私はギターについて,
学校の授業で触れた経験がある程度で,
持ち方などの基礎的な知識しか持っていなかった.
そのため,ソロギターの教則書を参考にしながら,
段階的に練習を進めていった.
「きらきら星」や「きよしこの夜」は比較的順調に演奏できたが,
「セーハ」と呼ばれる奏法において一度つまずいた.

セーハとは,
同一フレット上の複数の弦を一本の指で同時に押さえる奏法であり,
「バレー」とも呼ばれる.
本活動では,小セーハを中心に練習を行った.

現在は,「オーラ・リー」や「スマイル」を継続的に練習するとともに,
本番で演奏予定の「見つめていたい」の練習に取り組んでいる.
複数の楽器を演奏できるようになることで,
演奏技術だけでなく楽曲理解も深まることを実感した.


\section{美山音楽祭について}
\subsection{美山音楽祭とは}
美山音楽祭は,
京都府南丹市美山町で開催される音楽祭であり,
演奏会やワークショップなどを通して音楽を楽しむことができるイベントである.


\subsection{美山音楽祭での活動}
私は学生団体「京都ストリートミュージックプロジェクト(KSMP)」の一員として,
美山音楽祭の運営に携わった.
会場設営,ワークショップ補助,物販管理などを担当した.

本活動を通して,
音楽をより多くの人に届けるための工夫や,
音楽への心理的なハードルを下げる取り組みの重要性を学んだ.


\section{まとめと今後の展望}
一年次の活動を通して,
音楽の奥深さや複雑さを改めて実感した.
二年次以降は,
音楽に関する知識をさらに深めるとともに,
音楽と演出に焦点を当てた研究に取り組みたいと考えている.

% ========================================================
参考文献
\begin{thebibliography}{99}

\bibitem{yamaha}
Yamaha,
\textit{ギター各部位の名称と役割},\\
\url{http://www.guitar-guitar-guitar.com/know/cont01.html}
(参照日:2026年1月15日)

\end{thebibliography}

% ========================================================

\bibliographystyle{ipsjunsrt}
\bibliography{references}

\end{document}
