% !TEX root = _yoko.tex

% ==========================================
% プロジェクト予稿(1P)の本文
% ==========================================

% -1.はじめに -
\section{はじめに}
音楽演奏において聴き手に感動を与える表現は,楽譜の正確な再現に加え,演奏者による強弱やテンポの繊細な解釈から生まれる.
しかし,既存のデジタルツールには演奏者の視点において課題が存在する.

主流であるDAW\cite{DAW}は音響編集に強力な反面,五線譜に慣れた演奏者には音楽的文脈が掴みにくい.
一方,楽譜ソフトウェアは記譜が主な目的であり,演奏表現の微調整は直感的ではない.

そこで本研究では,楽譜ソフトウェアの視認性とDAWの柔軟な編集能力を組み合わせ,五線譜インターフェース上でフレーズに基づき直感的に演奏表情を制御できるシステムを構築した.

% -2.システム概要 -
\section{システム概要}
本システムは,演奏者が主体的に演奏表現を探求できる環境を提供するため,「五線譜インターフェースでの操作」「フレーズ単位でのパラメータ制御」「聴き比べによるフィードバック」の3点を設計の主軸に置いた.

従来のMIDIにおける強弱制御は,打鍵の強さを表すVelocityが主流であった.
これはピアノなどの打鍵楽器には有効だが,発音後の音量変化ができないため,管楽器のように音の持続中に息づかいで抑揚をつける表現には不十分であった.
この課題に対し,本システムでは管楽器をはじめとする持続音楽器の特性を考慮し,保科理論\cite{Hoshina}に基づく音楽構造を反映した以下の2つのパラメータを制御対象とした.

\begin{itemize}
    \item \textbf{Expression(CC2)}\\
        拍より細かい単位で音量を連続的に制御できるMIDIコントロールチェンジの一種である.
        本システムでは,ユーザが指定したフレーズの頂点(重心)に向かう山形のカーブを線形補間で生成し,滑らかな強弱変化を実現する.
    \item \textbf{Onset(発音時刻)}\\
        各音符の発音時刻をミリ秒単位でずらし,「溜め」や「走り」といった局所的なテンポの揺らぎを表現する.
\end{itemize}

システムはWebアプリケーションとして実装した(図\ref{fig:UI}).
ユーザはMusicXMLとMIDIファイルをアップロードし,ブラウザ上の楽譜でフレーズ範囲と頂点を指定する.
次に,「Cantabile(歌うように)」などの発想標語のプリセットを選択すると,サーバ側でMIDIデータが加工され,WAV音源が生成される.
ユーザは加工前後を聴き比べながら,パラメータを微調整し,自身の音楽的解釈を具体化できる.

% -3.適用事例と考察 -
\section{適用事例と考察}
システムの有効性を示すため,「G線上のアリア」\cite{Air}の主旋律パートに対し,楽曲の展開に合わせて複数の演奏表現を適用する演奏デザインを試みた.
例えば,冒頭の主題には「Cantabile(歌うように)」,中間部の盛り上がりには「Appassionato(情熱的に)」を適用するなど,セクションごとに異なる表情をデザインすることで,楽曲の物語性が強調された演奏表現を生成できた.

さらに,一度表現付けを行ったMIDIデータを再入力し,局所的な表現を重ねる「階層的表現」も可能であり,より多層的な演奏意図の具現化を支援する.
これらの結果は,ユーザが楽曲構造を解釈し,多様な音楽表現を創出できる本システムの有効性を示す.

また,先行研究\cite{Apex_group}を参考に,単旋律の音型情報に基づく頂点推定機能を実装した.
和声情報を考慮しないことによる限界はあるものの,ユーザの解釈を支援する指針として機能し,最終決定をユーザに委ねるUI設計が主体的な表現探求を支援する手段として有効であった.

% -4.おわりに -
\section{おわりに}
本研究では,五線譜上で直感的に演奏表現を編集できるWebシステムを開発した.
CC2とOnsetを組み合わせたパラメータ制御により,発想標語の持つニュアンスを反映した豊かな演奏表現を生成できることを確認した.
本システムは,演奏者が自身の音楽的解釈を具体的な音として試行錯誤するプロセスを支援する.
今後の課題として,和声情報を考慮した頂点推定の高度化や,アンサンブル演奏への応用が挙げられる.

% 図1
\begin{figure}[tb]
	\centering
	\includegraphics[width=\hsize]{../fig/UI_main.png}
	\caption{システム画面(左:操作パネル,右:楽譜表示)}
	\label{fig:UI}
\end{figure}
