% !TEX root = _main.tex
% ========================================
% 卒業論文 本文
% ========================================
\section{はじめに}


本研究では,楽曲の特徴量を数値的に抽出し,
その構造を分析することを目的とする。
本年度はバッハのコラールを対象とし,
特徴量分析およびクラスタリングによる構造把握を試みた。
研究目的としては、楽曲ごとの曲の違いをデータとして閲覧したかったというのと、
今後の研究にも応用できる研究であると考えたためである。

%2
\section{特徴量抽出}
楽曲の解析には Python を用い,
music21 ライブラリによって
以下の特徴量を抽出した。

\begin{table}[H]
\caption{抽出した特徴量一覧}
\label{tab:features}
\centering
\renewcommand{\arraystretch}{1.2} % ← 1.5倍に広げる(1.2〜1.8くらいで調整)
\begin{adjustbox}{width=\linewidth} % adjustboxで幅を100%に指定
\begin{tabular}{lllll}
特徴量                                      & 意味               & どの性質を表すか           &  &  \\
measures                                 & 小節数              & 曲の長さ               &  &  \\
avg\_chord\_size                         & 平均和音サイズ(同時発音数)   & 和声の厚み              &  &  \\
seventh\_ratio                           & 7th和音の割合         & 和声の複雑さ             &  &  \\
pitch\_mean                              & 平均音高             & 全体の音域の中心           &  &  \\
pitch\_std                               & 音高の標準偏差          & メロディの動きの大きさ        &  &  \\
pitch\_range                             & 最高音−最低音          & 音域の広さ              &  &  \\
leap\_rate                               & 跳躍進行の割合          & メロディの飛び方の多さ        &  &  \\
smoothness                               & 音程差の逆数的な指標       & 滑らかさ               &  &  \\
up\_rate                                 & 上行音程の割合          & メロディの方向性(上昇か下降か)   &  &  \\
\\
\shortstack{dur\_whole/half/quarter/\\eighth/sixteenth} & 各音価の出現比率         & リズムの構成(長音中心か短音中心か) &  &  \\
vel\_mean                                & 平均ベロシティ(強さ)      & 曲の平均的な音量           &  &  \\
vel\_std                                 & ベロシティの変動         & 強弱表現の豊かさ           &  &  \\
prog\_unique                             & ユニークな和音進行数       & ハーモニーの多様性          &  &  \\
rn\_seq\_len                             & ローマ数字解析による進行列の長さ & 和声進行の規模            &  & 
\end{tabular}
\end{adjustbox}
\end{table}
%3
\section{分析手法}
抽出した特徴量に対して標準化を行い,
主成分分析(PCA)によって次元削減を行った。
その後,クラスタリング手法を用いて
楽曲の分類を試みた。

%4
\section{分析1回目}
PCAによる散布図を図\ref{fig:pca}に示す。

\begin{figure}[H]
  \centering
  \includegraphics[width=0.8\linewidth]{fig/pca.png}
  \caption{バッハのコラールをPCAで可視化した散布図}
  \label{fig:pca}
\end{figure}

黄色で示した部分がクラスタ0、青色で示した部分がクラスタ1である。
また、クラスタごとの特徴量平均の比較を図\ref{fig:average1}に、クラスタごとのZ-score(平均からの相対距離)を図\ref{fig:Z-score1}に示す。

\begin{figure}[H]
  \centering
  % --- 左側の画像 ---
  \begin{minipage}[b]{0.45\linewidth}
    \centering
    \includegraphics[width=\linewidth]{fig/average1.png}
    \subcaption{クラスタごとの特徴量平均の比較}
    \label{fig:average1}
  \end{minipage}
  \hfill % 左右の間に適度な空白を入れる
  % --- 右側の画像 ---
  \begin{minipage}[b]{0.45\linewidth}
    \centering
    \includegraphics[width=\linewidth]{fig/Z-score1.png}
    \subcaption{クラスタごとのZ-score}
    \label{fig:Z-score1}
  \end{minipage}
  
  \caption{バッハのコラールの分析をヒートマップで可視化したもの}
  \label{fig:total1}
\end{figure}


\subsection{考察}
分析の結果,
図\ref{fig:Z-score1}より、クラスタ0がクラスタ1に対して曲の長さ、音域の広さ、16分音符の割合、和声進行の規模が大きくなっており、クラスタ0は曲が長く、多数の音域と細かいリズムで複雑な曲が含まれていると推測できる。
一方、クラスタ1は曲が短く、音域が狭く、長い音符が多い単純な曲が含まれていると考えられる。

\section{分析2回目}
クラスタ1を見た所、まだ約400曲含まれていたため、さらに分類ができるのではないかと思い、クラスタ1のみを使い再びクラスタリングを行った。
PCAによる散布図を図\ref{fig:cluster1pca}に示す。

\begin{figure}[H]
  \centering
  \includegraphics[width=0.8\linewidth]{fig/cluster1pca.png}
  \caption{バッハのコラールクラスタ1をPCAで可視化した散布図(クラスタ1のみ)}
  \label{fig:cluster1pca}
\end{figure}

黄色で示した部分がサブクラスタ0、青色で示した部分がサブクラスタ1である。
また、クラスタごとの特徴量平均の比較を図\ref{fig:average2}に、クラスタごとのZ-score(平均からの相対距離)を図\ref{fig:Z-score2}に示す。

\begin{figure}[H]
  \centering
  % --- 左側の画像 ---
  \begin{minipage}[b]{0.45\linewidth}
    \centering
    \includegraphics[width=\linewidth]{fig/average2.png}
    \subcaption{クラスタごとの特徴量平均の比較(クラスタ1のみ)}
    \label{fig:average2}
  \end{minipage}
  \hfill % 左右の間に適度な空白を入れる
  % --- 右側の画像 ---
  \begin{minipage}[b]{0.45\linewidth}
    \centering
    \includegraphics[width=\linewidth]{fig/Z-score2.png}
    \subcaption{クラスタごとのZ-score(クラスタ1のみ)}
    \label{fig:Z-score2}
  \end{minipage}
  
  \caption{バッハのコラールの分析をヒートマップで可視化したもの(クラスタ1のみ)}
  \label{fig:total2}
\end{figure}


\subsection{考察}
分析の結果,
図\ref{fig:Z-score2}より、サブクラスタ0は半音符の割合が高く、四分音符の割合が低くなっています。曲の長さもサブクラスタ0とサブクラスタ1では相違が見られます。
このことから、サブクラスタ0はリズムがゆったりとした曲が多く含まれていると推測できます。

\section{分析3回目}
サブクラスタ1を見た所、まだ約300曲含まれていたため、さらに分類ができるのではないかと思い、クラスタ1のみを使い再びクラスタリングを行った。
PCAによる散布図を図\ref{fig:cluster11pca}に示す。

\begin{figure}[H]
  \centering
  \includegraphics[width=0.8\linewidth]{fig/cluster11.png}
  \caption{バッハのコラールクラスタ1をPCAで可視化した散布図(サブクラスタ1のみ)}
  \label{fig:cluster11pca}
\end{figure}

また、クラスタごとの特徴量平均の比較を図\ref{fig:average11}に、クラスタごとのZ-score(平均からの相対距離)を図\ref{fig:Z-score11}に示す。

\begin{figure}[H]
  \centering
  % --- 左側の画像 ---
  \begin{minipage}[b]{0.45\linewidth}
    \centering
    \includegraphics[width=\linewidth]{fig/average11.png}
    \subcaption{クラスタごとの特徴量平均の比較(サブクラスタ1のみ)}
    \label{fig:average11}
  \end{minipage}
  \hfill % 左右の間に適度な空白を入れる
  % --- 右側の画像 ---
  \begin{minipage}[b]{0.45\linewidth}
    \centering
    \includegraphics[width=\linewidth]{fig/Z-score11.png}
    \subcaption{クラスタごとのZ-score(サブクラスタ1のみ)}
    \label{fig:Z-score11}
  \end{minipage}

  \caption{バッハのコラールの分析をヒートマップで可視化したもの(サブクラスタ1のみ)}
  \label{fig:total11}
\end{figure}


\subsection{考察}
分析の結果、図\ref{fig:Z-score11}より、サブクラスタ2とサブクラスタ3では大きな相違が見られている。
しかし、図\ref{fig:average11}を見た所、例で言うとdur\_sixteenthの割合が他のクラスタが0に近いのに対し、サブクラスタ2は0.07であり、全体的に数値が小さくなっていることが分かる。
また、図\ref{fig:cluster11pca}を見ると、各クラスタに大きな境がなく、クラスタリングがうまくいっていない可能性があると考えられる。このことから、クラスタリングを行い過ぎていると考えられる。
また、サブクラスタ1に含まれている曲がバッハが作成するコラール曲の特徴を多く含んでいるものと推測できる。

\section{分析に対するまとめ}
本研究では、バッハのコラール曲を特徴量に基づいてクラスタリングし、それぞれのクラスタの特徴を分析した。最初のクラスタリングでは、曲の長さや音域、リズムなどの特徴に基づいて2つのクラスタに分類された。
さらに、クラスタ1を再びクラスタリングすることで、より細かな分類が可能となった。その結果、サブクラスタ1に含まれる曲がバッハが作成するコラール曲の特徴を多く含んでいることが示された。
しかし、過度なクラスタリングにより、クラスタ間の明確な境界が失われる可能性があることも示された。また、今回はバッハが作成するコラール曲のみで分析を行ったため、他の作曲家やジャンルのと比較しての分析は行っていない。
他の楽曲に対しても同様の分析を行うことで、より、バッハが作成するコラール曲の特徴が理解できるだろう。

\section{長さによる影響}
先ほどのまとめで、楽曲の長さがクラスタリングに影響を与えている可能性があると述べた。そのため、楽曲の長さが各特徴量にどのような影響を及ぼしているのかを相関係数を用いて考えてみる。

\vspace{20pt}
ここから先は研究途中であるため省略する。

\section{長さに影響を及ぼす値を省いた分析}
楽曲の長さに影響を及ぼすと考えられる特徴量を省き、再度クラスタリングを行った。

\vspace{20pt}
ここから先は研究途中であるため省略する。


\section{今後の課題}
今後の研究では、特徴量の選択やクラスタリング手法の改善を検討し、より精度の高い楽曲分類を目指していく。
その際、特徴量に対しては、楽曲分析を行っている先行研究を参考にし、より音楽的な意味合いを持つ特徴量を多数選択、抽出し、分析に必要な特徴量の取捨選択をすることが重要だと考えている。
また、バッハが作成するコラール曲以外の楽曲も含め同様の分析を行い、比較研究を進めることで、楽曲の特徴や構造に関する理解を深めていきたいと考えている。橋田准教授より、データセット(PEDB)への運用を提案されたため、
今後の研究ではPEDBを用いた分析も検討していく。

\end{document}

\begin{acknowledgment}
	本稿の執筆にあたり,参考文献に挙げた方々のWebサイト,スライド,各種資料を大いに参考にさせていただいた.
\end{acknowledgment}

