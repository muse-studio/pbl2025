\section{はじめに}

舞台演出などにおいて、照明は演奏表現を視覚的に拡張する重要な要素である。
しかし、従来の舞台照明は照明オペレーターによる手動操作に大きく依存しているという課題がある。

近年では、音響信号を用いて照明演出を自動化する研究\cite{tsukihigashi25}や楽曲の印象に基づいて照明を生成する研究\cite{kanno21}が報告されている。
また、演奏者の操作や身体動作と照明を連動させる試み\cite{asada20}も行われている。
しかし、これらの多くは音響情報または身体動作のいずれかに基づくものであり、両者を統合してリアルタイムに照明制御を行う手法は十分に検討されていない。

そこで本研究では、ヴァイオリン演奏に着目し、音響情報と身体動作を組み合わせ、リアルタイムで照明の制御を行うシステムの構築を目的とする。

\section{提案システム} 

本研究では、ヴァイオリンの演奏音から抽出した音響特徴量に基づいて照明を制御すると同時に、演奏者の運弓動作を照明の上下動作として直接反映するシステムを提案する。
本システムの構築は、以下の生成フローから成る。

\noindent \textbf{(1) 演奏音入力と音響特徴量抽出}

演奏音をマイクを通して 1 秒ごとに取得することで、1秒単位で音響特徴量が抽出される。

抽出される特徴量は、音の明るさや周波数分布を表すスペクトル系、音量や変化の速さを捉える時間領域系、人の聴覚特性に基づく音色を表す MFCC 系、調性や和声構造を示すハーモニー系の4種類に大別される。

\noindent \textbf{(2) 機械学習モデルによる照明特徴量推定}

(1)にて抽出した音響特徴量を入力として、学習済みの機械学習モデルにより照明特徴量を推定する。

モデルは、照明のRGB(Red:赤、Green:緑、Blue:青)の各成分・明るさ・点滅速度の3要素を出力する。

\noindent \textbf{(3) 演奏動作取得}

演奏者は右手首付近にスマートフォンを装着しながら演奏を行う。
これにより、演奏者の弓の上下動作をジャイロセンサとして取得し、その動きをリアルタイムに照明制御へ反映している。

具体的には、上げ弓と下げ弓に応じて照明の上下動作を切り替えることで、演奏者が弓を動かす方向と同じタイミングで照明も上下に動く構成となっている。

\noindent \textbf{(4) DMX信号と照明制御}

(2)にて推定された照明の色相、明るさ、点滅速度は、0〜255 の DMX 値に変換され、USB-DMX インタフェースを介して照明機器に出力される。
また、(3)にて取得した演奏者の運弓動作(上げ弓・下げ弓)に基づき、照明の上下動作も同様に DMX 信号として生成される。

\begin{figure}[tb]
	\centering
	\includegraphics[width=\hsize]{../fig/gurenge-play.png}
	\caption{『紅蓮華』実演中の照明変化の様子(左から「イントロ」、「Aメロ~Cメロ」、「サビ」、「サビ」)}
	\label{fig:gurenge-play}
\end{figure}

\section{実演と評価}

提案システムの有効性を確認するため、筆者自身がヴァイオリン演奏による実演を計2回行った。
1回目は『君をのせて』、2回目は『紅蓮華』を演奏した。

実演は 2 回とも屋内(それぞれ別場所)で行い、演奏者の周囲四方向に DMX 対応照明機器を 4 台設置した。
照明機器は地面に配置し、床面から天井方向へ向けて照射される構成とした。
これにより、四方から演奏者を照らす光環境を構築した。

両実演において、音量変化に応じて照明の明るさが連続的に変化し、楽曲の展開に伴って照明色が滑らかに遷移する様子が観察された(\figref{fig:gurenge-play})。

さらに、スマートフォンのジャイロセンサから取得した運弓動作に基づき、弓の上下動作に同期して照明が上下動作することを確認した。

以上の結果から、演奏者が追加操作を行うことなく、音響情報および身体動作と連動したリアルタイム照明制御が可能であることを確認した。

\section{おわりに}

本研究では、ヴァイオリン演奏音の音響特徴量と演奏者の身体動作を統合的に用いたリアルタイム照明制御システムを提案した。
音響情報と身体動作の双方を照明演出に反映することで、演奏者主体の一体感のある照明演出の可能性を示した。
