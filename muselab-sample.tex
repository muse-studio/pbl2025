% !TEX root = _main.tex
% ========================================
% 卒業論文 本文
% ========================================

%1
\section{はじめに}
今年度行った研究活動はサウンドスケープという,1970年代にカナダの作曲であるマリー・シェーファーによって提唱された概念に関するものである\cite{article1}.

サウンドスケープとは「音の風景」という意味で,音もその場所の個性として捉える概念である.人々が「ある地域を想像してください」と言われたら,おそらく多くの人はその地域の視覚的な風景を思い浮かべると思われる.しかし,地域のアイデンティティには音も関わっていると考えることもでき,その土地においての,チャイムやサイレンの音,虫の声や川のせせらぎも地域の標識になる.

以上のようにサウンドスケープは地域に関わるテーマであり,地域での活動を方針としている大学に在学している筆者は「地域に関連する研究活動がしたい」と思っていたため,今年度の研究をサウンドスケープにした.

活動の内容としては,前期にサウンドマップを想定した音の収集を行い,その後後期に音の応用の手段を変更して,ノイズミュージックの制作を行った.また前期の5月にヤマハ株式会社の棚瀬氏に来ていただき,3Dマイクを用いた環境音の収録も行った.以降の章では、前期に行ったサウンドマップの音の収集,棚瀬氏と共同で行った音の収録,後期に行ったノイズミュージックの作成についての活動を記録する.

%2
\section{サウンドマップの音の収集}

\subsection{活動の方針}

2024年度の前期期間は,1期生の制作した「ふくちやまっぷ」\cite{article2}というサウンドマップ(その場所ではどのような音を聞くことができるかを示した地図)の拡張をテーマとし,その音の収集を行った.

\begin{figure}[H]
	\includegraphics[width=\hsize]{i1.png}
    \caption{橋田研究室1期生の作成した「ふくちやまっぷ」}
\end{figure}

この活動において当初目標としたのは2つある.1つ目は掲載スポットを増やすこと,2つ目は季節・時間帯ごとの同じ場所の異なる音を収録することである.

1つ目の掲載スポットの拡張について,これは背景として1期生の作成したものが福知山市の市街地に限っていたというものがある.三和・夜久野・大江を中心に,さらに広い範囲で福知山市の音を収録することを方針とし,計11箇所のスポットの収録をした.

2つ目の同じ場所における複数の音の収録について,このようにしようとした理由は,同じ場所でも状況によって音が変わるからである.例えば大学の学舎のエントランスの音を対象にしたときを考えると,学生の多いにぎやかな午前中と,学生の少なくなった静かな夕方とで,音の様子は変わる.また,ある地点で何かしらのイベントがあったとき,その機会でないと収録することができない音もある.加えて,同じ音でも角度や対象からの距離で聴こえ方は異なる.そこで,前期に想定した「ふくちやまっぷ」の拡張にあたっては,同じ場所の音をより追求することも方針とした.ただ,今年度の時点では音の収録は収録地それぞれ1回ずつのみ行っているという状況である.

また,音の収録はSENNHISER社のショットガンマイクを使って行った.

\subsection{収録地の解説}

ここでは,前期から夏季休業にかけて音の収録で赴いた各スポットについて,そのスポットの大まかな概要と,どのような音を収録したかを記述する.

\vskip\baselineskip
\noindent
1.観音寺(福知山市観音寺 1067)
観音寺は福知山市の東部にある寺で,鶏の放し飼いが行われている.今回収録の対象としたのはその鶏の鳴き声であり,間近で聴く鶏の鳴き声に加え,鶏から離れた寺の敷地内の地点で遠くから響いてくる鶏の鳴き声の2種類の音を収録した.観音寺は敷地内であれば鶏の鳴き声が聴こえてくる環境であり,寺の中では鶏が身近な存在に感じるということを示すことができるような音を収録した.

\begin{figure}[H]
	\includegraphics[width=\hsize]{i2.png}
    \caption{観音寺}
\end{figure}

\vskip\baselineskip
\noindent
2.廣谷神社(福知山市三和町大原191-1) 
廣谷神社は三和町内の国道137のはずれにある神社である.廣谷神社の場所は住宅地から離れていることもあり,境内は静かな環境で,風の音や木の揺れなどの自然の音に集中できるような場所であった.そして,廣谷神社は道路のわきにあるため車が走る音が聴こえるが,ひたすらたくさんの車が走っているのではなく,数分に1回1〜3台の車が走っているという様子であった.そして,そのような環境では自然の音の中に車が走る音も,騒音ではなく神社の環境音として感じることができた.

\begin{figure}[H]
	\includegraphics[width=\hsize]{i3.png}
    \caption{廣谷神社}
\end{figure}

\vskip\baselineskip
\noindent
3.三和荘(福知山市三和町寺尾4)
三和荘は宿泊施設やスポーツ施設などを備えた交流施設である.三和荘はそこまで人が多くいる施設ではなく,落ち着いた場所だと感じた.収録したのは敷地内のテニスコートで人々がテニスをしているときの音であり,静かな環境の中で人々がテニスで遊んでいるのどかな雰囲気を捉えられるようにした.

\begin{figure}[H]
	\includegraphics[width=\hsize]{i4.png}
    \caption{三和荘}
\end{figure}

\vskip\baselineskip
\noindent
4.大原(福知山市三和町大原)
大原は三和町の大字であり,ホタルが見られるスポットとして知られている\cite{article3}.大原に赴いた時間帯は夜であったため当時は動物の鳴き声が聴こえてくる環境であった.夜の静かな時間帯に動物の鳴き声が聴こえてくる様子を収録できた.

\begin{figure}[H]
	\includegraphics[width=\hsize]{i5.png}
    \caption{大原}
\end{figure}

\vskip\baselineskip
\noindent
5.土師川流域(福知山市三和町大身)
土師川は福知山市を流れる一級河川であり,この収録では三和町の長谷川の間近の場所まで行き,川の流れの音を収録した.収録では,マイクの向きや位置を変え,複数の角度や川からの距離で聴こえるそれぞれの音を収録した.

\begin{figure}[H]
	\includegraphics[width=\hsize]{i6.png}
    \caption{土師川流域}
\end{figure}

\vskip\baselineskip
\noindent
6.天岩戸神社(福知山市大江町佛性寺字日浦ケ嶽206)
天岩戸神社は天照大神がかつて降臨したとされる場所にある神社である\cite{article4}.この神社の規模は小さく,敷地の様子は川のほとりの近くの斜面の上に祠があるというものである.収録では,敷地内を流れる川の音や,祠の鐘を鳴らしたときの音を録った.山間部にある小規模な神社であり,人の話し声や車の音が聴こえてこないため,そこでは市街地と比べて環境の音が鮮明に聴こえると感じた.

\begin{figure}[H]
	\includegraphics[width=\hsize]{i7.png}
    \caption{天岩戸神社}
\end{figure}

\vskip\baselineskip
\noindent
7.元伊勢内宮皇大神社(福知山市大江町内宮217)
元伊勢内宮皇大神社は現在では伊勢神宮で祀られている神を過去に一時的に祀っていたとされる神社である\cite{article5}.境内は山間部にあり,同日に収録した6.の天岩戸神社とは異なり近くに川もないため,周囲の鳥や虫の鳴き声がよく聴こえてくる環境であった.この神社も人工的な音が少ない場所であったため,収録する音がはっきりと聴こえると感じた.

\begin{figure}[H]
	\includegraphics[width=\hsize]{i8.png}
    \caption{元伊勢内宮皇大神社}
\end{figure}

\vskip\baselineskip
\noindent
8.新童子橋(福知山市大江町佛性寺)
新童子橋は京都府道9号線の上にかかっている木製の吊り橋である.橋の下から車の走る音が聴こえてくるというのがこの収録地の特徴で,収録環境は橋が道路から約10mほどの高さがあり,車からある程度がある,加えて2.の廣谷神社と同じく一度に走る車の数が市街地に比べて少ないというものである.そしてそのような環境で音を録った結果,そこでも車の音を騒音ではなく心地よいと感じる環境音の一つとして捉えられた.

\begin{figure}[H]
	\includegraphics[width=\hsize]{i9.png}
    \caption{新童子橋}
\end{figure}

\vskip\baselineskip
\noindent
9.綾部大江宮津線のトンネル(福知山市大江町佛性寺)
京都府道9号線の綾部大江宮津線にはトンネルがあり,そこで音の収録を行った.トンネル内は音が響くという環境であり,現地に赴いた際はトンネルの中で歩いたときの足音,自分で数回手を叩いたときの音,車がトンネル内を通過するときの音それぞれが反響する様子を収録した.

\begin{figure}[H]
	\includegraphics[width=\hsize]{i10.png}
    \caption{綾部大江宮津線のトンネル}
\end{figure}

\vskip\baselineskip
\noindent
10.毛原の棚田(福知山市大江町毛原)
大江町毛原には棚田が形成されている.収録対象は水車小屋を流れる水の音と,農家が農作業をしている際の機械の音で,観光地としての側面が強い棚田での.人々の生活感を捉えられるようにした.

\begin{figure}[H]
	\includegraphics[width=\hsize]{i11.png}
    \caption{毛原の棚田}
\end{figure}

\vskip\baselineskip
\noindent
11.富久貴の滝(福知山市夜久野町畑)
富久貴の滝は夜久野町の山間部を流れる滝である.収録対象は滝の流れの音になるが,地点による音の聴こえ方の違いを収録することを目的とし,5.と同じく滝からの距離の異なる複数の地点の音を収録した.5.と同じ水の流れの音を収録することになったが,富久貴の滝の音は土師川の音に比べて,音色が太く低いと感じた.

\begin{figure}[H]
	\includegraphics[width=\hsize]{i12.png}
    \caption{富久貴の滝}
\end{figure}

\subsection{振り返り}
前期期間から夏季休業にかけて計11箇所の収録を行ったが,反省点として収録地の選定の基準がはっきりしていなかったということがある.本来音の収録にあたっては,その場所でどのような音を録ることができそうか,どのような音を録ればその場所の雰囲気を再現することができるのか事前に考える必要があると思われるが,今回の収録地の選定ではそれらの事項を考慮しておらず,とりあえずたくさん収録すればよいとしか考えていなかった.

%3
\section{棚瀬氏と共同で行った音の収録}

\subsection{活動の概要}
前期の5月18日に,ヤマハ株式会社の棚瀬氏に来ていただき,筆者を含めた学生3人と担当教員と共同で福知山市の音の収録を行った.このフィールドワークでは3Dマイクを使い,360度の角度に対応した立体的な音を収録した.

\subsection{収録地の解説}
2章と同じく,各スポットについて,そのスポットの大まかな概要と,どのような音を収録したかを記述する.

\vskip\baselineskip
\noindent
1.やくの玄武岩公園(福知山市夜久野町小倉98-1)
やくの玄武岩公園は田倉山の火山活動により流出したマグマが固まってできた柱状節理のある公園\cite{article6}で,音の収録における特徴となると感じたのは玄武岩の側面を滝が流れている様子である.しかし,途中で公園を訪れた他の観光客の話し声や,公園の裏の道路をバイクが走る音が聴こえる場面があり,これらも収録した.棚瀬氏からはこれらもその場所を構成する音として捉えられるという話をしていただいた.さらに,滝の音に関しても,地点や角度を変えながら収録を行った.

\begin{figure}[H]
	\includegraphics[width=\hsize]{i13.png}
    \caption{やくの玄武岩公園}
\end{figure}

\vskip\baselineskip
\noindent
2.福知山城の明智光秀の自販機(福知山市内記1)
福知山城の敷地には声優の声が流れる自動販売機があり,ここでは自動販売機のドリンク購入時に流れる声優の声を収録した.声優のセリフは基本は

\begin{enumerate}
    \item 硬貨投入時
    \item 商品選択時
    \item 商品落下時
\end{enumerate}
の3種類があり,加えて限定フレーズとして7時~9時,12時~13時,18時~20時のとき,硬貨投入時のセリフがそれぞれの時間帯限定のものになる\cite{warticle7}.基本の硬貨投入時のセリフは「ときは今!明智光秀.ここに見参!」であるが,自動販売機のボイスを収録した時間帯は18〜20時の時間帯で,そのときのセリフは「そなた!ただものではない!」というものであった.

\begin{figure}[H]
	\includegraphics[width=\hsize]{i14.png}
    \caption{福知山城の明智光秀の自販機}
\end{figure}

\vskip\baselineskip
\noindent
3.明智薮(福知山市内記1丁目102-2)
明智薮は由良川の流域に築かれた人工的に植林された竹藪からなる堤防で\cite{article8},収録した音は周辺の動物の鳴き声や川の音などの環境音である.1.のやくの玄武岩公園の収録と比べて収録時間は短かったが,収録後に「明智薮は人の手が入っていない所が多いから,特に夜になるといろんな動物や虫の声が聴けるかもしれない」という,明智薮のサウンドスケープの展望についての話をしていただいた.

\begin{figure}[H]
	\includegraphics[width=\hsize]{i15.png}
    \caption{明智薮}
\end{figure}

%4
\section{ノイズミュージックの作成}
\subsection{活動の方針}
今年度の当初は,最後までサウンドマップの作成に取り組む予定でいたが,その活動は収録した音を掲載するというのが主であるため,活動としてはやや単調なものだと感じていた.そして研究テーマについて無理にサウンドマップにこだわる必要はないと感じ,後期の開始と同時に,担当教員と相談して環境音の応用をノイズミュージックの作成にした.ノイズミュージックとは通常音楽の要素として見なされないものを用いて構成された音楽であり,今回の活動では環境音を楽曲の構成要素とした.環境音を収録する場所は,福知山市街地北部の由良川の河川敷にした.理由は川の流れの音や動物・虫の鳴き声などの様々な種類の環境音を収録することができるからということに加え,筆者の家からの距離が比較的近く,収録に赴きやすいからということにある.また後期からのこの活動の音の収録においては,収録する場所はこの1箇所のみに絞った.理由は,場当たり的に多くの場所の音を1回ずつ収録するよりも,1つの場所に絞ってその音を何度も収録した方が,限られた期間でその場所のサウンドスケープを深掘りできると考えたからである.

\begin{figure}[H]
	\includegraphics[width=\hsize]{i16.png}
    \caption{ノイズミュージックの作成での収録地(赤い円で囲んだ部分)}
\end{figure}

今回収録した音の対象は,その場そのときで聴こえてくる音に加えて,木の枝を折る音や,湿った土を繰り返し踏んだときの音などの,人為的な操作によって生じるものを含める.しかし,それを収録場所の音とするには条件があり,それはその音を得るための行為の対象が収録地にあるものであるということである.例えば収録地の木を揺らしてその音を収録する,という行為は木が収録地にあるため収録音はその場所の音として捉えられるが,収録地に楽器を持っていきそれを演奏したときの音を収録する,という行為については,楽器がその場所固有のものではないため,収録音をその場所の音として捉えることはしない.

\subsection{収録した音}
由良川の河川敷で収録したものは,以下の通りである.
\begin{itemize}
    \item 木の枝を折る音
    \item 木を揺らす音
    \item 草の上を歩く音
    \item ぬかるんだ土を繰り返し踏んだときの音
    \item 石の階段の上をかける音
    \item 付近の工場の機械が稼働する音
    \item 河川敷にある鉄の箱を拳でノックする音
    \item 近くの線路を電車が通る音
    \item 虫や動物の鳴き声
\end{itemize}

\subsection{作成環境}
楽曲の作成はApple社の音楽制作ソフトであるLogicProを使って行った.このソフトにはQuickSamplerという機能があり,インタフェースに音声ファイルをドラッグすることでその音声のサンプリングができる.音楽においてのサンプリングとは,既存の音声の一部を抽出してそれを音源として用いる手法で,サンプリングした音声は,それをそのまま音源として使用でき,ピッチの変更や波形の加工などの操作ができる.今回の研究活動では,その機能で得たサンプリング音源をソフト内のピアノロールに打ち込むことで楽曲の制作をした.

\begin{figure}[H]
	\includegraphics[width=\hsize]{i17.png}
    \caption{QuickSamplerのインタフェース}
\end{figure}

\begin{figure}[H]
	\includegraphics[width=\hsize]{i18.png}
    \caption{のインタフェース}
\end{figure}

\subsection{楽曲の解説}
楽曲は合計19分21秒の長さのものとなった.この楽曲は途中で曲調が変わるものであるが,今回これらを9つの楽章に分けた.以下で,9つの楽節それぞれについての解説をする.

\vskip\baselineskip
\noindent
1.第一楽章(0:00〜0:50)
木の枝を折ったときの音をリズムの基調とし,それに鳥の鳴き声の音源を加えた.鳥の鳴き声にはリバーブ(音の残響を付加する効果)を加え,鳥の声が響き渡っている様子を再現できるようにした.

\vskip\baselineskip
\noindent
2.第二楽章(0:50〜2:16)
収録地付近の工場の機械の音を使用したが,この音源のキーを上げると笛のような音になり,それを用いたフレーズを作った.

\vskip\baselineskip
\noindent
3.第二楽章(2:16〜4:12)
キーを上げるとシンセサイザーのような音になった風の音があり,これでメロディーを作った.途中からは,これに加えて第一楽章の鳥の鳴き声の音源を用いたメロディーも作った.

\vskip\baselineskip
\noindent
4.第四楽章(4:12〜5:17)
夜の虫の声を用いたメロディーをフレーズの中心とし,河川敷にある鉄の箱をノックした音のパートを加えた.鉄の音にリバーブとオーバードライブ(音圧を上げて音を歪ませること)を加えることで,不気味な雰囲気のフレーズを作ることができた.

\vskip\baselineskip
\noindent
5. 第五楽章(5:17〜8:06)
同じフレーズを繰り返しながら徐々にBPMを下げる場面で,BPMがそれ以前の200から100になる第六章以降へのつなぎとなっている.

\vskip\baselineskip
\noindent
6.第六楽章(8:06〜12:34)
草の上を歩く足音の音源を用いたフレーズが中心となっており,そこに一定の周期で夜の鹿や鳥の鳴き声を加えて,不気味な雰囲気を表現した.

\vskip\baselineskip
\noindent
7.第七楽章(12:34〜16:44)
鹿の鳴き声でメロディーを作り,それに第二楽章とは別の工場の機械の音源のキーを上げたものを伴奏として加えた.この工場の機械の音はキーを上げると温かみのあるシンセサイザーのような音となり,第六楽章と比べて柔らかい雰囲気を作ることができた.

\vskip\baselineskip
\noindent
8.第八楽章(16:44〜18:40)
第二楽章と同じフレーズを流しながら,徐々にBPMを上げていく場面で,BPMが200に戻る第九楽章への橋渡しとなっている.

\vskip\baselineskip
\noindent
9.第九楽章(18:40〜19:21)
第一楽章の始めと同じフレーズを流し,最後に鉄の音の音源をアクセントとして鳴らして終了した.

\subsection{振り返り}
反省点は,最初に決めた指針通りの楽曲制作をしなかったということである.当初は収録地の朝から晩までの経過を音楽として表現するという目標で収録に取り掛かったが,制作においてはそのことを意識せず,手当たり次第に思いついたフレーズを並べるというものになった.

環境音だけで楽曲を制作した感想として,同じ種類の音源でも想像より多くの表現ができたということがある.音源のピッチを変更するだけでも音の印象がかなり変わったと感じた他,波形を加工する操作をすることで同じ音でも多様な表現ができた.

○楽曲のリンク
\url{https://drive.google.com/file/d/1jkFAW4rQ58HN4ak70hsjJz1fVHRZY1O6/view?usp=share_link}
