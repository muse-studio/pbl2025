% !TEX root = _main.tex
% ========================================
% 卒業論文 本文
% ========================================

%1
\section{はじめに}

%2
\section{先行研究}

\subsection{物語論}
物語とは何かを分析するためのアプローチとして,プロップ(1968)やラボフ(1967)に代表される構造主義的アプローチが存在している.

プロップ(1968)はロシアの100の民話を分析し,民話の人物たちの多様な行動(例:「主人公が出発する」「敵が主人公を騙す」「主人公が試練に勝利する」)の中に,31種類の「機能」と呼ばれる不変の行動単位を見出した.一方,ラボフ(1967)はニューヨークの日常会話で語られる個人的な体験談を分析し,口承による個人経験談には,
\begin{enumerate}
    \item 抽象(話の要約),
    \item 指向(時・場所・人物などの状況設定),
    \item 出来事(核心となる事件の展開),
    \item 評価(話の「オチ」や語る意義を示す部分),
    \item 結果(事件の結末),
    \item 締めくくり(現在への結びつけによる終結宣言)
\end{enumerate}
という6つの構造的要素が典型的に順序立てて現れることを明らかにした.

これらは,物語を1つの構造物と見なし,それを構成する普遍的な要素とその配列における規則を発見しようとするアプローチであり,多様な物語表現の背後に形式的な共通パターンを抽出したという点で重要な進展をもたらすものであった.

しかしながらこのような構造主義的アプローチには主に次のような限界が指摘できる.

第一に,物語を形式的な完成度に依存して評価する傾向があり,物語を静的なものとしてしか捉えられていないこと,第二に,物語を静的な構造物として分析の対象とすることにより、物語が実際に語られる場面や社会的な文脈が看過されがちであること.第三に,特にラボフのモデルは典型的で首尾一貫した経験談を想定しており,相手の相づちや質問によって中断され,話の方向性を修正されるという過程を通じて複数人で作られていくという日常的な語りの特徴を捉えきれないことである.結果としてこのアプローチは,物語が語り手や聞き手の人生観や世界観にいかに影響を与え,意味を生成するかという,動的で機能的な側面を十分に説明することが難しい.

こうした構造主義の限界を考慮した上で,物語を語る行為そのものの場面やそれが聞き手に与える影響という事項を含めた物語のより本質的な意味を探ることを目的とする理論を考案したのが,シフ(2012)である.シフは,物語研究の焦点を名詞の「語り」から動詞の「語る」へと移すことを提唱している.

 シフはこの語ることの最も基本的な機能は「現在化」にあるとした。現在化とは,人生への理解が、現在の状況において,また物語の語り手と聞き手の相互行為を通じて,
\begin{enumerate}
    \item 宣言的に経験に存在感を与え,
    \item 時間的に過去・現在・未来に意味の連続性を構築し,
    \item 空間的(社会的)に他者と世界の理解を共創造する,
\end{enumerate}
という三側面を持つ包括的な行為である。

\section{作品の解説}
\subsubsection{第1場面}

\begin{itemize}
  \item 1
\end{itemize}

これは,嵐の中で主人公が墜落するという物語の序盤を象徴する,激しい雷雨の環境音である.

本音声は,同場面において一貫して流れ続けており,物理的な環境の様子を示すとともに,聴者に切迫感と不安感を与える役割を担っている.

また,この雷雨の音声は場面内の他のすべての音声の背景として配置されており,視点の変更に関わらず持続的に鳴り続けることで,序盤の空気感の一貫性を支えている.

\vskip\baselineskip
\begin{itemize}
  \item 2,3,4,5,6
\end{itemize}

これらは,全体を俯瞰する視点で,嵐の中でハチドリの群れが騒いでいる混沌とした状況を表現した音声である.

この混沌さと群れとしての一体感を聴覚的に伝えるため,複数の異なるハチドリの鳴き声を重ねて再生し,それらすべてに共通してリバーブを付加している.

一般に,音源が遠方にある場合,空気中での音の減衰や反射の影響により,聴取点に到達する直接音の成分は低下し,反射音や残響音の比率が相対的に高くなる.この拡散した音響的特徴は,リバーブを付加した表現によって部分的に再現することが可能である.

本作ではこの効果を利用し,俯瞰する視点において,一定の距離からハチドリの群れの鳴き声を聴いているという距離感を聴者に与えている.さらに,複数のハチドリの声をリバーブ付きで同時に再生することで,ハチドリが群れとして行動している様子が聴覚的に伝わるようにしている.

\vskip\baselineskip
\begin{itemize}
  \item 7
\end{itemize}

これは,主人公を中心とした視点への切り替えを示す風の環境音である.

これまでの群れ全体を眺める視点で用いていた雷雨の環境音に加え,ここでは,より明瞭な風切り音を音声素材として使用している.これにより,翼が間近で空気を切っているという身体的な近接感を聴者に提示し,視点の移行を聴覚的に明示することを意図している.

鳥が飛行する際,翼や身体が空気を切ることで実際に風の音が生じるという物理的現象に基づき,従来の嵐の環境音の中でも際立って聴こえる音声を加えることで,視点が群れ全体ではなく特定の個体に変化したという印象を与えている.

このようにして,風の音がすぐそばで発生している感覚を生じさせ,視点の切り替えを聴覚的に示している.

\vskip\baselineskip
\begin{itemize}
  \item 8
\end{itemize}

これは,主人公中心の視点に切り替わった直後に配置される,主人公の鳴き声である.

ここでは,それより前の視点における群れの鳴き声(「2,3,4,5,6」)とは対照的に,音声素材にリバーブを意図的に付加せず,ドライな音質としている.これにより,音源が主人公自身,すなわち聴者の視点と近い位置にあることを示している.

また,本音声は,主人公が嵐の中で仲間を探し,呼びかけている様子を表している.この意図を強調するため,ハチドリの鳴き声の音声素材のうち,ピッチが低音から高音へと上昇する箇所を使用している.

これは,多くの言語において疑問文の語尾が上昇する傾向にあることを踏まえた表現であり,主人公が迷い,不安を抱えている心理状態を直感的に伝える効果を意図している.

\vskip\baselineskip
\begin{itemize}
  \item 9
\end{itemize}

これは,主人公と同じ群れに属するハチドリが,主人公に対して居場所を伝えている鳴き声である.

本音声は,主人公中心の視点から聴こえるものであるが,ここでは,最初の視点における群れの鳴き声と同様にリバーブを付加している.これは,仲間のハチドリが主人公から一定の距離を隔てた位置にいることを聴覚的に示すためである.

これにより,主人公が激しい雷雨の中をさまよっていながらも,仲間が認識可能な範囲に存在していることを示している.

また,音声素材としては比較的落ち着いた鳴き声を選択している.激しい雷雨の混沌とした環境音の中において,本音声は,主人公にとっての頼りとなる仲間の存在を聴覚的に示す役割を果たしている.

\vskip\baselineskip
\begin{itemize}
  \item 10
\end{itemize}

これは,主人公が仲間に対して自身の居場所を伝えている際の鳴き声である.

ここでは,「9」で用いた落ち着いた仲間の声とは対照的に,より激しく切迫した印象を持つ鳴き声の音声素材を使用している.この素材は,主人公が仲間の位置を把握した際の高揚した心理状態や,状況の緊急性を表現するためである.

さらに,この激しい鳴き声は,その後に続く,仲間のもとへ向かって飛行する場面への予兆としての役割も担っている.鳴き声に含まれる切迫感が,次の行動への移行を導く導入となっている.

\vskip\baselineskip
\begin{itemize}
  \item 11
\end{itemize}

これは,「10」に続いて,主人公が仲間のもとへ向かって急速に飛行する際,体勢転換を行う瞬間に発生する風切り音である.

ここでは,視点の切り替えにおける通常の風切り音として用いた「7」よりも,さらに瞬間的で勢いのある突風の音を使用している.これにより,空気抵抗の急激な増大を表現し,主人公が全力で移動しようとしている印象を聴者に与えている.

また,本音声は,「10」の鳴き声が持つ心理的な切迫感が,身体的な動作へと転化することを示している.この一連の流れは,内面的な緊張が外面的な行動へと移行する場面の転換点となっている.

\vskip\baselineskip
\begin{itemize}
  \item 12
\end{itemize}

これは,「11」における体勢転換と同時に,主人公が急速な飛行を開始し,その後障害物に衝突するまでの一連の動作に伴う風切り音である.

ここでは,主人公中心の視点の場面で持続的に聴こえる風切り音である「7」とは異なる音声素材を使用している.これらの素材を重ねて配置することで,緊張感を伴った飛行が継続している中に,急加速という別の動きが加わっている印象を与えている.

また,後半(1:09〜1:14)では,本音声のボリュームを−28 dBから−6 dBへと徐々に増大させている(図○参照).このボリュームの変化には,2つの意図がある.

第一に,主人公の飛行速度の増加に伴い,空気抵抗が増大することで風切り音が大きく聴こえるという物理的変化を再現することである.

第二に,主人公が障害物に衝突するという脅威に直面している状況を,ボリュームの増大によって表現し,場面の緊迫感を聴覚的に高めることである.

\vskip\baselineskip
\begin{itemize}
  \item 13
\end{itemize}

これは,主人公が「10」における鳴き声よりもさらに切迫感を強め,仲間に向かって自身の存在を呼びかけている際の鳴き声である.

激しい雷雨の中を,仲間のもとへ急速に接近していく過程において,本音声は,主人公の焦燥感や危機感が高まっている様子を聴者に与えている.

これにより,状況の切迫感が環境音だけでなく,主人公自身の鳴き声によっても表現され,物語全体の緊張がさらに高まっていることが示されている.

\vskip\baselineskip
\begin{itemize}
  \item 14
\end{itemize}

これは,主人公が障害物に衝突した瞬間の効果音である.

ここでは,視覚情報を用いない本作品において,主人公が何かに衝突したという出来事を明確に示すため,本素材にリバーブ処理を施している.

リバーブによって衝突音の持続時間が延長されることで.ドライな打撃音と比較して衝撃の大きさとその余韻が強調され,聴者に強い衝突の印象を与えることができる.

\vskip\baselineskip
\begin{itemize}
  \item 15
\end{itemize}

これは,主人公が墜落したときに発する鳴き声である.

ここでは,主人公の墜落を強調するため,最初の鳴き声に続けて,同一の音声素材を6回繰り返し配置している.そして,最初の鳴き声のベロシティを128とし,それに続く音声素材の繰り返しにおいて,52→45→28→15→6→1と,段階的に低下させている(図○参照).

このように擬似的にディレイ効果を与えることで,時間的な引き伸ばしが生じ,主人公がゆっくりと墜落していく過程に聴者の注意が向くよう意図している.

\vskip\baselineskip
\begin{itemize}
  \item 16
\end{itemize}

これは,墜落した主人公が最終的に地面へ到達した際の効果音である.

この衝突が致命的なものではなく,ある程度衝撃を緩和した状態で着地したことを示すため,ここでは,ボリュームを−1.6dBに設定している.

\vskip\baselineskip
\subsubsection{第2場面}
\begin{itemize}
  \item 17
\end{itemize}

これは,「1」とは異なり,雷を伴わない雨の環境音である.

ここでは,「1」と比較して,雨が地表に当たる際のしずくの音が明瞭に聴こえる音声素材を使用しており,主人公が地上にいるという状況を聴覚的に示すことを意図している.

これにより,第1場面に見られた切迫した緊張感は弱まっているが,雨は依然として降り続いており,完全に安心できる状態ではないことを示している.

\vskip\baselineskip
\begin{itemize}
  \item 18,19
\end{itemize}

これは,ウサギが主人公のもとへ近づいてくる際の足音である.

ここでは,「芝生の上を歩く」という音声素材と,「草むらを歩く」という音声素材を使用し,両者のMIDIノートを同じタイミングで配置することで,ウサギが跳ねて着地する際の音と,その際に草むらが揺れる音を表現している.そして,これらを一定の間隔を空けて計8回配置することで,草の上を跳ねながら移動しているウサギの様子を示している.

各足音の間隔は,1〜4歩目までは約2.2秒,4〜7歩目までは約0.6秒,7〜8歩目までは約0.7秒としており,「最初はゆっくりと近づき,ある程度距離が縮まった段階でさらに速く接近し,最後に踏ん張って一歩を踏み出す」という動きの変化を示している.

加えて,各足音のベロシティを17→22→36→49→59→67→86→124と段階的に上昇させており,ウサギが主人公に徐々に近づいてきていることが聴覚的に伝わるよう調整している(図○参照).

また,最後の一歩にあたる8歩目では,それまでと比較して,ベロシティを大きく上昇させており,ウサギが主人公のもとに到達した瞬間を強調することを意図している.

\vskip\baselineskip
\begin{itemize}
  \item 20
\end{itemize}

これは,ウサギが主人公の様子をうかがっている際に発する鳴き声である.

ここでは,短い間隔で繰り返されるウサギの鳴き声の音声素材を使用しており,「18,19」で足音のみが提示されていた存在が,威嚇や攻撃を目的としたものではないことを,聴者に直感的に示している.

また,本音声は,当該の動物がウサギであることを明確に特定させることまでは意図していないが,主人公に接近してきた何者かが,周囲の状況を確認しながら行動している存在であるという新たな情報を付加する役割を担っている.

\vskip\baselineskip
\begin{itemize}
  \item 21
\end{itemize}

これは,主人公がウサギの存在に気づいて体を起こした際の鳴き声と,その後,体を起こした状態を維持できずに再び倒れ込んだ際の鳴き声である.

ここでは,両者に共通して,同一のハチドリの鳴き声の音声素材を2回用いて,それぞれの素材を瞬間的に再生されるようMIDIノートを極端に短くして配置している.このように短いMIDIノートを用いることで,連続した鳴き声の中で生じる,瞬間的な感情や反応を表現することを可能としている.

主人公が体を起こしたときの鳴き声では,2つの音をそれぞれG2,A3に配置している(図○参照).これは「8」と同様に,疑問文の語尾が上昇する傾向を踏まえた表現であり,主人公がウサギの存在に気づき,驚いた心理状態を直感的に伝える効果を意図している.

一方,主人公が再び倒れ込んだときの鳴き声では,同じ音声素材を用いながら,音高をC3からF2へと下降させて配置している.一般に,落胆や疲労などのネガティヴな情動を伴う発声では,音高が高い位置から低い位置へと移行する傾向がみられる.この傾向を踏まえることで,主人公が力尽き,元気を失っている状態を表現している.

また,これらはいずれも瞬間的なピッチ移行によって構成されており,2つの音が分断された別個の音ではなく,同一のハチドリによる連続的な鳴き声として知覚されるように調整されている.

\vskip\baselineskip
\begin{itemize}
  \item 22
\end{itemize}

これは,主人公が「21」において体を起こしたときと,その後に再び倒れ込んだときの身体動作を示す効果音である.

本音声は,「21」における2つの鳴き声を構成する音と同時に再生されるよう配置されており,鳴き声のみでは捉えきれない主人公の身体的な状態や動きを,補完的に表現する役割を担っている.

ここでは,ウサギの足音で使用していた「草むらを歩く」を使用している.歩行によって草が揺れる際に生じる本音声は,映像作品において登場人物の体の動作を強調するために付加される,物理的な再現性よりも,直感的なわかりやすさを優先した音の用法と同様であり,主人公が体を起こし,あるいは倒れ込むといった動作を聴覚的に表現することが可能となっている.

\vskip\baselineskip
\begin{itemize}
  \item 23
\end{itemize}

これは,主人公を体に乗せる直前のウサギの鳴き声である.

本音声は,ウサギがこの後,主人公を巣穴へ運ぶために自身の体に乗せるという動作に先立って配置されており,次に何らかの行動が起こることを示す予兆として機能している.

\vskip\baselineskip
\begin{itemize}
  \item 24
\end{itemize}

これは,ウサギが主人公を体に乗せる際に発する掛け声としての鳴き声である.

ここでは,G2のMIDIノートとG3のMIDIノートを並べて配置しており,G2は動作に先立つ溜めの声,G3は実際に力を発揮する瞬間の掛け声として機能している(図○参照).

これにより,ウサギが主人公を体に乗せるために意識的に身体へ力を込めた動作を行っている様子を,聴覚的に示している.

\vskip\baselineskip
\begin{itemize}
  \item 25
\end{itemize}

これは,ウサギが自身の体を持ち上げていることに驚いた主人公の鳴き声である.

ここでは,「21」で用いた表現手法と同様に,極端に短くしたC3のMIDIノートとA3のMIDIノートを連続的に配置し,それらが同一の連続的な鳴き声として知覚されるよう調整している(図○参照).

また,疑問文の語尾が上昇する傾向を踏まえたピッチ構成とすることで,主人公が驚いている心理状態を直観的に示している.

\vskip\baselineskip
\begin{itemize}
  \item 26
\end{itemize}

これは,ウサギが主人公を持ち上げる際の動作を示す効果音である.

ここでは,「手で穴を掘る1」という音声素材を使用している.この素材には,土に手を入れる動作と,続いて土を掘り出す動作という2段階の音が含まれており,それぞれが,ウサギが主人公を頭部付近に乗せ,その体を持ち上げる一連の動作に対応するよう構成している.

「22」と同様に,本音声は,物理的な音の再現性よりも,身体動作の構造や力のかかり方を聴覚的に強調して示すことを目的とした表現である.

\vskip\baselineskip
\begin{itemize}
  \item 27,28
\end{itemize}

これは,ウサギが主人公を乗せてその場を去っていくときの足音である.

ここでは,「18」,「19」と同様に2つの音声素材を用い,約1秒間隔でMIDIノートを配置することで,ウサギが跳ねながら移動している様子を表現している.

また,ベロシティを65→60→52→48→42→33→24→11→9と段階的に下降させることで,音源が次第に遠ざかっていく印象を与え,ウサギが主人公を乗せてその場から離れていく様子を聴覚的に示している(図○参照).

\vskip\baselineskip
\subsubsection{第3場面}
\begin{itemize}
  \item 29
\end{itemize}

これは,「1」および「17」と異なる,ウサギの巣穴から聴こえる外部の雨の環境音である.

ここでは,「街の小雨」という音声素材を使用している.この素材は,雨粒が地表ではなく,舗装された道路や屋根のような硬質な面に衝突する際の音響的特徴を持っている.

この特徴により,主人公たちが直接雨にさらされる場所にいるのではなく,雨音が直接降雨から遮蔽された空間において間接的に知覚されていることが,聴覚的に伝わるようにしている.

また,本音声には強めのリバーブをかけており,雨音が巣穴内部で反射や減衰を経て聴こえる印象を付加することで,主人公たちが雨に当たらない空間にいることを強調している.

\vskip\baselineskip
\begin{itemize}
  \item 30
\end{itemize}

これは,ウサギの巣穴の中を雨粒がしたたり落ちる際の環境音である.

ここでは,「下水道」という音声素材を使用している.この素材は地下空間において水のしずくが落下する際の音響的特徴を持っており,屋外の降雨音とは異なる性質を示している.

これを用いることで,「29」と並行して,主人公たちが直接雨にさらされることのない場所にいるという印象を聴者に与えている.

\vskip\baselineskip
\begin{itemize}
  \item 31
\end{itemize}

これは,巣穴で休んでいた主人公が目覚めたときの戸惑いと,その直後に発せられる鳴き声である.

ここでは,「8」「21」と同様に,同一の鳴き声の音声素材を2回使用し,瞬間的にA2からB3へと再生されるようにしている.これによって,主人公が,自身がいつの間にか別の場所にいるという状況を即座に理解できず,戸惑っている様子を示している(図◯参照).

さらに,本音声に続いて,主人公の注意がウサギの様子へと向けられる箇所があるが,ここでも鳴き声を構成する音は瞬間的に再生されている.これにより,主人公がウサギに対して積極的な関心を示しているというよりも,状況を十分に把握できないまま,近くで発生している音や動きに反応している状態を聴覚的に表現している.

\vskip\baselineskip
\begin{itemize}
  \item 32
\end{itemize}

これは,ウサギが食べ物を咀嚼する際の効果音である.

ここでは,「rabbit-eating」という音声素材を使用しており,ウサギが何かしらの食べ物を摂取している状況を,音そのものによって直接的に示している.

また,本音声は,「31」において戸惑いを示している主人公の鳴き声と同時に配置されており,状況を把握できずにいる主人公の状態とは対照的な存在として,ウサギの落ち着いた様子を示している.

\vskip\baselineskip
\begin{itemize}
  \item 33
\end{itemize}

これは,主人公が体を起こしたときの効果音である.

ここでは,「22」と同様に「草むらを歩く」の音声素材を使用しており,「31」における主人公の目覚めと同時に配置することで,鳴き声のみでは捉えきれない主人公の身体動作を補完的に表現している.

\vskip\baselineskip
\begin{itemize}
  \item 34
\end{itemize}

これは,ウサギが食事を終了したタイミングで発する鳴き声である.

ここでは,「33」での音声の終了に加えて,ウサギの食べる動作が終了したことを,聴覚的に示している.

また,ここで一度ウサギが鳴くことは,その後に続く行動への予兆として機能している.

\vskip\baselineskip
\begin{itemize}
  \item 35
\end{itemize}

これは,ウサギが主人公に食べ物をわけようと,食べ物を転がす際に発する鳴き声である.

ここでは,「24」と同じく,B2のMIDIノートとF3のMIDIノートを並べて配置しており,前者を動作の前の溜めの声,後者を動作を実行に移すときの掛け声としている(図○参照).

\vskip\baselineskip
\begin{itemize}
  \item 36
\end{itemize}

これは,ウサギが放った食べ物が転がる際の効果音である.

ウサギが放った食べ物は硬質なものを想定しており,ここでは,「ビリヤードでポケットする1」という音声素材を使用することで,その転がる音をビリヤードのボールの音によって表現している.

また,ビリヤードのボールが転がる硬質な音が,食べ物の硬さを示す要素となっている.

\vskip\baselineskip
\begin{itemize}
  \item 37
\end{itemize}

これは,ウサギの行いに新たな戸惑いを示す主人公の鳴き声である.

ここでは,音声素材をF2→C♯2→A2の順に瞬間的に再生するようにしており,一度ピッチを下げてから再び上昇させることで,主人公が状況を即座に理解できず,一瞬立ち止まってから疑問を向ける形での戸惑いを示している(図○参照).

本音声は,これまでの単純な上行のみのピッチ変化とは異なり,思考の揺らぎを含んだ反応として知覚される.

\vskip\baselineskip
\begin{itemize}
  \item 38
\end{itemize}

これは,「37」の主人公の困惑した態度への返答となるウサギの鳴き声である.

ここでは,音声素材をピッチの変化を伴わず短く鳴くだけの音として配置しており,「32」に引き続き,主人公の鳴き声とは対照的に,落ち着いたウサギの状態を強調している.

\begin{itemize}
  \item 39
\end{itemize}

これは,「38」のウサギに対する返答となる主人公の鳴き声である.

ここでは,音声素材をB2からF2へと下降するように配置しており,一定の理解を示しつつも,なお戸惑いを残した反応を表現している(図○参照).

上昇するピッチによる疑問表現とは異なり,下降するピッチを用いることで,主人公が状況を暫定的に受け止めながらも,それを完全には整理しきれていない心理状態が示されている.

\vskip\baselineskip
\begin{itemize}
  \item 40
\end{itemize}

これは,なお戸惑いを残している主人公へのさらなる返答となるウサギの鳴き声である.

「38」ではピッチがB2であったのに対し,ここでは,D3に設定しており,それまでの返答よりも高いピッチを用いることで,ウサギが肯定的な態度を示している印象を聴者に与えている(図○参照).

また,この一連のウサギの返答は,食べ物をわけ与えられた状況を把握しきれていない主人公に対して,それを食べることを許可する役割を持っている.

\vskip\baselineskip
\begin{itemize}
  \item 41
\end{itemize}

これは,ウサギの返答を受けた主人公が,食べ物の摂取に移る直前に発する鳴き声である.

この後,主人公は実際にウサギからわけられた食べ物を摂取することになるが,本音声は,短く単一の音として鳴るものの,その次に続く咀嚼の際の効果音の主体が主人公であることを示す役割を持っている.

\vskip\baselineskip
\begin{itemize}
  \item 42
\end{itemize}

これは,主人公が食べ物を咀嚼する際の効果音である.

ここでは,「ガムを噛む」という音声素材を使用し,咀嚼動作が聴覚的に明確に伝わることを重視している.

また,MIDIノートを周期的に配置せず,不規則な間隔で並べることで,自然な食事動作のリズムを表現している(図○参照).

\vskip\baselineskip
\begin{itemize}
  \item 43
\end{itemize}

これは,主人公が食べ物を飲み込む際の効果音である.

ここでは,「飲む」という音声素材を使用し,「42」の後に配置することで,主人公が食べ物を飲み込む動作を明確に示している.

\vskip\baselineskip
\begin{itemize}
  \item 44
\end{itemize}

これは,食事が終了した後の主人公の鳴き声である.

ここでは,音声素材がそれぞれ瞬間的に F3→E3→D2→C3の順に再生されるようになっており,食事の動作の終了を示すとともに,まだ主人公に疲弊が残っている印象を聴者に与えている(図○参照).

また,この後に雷に怯える場面が続くため,本音声によって主人公の疲弊が完全には回復していないことを示しておくことで,後続場面における主人公の心理状態との連続性を保っている.

\vskip\baselineskip
\begin{itemize}
  \item 45
\end{itemize}

これは,食事を終了した主人公の様子を受けたウサギの鳴き声である.

ここでは,ピッチをF3と,「38」のB2および「40」のD3よりも高く設定することで,ウサギの喜びを含んだ肯定的な反応を示している.

また,本音声は,食事が無事に行われたことに対する肯定を,主人公に代わって表現する役割を持っている.

\vskip\baselineskip
\begin{itemize}
  \item 46
\end{itemize}

これは,突然鳴った雷に動揺する主人公の鳴き声である.

ここでは,C3の音声素材を瞬間的に再生した直後にA3の音声素材を再生することで,雷という突発的な刺激に対し,主人公が反射的に声を上げる様子を示している(図○参照).

また,この場面は第1場面で描かれた主人公の困難を想起させるものであり,主人公の心身がまだ十分に回復していないことを示唆している.

\vskip\baselineskip
\begin{itemize}
  \item 47
\end{itemize}

これは,「46」と同時に再生される落雷の環境音である.

主人公の鳴き声と同時に配置することで,雷という突発的な現象が,主人公の動揺を引き起こしている状況を明確にしている.

\vskip\baselineskip
\begin{itemize}
  \item 48
\end{itemize}

これは,雷に取り乱した後の主人公の鳴き声である.

ここでは,ハチドリの鳴き声の音声素材のうち,瞬間的な短い鳴き声の箇所を使用しており,突発的な動揺の後に生じる,一時的な緊張の収束を示している.

ただし,ここでは,完全に落ち着いた状態を示すものではなく,その後に続く反応へと移行する過程として位置づけられている.

\vskip\baselineskip
\begin{itemize}
  \item 49
\end{itemize}

これは,主人公の雷への反応に対するウサギの鳴き声である.

ここでは,これまで同場面において用いられてきたウサギの鳴き声である「38」,「40」,「45」と比較して,最もピッチの低いF2で音声素材を再生しており,ウサギが主人公を心配している印象を聴者に与えている.

一般に,低いピッチの音は高い音と比較して,興奮や緊張を抑えた印象として知覚されやすく,本作ではこの傾向を利用して,ウサギの落ち着いた心配の態度を表現している.

\vskip\baselineskip
\begin{itemize}
  \item 50
\end{itemize}

これは,ウサギの返答を受けて,精神的な疲弊が再び示す主人公の鳴き声である.

ここでは,瞬間的にA2で音声素材を再生した直後にC2での再生を行うことで,主人公が気力を再び失いつつある様子を聴覚的に表現している.

また,本場面の最後に配置される本音声は,激しい落雷の後に残る緊張感の余韻を形成している.

\vskip\baselineskip
subsubsection{第4場面}
\begin{itemize}
  \item 51
\end{itemize}
これは,第3場面の次の日の朝を告げるスズメの鳴き声の環境音である.

スズメは日の出前後に鳴き始める行動が観察されており,鳥が活発に声を出すこの時間帯の音が,視聴者に自然な朝の訪れを想起させる効果を持つ.

本音声を用いることで,平穏な朝の印象を示し,これまでの激しい雨とは対照的な場面を構築している.

加えて,本音声は,物語の登場キャラクターの鳴き声ではなく環境音として扱っており,朝の屋外空間に鳴き声が広がっている印象を表現するために,音声素材にリバーブをかけている.

\vskip\baselineskip
\begin{itemize}
  \item 52
\end{itemize}

これは,移動中のウサギの鳴き声である.

跳ねながらの移動に伴う着地音である「53」「54」と同じタイミングで本音声を再生しており,移動の主体がウサギであることを聴覚的に示している.

\vskip\baselineskip
\begin{itemize}
  \item 53,54
\end{itemize}

これは,主人公を乗せて移動するウサギの足音である.

「18」「19」および「27」「28」と同じく,芝生上を接地する音と,草むら上を接地する音の音声素材を同じタイミングで再生することによって,ウサギが跳ねて着地する様子をより豊かに表現している.

また,今回は視点がウサギ中心のものであり,ウサギが近づいてくる様子や,逆に遠ざかっていく様子の表現はないため,ベロシティは時間経過とともに変化させておらず,一定にしている.

\vskip\baselineskip
\begin{itemize}
  \item 55
\end{itemize}

これは,ウサギに声をかける主人公の鳴き声である.

ここでは,B2で一度音声素材をやや長めに再生した後,短い間を挟んでG2で音声素材を再生しており,最初の音を保ち,その後に間を設けることで,これが反射的ではなく落ち着いた反応であることを示している(図◯参照).

このような時間的余裕を持たせた表現により,主人公が動揺から回復しつつあり,比較的穏やかな状態にあることを聴覚的に示している.

また,同場面ではそれまでウサギの存在を示す音のみが配置されていた中で,ここで主人公の鳴き声を加えることで,その空間に主人公の存在があることを明確にしている.

\vskip\baselineskip
\begin{itemize}
  \item 56
\end{itemize}

これは,「55」の主人公への返答となるウサギの鳴き声である.

ここでは,ピッチを「43」と同じD2としており,ウサギの肯定的な感情を表現している.

\vskip\baselineskip
\begin{itemize}
  \item 57
\end{itemize}

これは,シカが現れる前触れとなる草の揺れの効果音である.

ここでは,「風に揺れる草木2」という音声素材を使用しており,揺れが比較的強調された草木の音を配置することで,何者かの気配があるという印象を聴者に与えている.

\vskip\baselineskip
\begin{itemize}
  \item 58
\end{itemize}

これは,「57」の草の揺れに戸惑いを伴う反応をする主人公の鳴き声である.

ここでは,「37」と同じく,F2→C♯2→A2の順に音声素材を瞬間的に再生しており,

一度ピッチを下げてから再び上昇させることで,状況を即座に理解できず,一瞬立ち止まってから反応している主人公の戸惑いを聴覚的に示している.

\vskip\baselineskip
\begin{itemize}
  \item 59
\end{itemize}

これは,「58」の主人公の反応に対するウサギの鳴き声である.

ここでは,ピッチをF3に設定することで,状況に対して注意を向けつつも,過度な動揺を示さないウサギの様子を表現している.

また,本鳴き声では,低いピッチの音の後に高いピッチの音を再生するといった操作は行っておらず,これにより,疑問を含みながらも落ち着いた態度を保っているという,

ウサギのこれまでの反応との一貫性を持たせている.

\vskip\baselineskip
\begin{itemize}
  \item 60
\end{itemize}

これは,主人公とウサギのもとに向かってくるシカの足音である.

使用条件に合致するシカの足音の音声素材が確認できなかったため,比較的大型の動物が走っていることを表現する意図のもと,ここでは,「馬が走る1」という馬の足音の音声素材を便宜的に使用している.

また,ボリュームを時間経過とともに−20 dBから−2.1 dBへと段階的に上昇させており,これによって,シカが徐々にこちらへ近づいてきている様子を聴覚的に示している(図◯参照).

加えて,原音のC3のピッチでは移動速度が速く感じられたため,それより低いC♯2にピッチを下げて音声素材を再生している.

\vskip\baselineskip
\begin{itemize}
  \item 61
\end{itemize}

これは,シカの鳴き声である.

ここでは,「baby-deer-calling-mama」という音声素材を使用しており,子鹿の鳴き声であるため,やや高い声のシカの鳴き声となっている.

「57」における草の揺れの音や,「60」の足音によって示されていた存在について,「60」の足音の再生が終了したタイミングで本音声を再生することで,その足音の主体が誰であるかをここで示している.

\vskip\baselineskip
\begin{itemize}
  \item 62
\end{itemize}

これは,シカに対するウサギの反応となる鳴き声である.

ここでは,A2で音声素材を再生した後にF3で音声素材を再生するようにしており,主人公の鳴き声(例:「37」)で用いてきた表現と同じ要領で,突然現れた存在に対するウサギの困惑を聴覚的に示している(図◯参照).

また,最初の音を瞬間的に再生しないことで,困惑を表現しつつも,これまで示してきたウサギの落ち着いた態度との一貫性も保っている.

\vskip\baselineskip
\begin{itemize}
  \item 63
\end{itemize}

これは,この次の「64」での移動の合図となるシカの鳴き声である.

シカはこの後さらに主人公とウサギに接近するが,事前に鳴き声を配置することで,その際の歩行の主体がシカであることをあらかじめ示している.

また,ここでは,「61」と同じ音声素材を使用しているが,素材の別の箇所を再生することで,シカの鳴き声が単調に聴こえることを避けている.

\vskip\baselineskip
\begin{itemize}
  \item 64
\end{itemize}

これは,「60」に続いてさらに主人公とウサギのもとに接近するシカの足音である.

ここでは,ピッチを「60」のA♯2よりさらに下げたC♯2に設定しており,シカが先ほどよりもゆっくりとした速度で接近していることを聴覚的に示している.

また,ボリュームを時間経過とともに−2.1 dBから+6 dBへと段階的に上昇させており,「60」と同じく,シカの接近を音量の変化で表現している.

\vskip\baselineskip
\begin{itemize}
  \item 65
\end{itemize}

これは,シカの接近に反応する主人公の鳴き声である.

ここでは,「48」と同様に,ハチドリの鳴き声の音声素材のうち,短い鳴き声の箇所を用いることで,驚きを示しつつも,強い動揺には至っていない様子を表現している.

また,本箇所では,これまでと同様に短い鳴き声によって驚きを表現しつつも,同じ表現の繰り返しによる単調さを避けるため,MIDIノートによるピッチ操作ではなく,音声素材内部のピッチ変化を利用している.

\vskip\baselineskip
\begin{itemize}
  \item 66
\end{itemize}

これは,「65」での主人公と同じタイミングでシカに反応するウサギの鳴き声である.

ピッチをE3と,原音のC3よりやや高く設定することで,落ち着きを残しつつも,状況をまだ把握しきれていない様子を表現している.

\vskip\baselineskip
\begin{itemize}
  \item 67
\end{itemize}

これは,同時に再生される「69」においてシカに匂いを嗅がれた際に驚く主人公の鳴き声である.

ここでは,ハチドリの鳴き声の音声素材のうち,途中でピッチが高いところから低いところに移る箇所を使用しており,過度な動揺には至らない驚きを示している様子を表現している. 

\vskip\baselineskip
\begin{itemize}
  \item 68
\end{itemize}

これは,「67」での主人公と同じタイミングでシカの行動に反応するウサギの鳴き声である.

ここでは,ピッチを,直前の鳴き声である「66」のE3よりも低いB2に設定しており,焦りを示していないが,身構えているウサギの様子を聴覚的に示している.

\vskip\baselineskip
\begin{itemize}
  \item 69
\end{itemize}

これは,シカが主人公とウサギの匂いを嗅ぐ際の効果音である.

ここでは,「イノシシが鼻をフンフン」という音声素材を使用しているが,シカが匂いを嗅ぐ音の素材が確認できなかったため,嗅ぎ取り動作として知覚される音声素材の特徴を優先し,イノシシの音を便宜的に使用している.

加えて,最初にC3で音声素材を再生し,短い間を挟んでA2で音声素材を再度再生することで,一度匂いを確認した後,より慎重に再確認する動作を表現している.後半ではピッチを下げることで,それまでよりもゆっくりとした嗅ぎ取り動作を想起させ,動物の自然な行動として知覚されるようにしている(図◯参照).

また,本音声は,「67」および「68」と同時に再生されており,主人公とウサギの反応が,このシカの行動に対する反射的なものであることを示している.

\vskip\baselineskip
\begin{itemize}
  \item 70
\end{itemize}

これは,「69」での匂いを嗅ぐ動作が終了した後のシカの鳴き声である.

ここでは,「63」と同じ音声素材を,同じピッチで再生しているが,シカがこれまでと同様の鳴き方をすることで,直前の行動によって生じた緊張が過度に持続せず,危険な存在ではないと知覚される方向へ聴取の印象を誘導している.

また,「63」ではベロシティを63に設定していたが,ここでは,それを117に設定しており,シカがより主人公とウサギの近くにいることを聴覚的に示している.

\vskip\baselineskip
\begin{itemize}
  \item 71
\end{itemize}

これは,「70」のシカの鳴き声に対する反応となる主人公の鳴き声である.

前回の主人公の鳴き声である「65」を基準として,「65」ではピッチがE3であったのに対し,ここでは,A3に設定している.このように前回の音よりもピッチを上げることによって,主人公の警戒がわずかに和らいだ状態であるという印象を聴者に与えている.

\vskip\baselineskip
\begin{itemize}
  \item 72
\end{itemize}

これは,シカが川への道案内をするために,再び主人公とウサギのもとから歩き始める場面の足音である.

ここでは,「60」と同じくピッチをC♯2に設定しており,ゆっくりとした歩行の様子を表現している.

また,この場面ではシカが物理的に主人公とウサギから距離を取り始めるため,ボリュームを時間経過とともに+6.0 dBから−2.0 dBへと減少させており,遠ざかっていく様子を聴覚的に示している(図◯参照).

\vskip\baselineskip
\begin{itemize}
  \item 73
\end{itemize}

これは,ウサギに自身についてくるよう促す際のシカの鳴き声である.

ここでは,「72」での歩行動作が終わってから約1秒の間を空けて本音声を再生するようにしており,シカが立ち止まって何かを伝えている様子を聴覚的に表現している.

\vskip\baselineskip
\begin{itemize}
  \item 74
\end{itemize}

これは,「73」でのシカの呼びかけに戸惑いを示すウサギの鳴き声である.

ここでは,音声素材をF2で再生した後,間を空けずにE3で再生しており,音高の変化によってウサギの困惑した状態を表現している.また,これまでの疑問を含む反応とは異なり,MIDIノートを短くして配置することで,戸惑いがより即時的な反応として知覚されるようにしている(図◯参照).

一方で,ピッチをF2からE3と全体的に低く設定することで,これまでに提示されてきたウサギの印象とのギャップが生じないよう配慮している.

\vskip\baselineskip
\begin{itemize}
  \item 75
\end{itemize}

これは,「74」のウサギに合わせて,シカへの戸惑いを示す主人公の鳴き声である.

ここでは,A2で音声素材を瞬間的に再生した直後にF♯3で音声素材を再生することで,「74」でのウサギに続いて,戸惑いを示す反応を表現している.

一方で,同場面の「65」および「71」と同じく短い鳴き声を使用しており,反応が瞬間的に収束することで,強い動揺には至らず,主人公がある程度の落ち着きを保っている状態であることを示している.

\vskip\baselineskip
\begin{itemize}
  \item 76
\end{itemize}

これは,静かな風によって草木が揺れる環境音である.

ここでは,「風に揺れる草木1」という音声素材を使用している.この直後の「77」にあたるシカの鳴き声の後,約15秒間登場キャラクターの鳴き声が入らない沈黙の場面が入るが,本音声は,その沈黙を完全な無音とせず,場の空気感を保持する役割を担っている.

また,「57」で使用した「風に揺れる草木2」と比較して,揺れの強調が控えめな素材を用いることで,何者かの気配を示す音ではなく,静かな環境としての持続を意図している.

また,本音声は,シカの鳴き声の直前に配置しているが,時間的な差は1秒以内とごく短く,両者はほぼ同時に知覚される.

そのため,本場面では厳密な時間差による意味付けを行うというよりも,沈黙が完全な無音にならないよう,環境音を添えることを主な目的としている.

\vskip\baselineskip
\begin{itemize}
  \item 77
\end{itemize}

これは,「74」および「75」において,状況を把握できていない様子を示した主人公とウサギへの返答となるシカの鳴き声である.

ここでは,ピッチをE3と,これまでのシカの鳴き声の中で最も高く設定しており,シカが敵意のない存在であることを聴覚的に示している.

\vskip\baselineskip
\begin{itemize}
  \item 78
\end{itemize}

これは,シカの行動や呼びかけに対して,納得を示すウサギの鳴き声である.

ピッチを原音のC3よりもやや高いE3に設定することで,これまでの戸惑いから沈黙の過程を経て,これまでの状況への理解を示したウサギの様子を聴覚的に示している.

また,本音声は,「79」および「80」でウサギが移動する動作の前触れとしての役割を担っている.

\vskip\baselineskip
\begin{itemize}
  \item 79,80
\end{itemize}

これは,シカの後を追うウサギの足音である.

これまでのウサギの移動時の足音と同様に,本音声も2つの音声素材を組み合わせて構成している.

また,ここでは,一度に移動を完了させるのではなく,ウサギが4歩移動した時点で立ち止まるようにしている(図◯参照).これは,シカが先行して歩き,それに対してウサギが距離を保ちながら追従している関係性を,時間的なずれとして表現するためである.この後の「82」においてシカが再び歩き始める場面を加えることで,ウサギがシカの後を着いていっている様子がより明確になる.

\vskip\baselineskip
\begin{itemize}
  \item 81
\end{itemize}

これは,「79,80」で接近したウサギに対し,再び先行して歩く際の合図となるシカの鳴き声である.

ここでは,A2で音声素材を再生した後にE3で音声素材を再生するようにしているが,これまでに見られた疑問や動揺を表現する意図はなく,これまでのシカの,一度鳴くだけという単調な表現の繰り返しを避けることを目的としている(図◯参照).

また,本シカの音声素材は,再生直後に息の音が含まれ,その後に明確な発声が続く構造となっている.ここでは,最初のMIDIノートを,息の音が聴こえる時点で音声が止まる長さで配置しており,明確な発声が再生されないようにしている.これにより,本音声が疑問や動揺といった感情表現としてではなく,行動の合図として知覚されるようにしている.

\vskip\baselineskip
\begin{itemize}
  \item 82
\end{itemize}

これは,再び歩き出したシカの足音である.

ここでは,ピッチは「64」および「72」と同じくC♯2に設定しており,これまでの足音と同様に,ゆっくりとした速度で歩く様子を表現している.このことから,シカがウサギの移動に合わせて進んでいく様子を聴覚的に示している.

また,ここからはウサギがシカに合わせて共に移動する場面となるため,主人公・ウサギとシカの物理的な距離が一定に保たれていることを表現する意図のもと,ボリュームは時間経過によって変化させず,一定に設定している.

\vskip\baselineskip
\begin{itemize}
  \item 83,84
\end{itemize}

これは,「82」で歩き始めたシカに合わせて一緒に移動するウサギの足音である.

本音声は,これまでのウサギの移動の表現と同じ手法を用いており,移動が継続していることを示している.

ただし,シカが先に目的地へ到着した後も,その後を追うウサギの移動は続くため,ここでのウサギの足音は,シカの移動が終了した後も約2秒間再生されるように設定している.

また,ベロシティは基本的に65としているが,シカの歩行が終了した後の最後の3歩については,101→112→127と段階的に上昇させており,ウサギがシカの示した目的地である川のふもとに到着したことを強調している.

\vskip\baselineskip
\begin{itemize}
  \item 85
\end{itemize}

これは,約38秒間にわたって歩き続けるウサギとシカの様子の背景となる風の環境音である.ここでは,「76」と同じ「風に揺れる草木1」という音声素材を使用しているが,穏やかな草木の揺れの音を持続的に配置することで,特定のキャラクターの動作を前面に出すのではなく,その場の環境全体を示す背景音として機能させている.

これにより,ウサギとシカの移動は出来事として強調されるのではなく,自然環境の中で淡々と進行する様子として知覚され,聴取者が登場キャラクターを環境の一部として俯瞰的に捉える効果を生んでいる.

\vskip\baselineskip
\begin{itemize}
  \item 86,87,88
\end{itemize}

これは,「85」と同様に,ウサギとシカの移動の場面を支える背景音として配置した鳥の鳴き声である.

ここでは,それぞれ「サンコクチョウのさえずり」「オオルリのさえずり1」「トンビの鳴き声」という音声素材を使用しており,約38秒間にわたる移動のシーンにおいて,聴覚的な印象が単調になることを避けている.

また,これらの音声にはリバーブを付加しており,「51」と同様に,一連の鳥の鳴き声が登場キャラクターの鳴き声ではなく,環境音として知覚されることを意図している.

\vskip\baselineskip
\begin{itemize}
  \item 89
\end{itemize}

これは,シカが主人公とウサギを導こうとしている川の環境音である.

ここでは,登場キャラクターが時間の経過とともに徐々に川へ近づいていく様子を表現するため,本音声の再生開始から約14秒間にわたり,ボリュームを無音から−8.0 dBへと段階的に上昇させている.

\vskip\baselineskip
\begin{itemize}
  \item 90
\end{itemize}

これは,川に到達した際のウサギの鳴き声である.

ここでは,G2で音声素材を再生した後,2秒の間を挟んでD3で再度音声素材を再生している.

このように1回目と2回目の間に十分な時間を設けることで,ピッチの上昇を伴う表現が,疑問や戸惑いではなく,シカの示した目的地に到達したことへの納得や達成感として知覚されることを意図している(図◯参照).

また,本音声は,「83,84」での足音の再生が終了した約4秒後に再生しており,これまでのシカとウサギの移動のシーンから次のシーンへ切り替わったことを示す役割を担っている.

\vskip\baselineskip
\begin{itemize}
  \item 91
\end{itemize}

これは,「90」のウサギに続く主人公の鳴き声である.

ここでは,「75」の音声を流用しており,これによって,主人公が状況に対して軽微な疑問を抱いている様子を聴覚的に示している.

主人公もウサギと同様に状況に反応していることを示しているが,「90」の時点で納得や達成感を示しているウサギとは異なり,本音声では疑問を含んだ反応にとどまっている.この違いは,この後の場面において,シカの促しを直ちに行動に移すウサギと,同様の内容の促しに対して疑問を示す主人公との態度の差として継続的に表れることになり,本音声は,その導入として位置づけられている.

\vskip\baselineskip
\begin{itemize}
  \item 92
\end{itemize}

これは,主人公とウサギに声をかけるシカの鳴き声である.

この後,シカは川の水面まで歩いて水を飲む行動に移るが,本音声は,その行動に移ることを示す合図として配置している.

同時に,本音声は,シカが自身の示す行動を通して,主人公とウサギに同様の行動を促す際の呼びかけとしても機能している.

\vskip\baselineskip
\begin{itemize}
  \item 93
\end{itemize}

これは,シカが水面まで移動する際の足音である.

ここでは,MIDIノートを歩行音が2歩分聴こえる長さで配置しており,水面と現在地との距離が大きく離れていない短距離の移動であることを示している.

また,シカと主人公・ウサギとの物理的な距離が大きく変化しない場面であるため,本音声のボリュームは一定に設定している.

\vskip\baselineskip
\begin{itemize}
  \item 94
\end{itemize}

これは,シカが水を飲む際の効果音である.

ここでは,使用条件に合致するシカの飲水音の音声素材が確認できなかったため,「dog-drinking-water-5」というイヌが水を飲む音の音声素材を便宜的に使用している.

ただし,イヌよりも体格の大きい動物であるシカが水を飲んでいる様子を表現するために,ピッチを原音のC3からA2に下げて音声素材を再生している.一般に,大型の動物ほど共鳴腔が大きく,発せられる音には低周波成分が強調されやすい傾向があるため,本操作によって音を低く,ゆったりと知覚させることで,大型動物の飲水動作として成立するよう調整している.

\vskip\baselineskip
\begin{itemize}
  \item 95
\end{itemize}

これは,「94」での水を飲む動作が終了した後のシカの鳴き声である.

本音声は,「94」の再生の終了の約1秒後に再生されるように配置しており,シカの水を飲む動作が一区切りついたことを補完的に表現している.

「93」ではピッチをB2に設定していたのに対し,ここではピッチをE3に設定しており,比較的高いピッチで鳴かせることで,主人公とウサギに対して肯定的に呼びかけるような態度を示している.この呼びかけは,直前に示されたシカ自身の行動と連続するものとして知覚されることを意図している.

\vskip\baselineskip
\begin{itemize}
  \item 96
\end{itemize}

これは,「95」のシカの鳴き声に反応する主人公の鳴き声である.

ここでは,ハチドリの鳴き声の音声素材のうち,ハチドリが3回続けて鳴く箇所を使用し,これをE3で再生している.

間隔を空けずに連続して鳴く音声を配置することで,ためらいや警戒を示す状態よりも,感情が外向きに表出している状態として知覚されやすくなる.

これによって,主人公が直前のシカの呼びかけを肯定的に受け取っている様子や,一時的に場の雰囲気が和らいだ状態を聴覚的に表現しているが,行動の判断が確定した状態を示すものではない.

\vskip\baselineskip
\begin{itemize}
  \item 97
\end{itemize}

これは,「96」の主人公に続いて,シカへの反応を示すウサギの鳴き声である.

ここでは,「90」の音声をそのまま流用しており,ウサギの状況への納得を示している.

また,本音声は,この後にウサギがシカの示した行動をすぐに実行する場面へと続いており,結果として,ウサギが状況を比較的早く受け入れて行動に移す傾向を持つ存在として知覚される構成となっている.

\vskip\baselineskip
\begin{itemize}
  \item 98
\end{itemize}

これは,この後水面まで移動する前の合図となるウサギの鳴き声である.

ここでは,ピッチを原音と同じC3に設定することで,感情的な高揚や戸惑いを示すのではなく,

状況を把握した上で次の行動に移ろうとする,ウサギの意思の確定した状態を聴覚的に示している.

また,本音声は,この後の移動および水を飲む動作の主体がウサギであることを聴覚的に示す役割も担っている.

\vskip\baselineskip
\begin{itemize}
  \item 99,100
\end{itemize}

これは,ウサギが水面まで移動する際の足音である.

本音声も,これまでのウサギの足音と同様の表現を用いているが,「93」と同じく,現在地から水面までの距離が大きく離れていないことを示すため,歩行音が3歩分のみ再生されるようMIDIノートを配置している.

\vskip\baselineskip
\begin{itemize}
  \item 101
\end{itemize}

これは,この後の「102」において水を飲む直前のウサギの鳴き声である.

ここでは,音声素材をG♯2で再生することで,高揚や戸惑いといった感情的な反応ではなく,水を飲むという次の行動に移る準備が整った,ためらいの少ない状態を聴覚的に示している.

\vskip\baselineskip
\begin{itemize}
  \item 102
\end{itemize}

これは,ウサギが水を飲む際の効果音である.

ここでは,「dog-drinking」という音声素材を使用している.これは「94」と同様にイヌが水を飲む際の音声素材であるが,シカと比較して小型の動物が水を飲んでいる様子を表現するため,水面への接触に伴う水のしぶきが比較的控えめで,軽い飲水音が中心となる音声素材を用いている.

また,ウサギは一般的にイヌよりも小型の動物であるため,小型の動物が水を飲む際の印象に近づけることを意図し,音声素材はD♯4で再生している.

\vskip\baselineskip
\begin{itemize}
  \item 103
\end{itemize}

これは,水を飲む動作が終了した後のウサギの鳴き声である.

ここでは,ウサギの鳴き声の音声素材のうち,2回続けて鳴く部分を使用し,これをD3で再生することで,動作の完了後に生じる反応として,音声が収束した印象を持つようにしている.

連続した2回の鳴き声を配置することにより,水を飲む動作に対する満足や納得といった,動作後に生じる肯定的な感情が表出しているように知覚されやすくなる.

\vskip\baselineskip
\begin{itemize}
  \item 104
\end{itemize}

これは,この後の「105」において主人公を自身の体から下ろす行動に先立つ,溜めとしてのウサギの鳴き声である.

ここでは,ウサギが主人公を体に乗せる際の鳴き声である「24」の音声を基にしているが,実際に動作を実行する瞬間に対応するG3のMIDIノートは用いず,動作に先立つ溜めの声に相当するG2のMIDIノートのみを再生するようにしている.

これにより,主人公を体から下ろす動作が,「24」における持ち上げる動作と比較して,大きな力の発揮を伴わない動作であることを聴覚的に示している.

\vskip\baselineskip
\begin{itemize}
  \item 105
\end{itemize}

これは,ウサギが主人公を自身の体から下ろした際の主人公の反応を示す鳴き声である.

ここでは,C3での音声素材を短く再生する形でMIDIノートを配置しているが,ウサギが主人公を持ち上げた際の驚きを表現した「25」とは異なり,動揺を示すピッチの上昇は用いていない.

これは,主人公にとってウサギの行動が,「25」の時点と比較して,ある程度予測可能なものとなっていることを示す表現である.

\vskip\baselineskip
\begin{itemize}
  \item 106
\end{itemize}

これは,ウサギが主人公の体を下ろした際,主人公の体が地表に触れる効果音である.

これまでウサギの足音として使用していた「草むらを歩く」の音声素材を使用しているが,ここでは,主人公の体が地表の草に触れる際の音として該当の素材を使用している.

また,ウサギの足音を表現する際は音声素材をF♯3で再生していたが,今回はA2で音声素材を再生しており,ウサギが主人公を落下の衝撃を抑えていることを聴覚的に示している.加えて,同様の目的で,ウサギの歩行の際に「草むらを歩く」と同時に使用する「芝生の上を歩く」は「106」と同時に使用しておらず,草むらの音声素材のみの使用にとどめている.

\vskip\baselineskip
\begin{itemize}
  \item 107
\end{itemize}

これは,ウサギから体を下ろされた直後の戸惑いを示す主人公の鳴き声である.

ここでは,A2で音声素材を瞬間的に再生した直後にE3で音声素材を再生するようにしており,ウサギが主人公を地表に下ろした意図を理解しきれていない主人公の疑問を,聴覚的に示している(図◯参照).

また,2回目の音声素材の再生は1回目よりも長く設定しており,発声をすぐに止めないことで,主人公の戸惑いが比較的小さい状態であることを表現している.

\vskip\baselineskip
\begin{itemize}
  \item 108
\end{itemize}

これは,主人公に声をかける際のウサギの鳴き声である.

ここでは,ウサギの鳴き声の音声素材のうち,2回連続で鳴く部分をE3のピッチで再生しており,ウサギの前向きな態度が知覚されやすくなるようにしている.

\vskip\baselineskip
\begin{itemize}
  \item 109
\end{itemize}

これは,「108」でのウサギに対して,疑問を含めた返答を示す主人公の鳴き声である.

ここでは,音声素材をA3からG3へと瞬間的に下降する形で再生しており,ピッチの上昇を用いずに,ウサギの意図を理解しきれていない主人公の疑問を聴覚的に示している.

本音声では,高めの音域内での短い下降を用いることで,ピッチの上昇を用いずに,戸惑いを含んだ反応として知覚されやすくなるようにしている.

\vskip\baselineskip
\begin{itemize}
  \item 110
\end{itemize}

これは,主人公が水面まで移動する際の足音である.
ここでは,「革靴で歩く」という音声素材を使用し,これをE3で再生することで,鳥の足が地面に接触する際の硬質な感触を表現している.
また,実際にはハチドリが移動する際にここまではっきりとした足音は生じないが,本音声は物理的な再現性よりも,聴者にとって登場キャラクターの移動動作を把握しやすくすることを優先した音表現となっている.

\vskip\baselineskip
\begin{itemize}
  \item 111
\end{itemize}

これは,「109」での疑問を経て,次の「112」で水を飲む動作を実行に移す直前の主人公の鳴き声である.

ここでは,ハチドリの鳴き声の音声素材のうち,瞬間的な短い鳴き声の箇所を使用しており,完全な理解や納得を示すものではないが,一定の思考を経て次の動作に踏み出すための判断が成立した状態を,聴覚的に示している.

\vskip\baselineskip
\begin{itemize}
  \item 112
\end{itemize}

これは,主人公が水を飲む際の効果音である.  

ここでは,「102」でのウサギが水を飲む音として使用した「dog-drinking」を再び使用しているが,主人公であるハチドリはウサギよりもさらに小型の動物であるため,音声素材は「102」のD♯4に対して,より高いピッチであるF♯4で再生している.  

また,水のしぶきが1回起こる長さのMIDIノートを約3秒間隔で計3つ配置し,その後に再び約3秒の間隔を空けて,水しぶきが4回起こる長さのMIDIノートを配置している(図◯参照).この構成により,主人公が一気に水を飲むのではなく,間を取りながら慎重に水を口にしている様子が知覚されやすくなっており,同時にこれまでの主人公の態度の一貫性を保つ表現となっている.

\vskip\baselineskip
\begin{itemize}
  \item 113
\end{itemize}

これは,「112」での水を飲む動作が終了した後の主人公の鳴き声である.

ここでは,音声素材をD4→C4→B3→A3の順で再生しており,ピッチを段階的に下降させることで,動作を終えた後の安堵感を伴う反応を聴覚的に示している(図◯参照).

\vskip\baselineskip
\begin{itemize}
  \item 114
\end{itemize}

これは,「112」での主人公の行動に反応するウサギの鳴き声である.

ここでは,「108」のウサギの鳴き声を基調として,「108」ではE3であったピッチを,ここではD♯3としている.

同一の音声表現を用いつつピッチを半音下げることで,「108」における働きかけの意図を含む前向きな態度から,主人公の行動を受け止める側の,より落ち着いた納得の反応へと,印象をわずかに変化させている.

\vskip\baselineskip
\begin{itemize}
  \item 115
\end{itemize}

これは,シカの鳴き声である.

ここでは,音声素材のピッチをB2に設定しており,直前まで高めのピッチで再生されていた主人公やウサギの鳴き声とは対照的な,低いトーンの発声となっている.

この対比によって,本音声が主人公やウサギとのやり取りとしての鳴き声ではなく,周囲の環境に対する反応であることが知覚されるようにしている.

\vskip\baselineskip
\begin{itemize}
  \item 116
\end{itemize}

これは,周囲の環境への反応を示すシカの足音である.

ここでは,足音が1歩分のみ再生される長さでMIDIノートを配置しており,これによって,本音声が移動を目的とした歩行音ではなく,「115」に続いて,シカが周囲の環境に何かを察知してとっさに体を動かした際の音として知覚されるようにしている.

また,「115」および「116」は,この後の展開に対する予兆としての役割を担っている.

\vskip\baselineskip
\begin{itemize}
  \item 117
\end{itemize}

これは,サルが現れる前触れとなる草の揺れの効果音である.

シカが現れる前に使用した「57」の音声を流用しており,何者かの気配が明確に生じたことが知覚されるよう意図して配置している.

また,本音声は,「115」「116」で示されていた予兆をここで具体化する役割を担っている.

\vskip\baselineskip
\begin{itemize}
  \item 118
\end{itemize}

これは,「117」で示された新たな存在の気配に対する動揺を示す主人公の鳴き声である.

ここでは,ハチドリの鳴き声の音声素材のうち,ピッチが低音から高音へと上昇する箇所を使用しており,これをF3で再生することで,これまでに見られた比較的落ち着いた反応とは異なる,主人公の強い動揺を聴覚的に示している.

\vskip\baselineskip
\begin{itemize}
  \item 119
\end{itemize}

これは,「118」での主人公とともに,「117」で示された存在への警戒を表すウサギの鳴き声である.

ここでは,ウサギの発声がピッチの変化を伴わずに瞬間的に再生される長さでMIDIノートを配置しており,これをF3で再生することで,ウサギが動揺よりも,警戒に近い態度を取っている様子を聴覚的に示している.

これにより,「118」において強い動揺を示した主人公とは異なり,ウサギが比較的制御された反応を示していることが対照的に表現されている.

\vskip\baselineskip
\begin{itemize}
  \item 120
\end{itemize}

これは,サルが草むらから飛び出したときの草の揺れの効果音である.  

これまでウサギの歩行音で使用していた「草むらを歩く」を使用しているが,サルという作中においてウサギよりも大きな体を持つサルが飛び出していることを表現するため,ウサギの足音では同時に1つのMIDIノートを再生していたのに対し,ここでは5つの異なるピッチのMIDIノートを同時に再生している(図◯参照).  

複数のピッチを同時に再生することで,音響的な密度や周波数帯域の広がりが生じ,単一音の音量増加では表現しきれない体の質量感や存在感が強調される.これにより,大型の動物が大きな動作で草むらを飛び出すというダイナミックな動きを,音響構造そのものによって表現している.

\vskip\baselineskip
\begin{itemize}
  \item 121
\end{itemize}

これは,「120」と同時に再生されるサルの鳴き声である.  

ここでは,「monkey」というサルの鳴き声の音声素材にリバーブを付与することで,サルが主人公たちから物理的に離れた場所にいる印象を聴者に与えている.  

また,この後の「124」において主人公がサルに対してさらなる警戒を示すことになるが,

本音声におけるリバーブの効果は,サルの位置や性質が即座には把握されない音像を形成することで,主人公にとってサルが警戒の対象として知覚される存在であることを示している.

\vskip\baselineskip
\begin{itemize}
  \item 122
\end{itemize}

これは,「120」で草むらから飛び出したサルが着地する際の効果音である.  

本音声でも,ウサギの歩行音で使用していた「芝生の上を歩く」という音声素材を使用しているが,ここでは,音声素材を瞬間的に再生することで,着地の衝撃が一度きり強く生じる印象を与えるようにしており,あわせてリバーブを付与することで,サルの着地の存在感を強調している.  

また,本音声は「120」の約3秒後に再生されるよう配置しており,サルの跳躍から着地までに一定の時間を空けることで,サルの動作の大きさや跳躍の規模感を補完的に表現している.

\vskip\baselineskip
\begin{itemize}
  \item 123
\end{itemize}

これは,サルの主人公たちの方に向かっての移動に伴う草むらの効果音である.  

ここでは,「草むらを走る」という音声素材を使用しており,本素材は,ウサギの歩行音などで使用していた「草むらを歩く」とは異なり,草の揺れ成分と足音成分がともに強調されているため,より運動量の大きい移動動作を表現するのに適した音声素材となっている.これにより,サルがこちらに向かって急速に移動していることを聴覚的に示している.  

また,ボリュームを時間経過とともに-17.4dBから+3.5dBに上昇させるよう設定しており,サルと主人公たちの物理的な距離が狭まっていく様子を表現している(図◯参照).  

加えて,本音声は「121」に続いて,主人公の警戒対象としてのサルのイメージの一貫性を形成している.

\vskip\baselineskip
\begin{itemize}
  \item 124
\end{itemize}

これは,「123」の再生中に合わせて再生される,こちらに向かっているサルへの反応を示す主人公の鳴き声である.

ここでは,ハチドリの鳴き声の音声素材のうち,一度ハチドリが鳴いた後に,再度1回目よりもピッチの高い声で鳴く部分を使用しており,サルへの警戒に対して,ためらいを含んだ反応と,それに続く強い動揺を聴覚的に示している.

\vskip\baselineskip
\begin{itemize}
  \item 125
\end{itemize}

これは,主人公たちの間近まで接近した際のサルの鳴き声である.

ここでは,「121」と同じ「monkey」という名称であるが,異なる種類の音声素材を使用している.本素材は,「121」で用いた鋭く切迫した発声とは異なり,ピッチが比較的低く,急激な立ち上がりを伴わない穏やかな発声となっている.これにより,これまで警戒対象として提示されてきたサルとは異なる印象を聴者に与えている.

\vskip\baselineskip
\begin{itemize}
  \item 126
\end{itemize}

これは,サルが移動を終了した際の反動の足音である.  

ここでは,「122」でサルの着地の際の効果音として使用した「芝生の上を歩く」を使用しており,音声素材を瞬間的に再生することで,走行の停止に伴って生じる慣性の影響により,足がワンテンポ遅れて地面に接地する動作を想起させる音として配置している.  

また,本音声にはリバーブを付加しておらず,サルが主人公たちから物理的に近い位置にいることを聴覚的に示している.

\vskip\baselineskip
\begin{itemize}
  \item 127
\end{itemize}

これは,サルがさらに主人公たちのもとへ歩行しながら接近する際の足音である.

ここでは,「芝生の上を歩く」をサルの歩行音として使用しており,ボリュームを時間経過とともに−8.3dbから0dbまで上昇させるよう設定することで,サルが徐々に主人公たちへと接近している様子を聴覚的に示している.

\vskip\baselineskip
\begin{itemize}
  \item 128
\end{itemize}

これは,「127」の再生中に合わせて再生される,サルに対する警戒を示す主人公の鳴き声である.

ここでは,ハチドリの鳴き声の音声素材のうち,相対的に音圧の抑えられた発声の後に,それよりもピッチが高く音圧の大きい発声が2回続く箇所を使用しており,低音から高音へと移行しつつ発声が反復される構造を持っている,これをE3で再生することで,注意喚起や緊急性を想起させやすい音型となっている.

これにより,「125」において比較的穏やかな発声を示したサルが提示された後であっても,主人公の警戒が依然として維持されている様子を聴覚的に示している.

\vskip\baselineskip
\begin{itemize}
  \item 129
\end{itemize}

これは,「128」と同時に再生される,主人公とは異なる形でサルに対する警戒を示すウサギの鳴き声である.

ここでは,音声素材をC2で再生しており,低いピッチでの発声により,過度な動揺を伴わず,一定の落ち着きを保った警戒の様子を聴覚的に示している.

\vskip\baselineskip
\begin{itemize}
  \item 130
\end{itemize}

これは,「127」での歩行が終了した後に再生されるサルの鳴き声である.

ここでは,「121」で使用した「monkey」を,サルと主人公たちとの距離感を考慮し,リバーブを付与せずに再生している.

また,「121」ではC3で音声素材を再生していたのに対し,本音声ではD3で再生しており,これによりサルの肯定的な感情が知覚されやすくなっている.この設定によって,「125」に続き,サルが警戒対象とは異なる存在として知覚される印象を聴者に与えている.

\vskip\baselineskip
\begin{itemize}
  \item 131
\end{itemize}

これは,サルに対して思考を含めた反応を示す主人公の鳴き声である.

ここでは,ハチドリの鳴き声の音声素材のうち,2回続けて鳴く箇所を使用しており,単発の鳴き声とは異なり,反射的な反応ではなく,一度状況を受け止めた上で応答しているように知覚されやすくなる.これを比較的低いピッチのG2で再生することで,警戒とは異なる,思考的な側面を含んだ反応である印象を聴者に与えている.

また,本音声は「130」でのサルの鳴き声の約2秒後に再生されるように配置しており,主人公の反応の対象となる鳴き声から間を置くことで,主人公が思考している様子を補完的に表現している.

\vskip\baselineskip
\begin{itemize}
  \item 132
\end{itemize}

これは,この後の「133」および「134」において,サルが主人公に草を渡す動作の前触れとなる鳴き声である.  

ここでは,「130」で使用した「monkey」の音声素材のうち,サルが軽快な様子で鳴く箇所を用い,これをD3で再生することで,サルの友好的かつ肯定的な態度を聴覚的に示している.  

また,「131」で主人公が思考を経た直後に本音声を配置することで,警戒対象として提示されてきたサルと,草を差し出す存在としてのサルとの間に生じ得る印象の断絶を緩和し,知覚の連続性を保つことを意図している.

\vskip\baselineskip
\begin{itemize}
  \item 133
\end{itemize}

これは,サルが草を渡す際の腕の動作の効果音である.

ここでは,「衣擦れ1」という,衣服のすそなどが擦れる音の音声素材を使用しているが,この音声素材は,サルが草を渡すという動作が生じたことを補完的に表現するものとして配置している.

また,本音声は,物理的な再現性よりも,視覚情報を伴わない中で登場キャラクターの動作を聴者が直感的に把握しやすくなることを意図して再生されている.

\vskip\baselineskip
\begin{itemize}
  \item 134
\end{itemize}

これは,「133」での動作に伴って生じる,サルが差し出した草の効果音である.

ここでは,「collapsing-in-grass」という,草の上に物体が倒れ込む音響的特性を持つ音声素材を使用しており,サルが主人公に差し出したものが草であることを聴覚的に示している.

また,本音声は音声素材をE3のピッチで再生している.原音であるC3では,草に倒れ込む物体の重々しさが強調される音響的特徴となったためであり,サルが手に持って差し出すことのできる程度の質量感を表現する目的から,ピッチを上げて再生している.

\vskip\baselineskip
\begin{itemize}
  \item 135
\end{itemize}

これは,サルの意図を把握できていない主人公の鳴き声である.

ここでは,ハチドリの鳴き声の音声素材のうち,「128」と同じ箇所を使用しているが,今回はMIDIノートの長さを2回目の鳴き声の鳴き始めで止まるように設定している.

音を途中で切断することで,発声が完結せず,音の行き先が提示されない構造となっており,これによって主人公が状況を理解しきれず,判断が途中で止まっている様子,すなわち困惑の状態が聴覚的に知覚されるようになっている.

また,本音声は「134」の4秒後に再生されるようになっており,「131」と同様に,直前にあった動作の効果音から間を持たせることで,主人公が状況を解釈しようとする思考の過程を補完的に表現している.

\vskip\baselineskip
\begin{itemize}
  \item 136
\end{itemize}

これは,サルが主人公に自身の意図を伝えるために,再度,注意を引くように腕を振る際の鳴き声である.

ここでは,「132」で使用した音声素材の箇所を,「132」ではD3のピッチで再生していたのに対し,今回はE3のピッチで再生しており,サルが主人公に対して,より明確に自身の意図を伝えようとしている様子を表現している.

\vskip\baselineskip
\begin{itemize}
  \item 137
\end{itemize}

これは,サルが主人公に自身の意図を伝えるために,腕を振る動作の効果音である.

ここでは,「133」で使用した「衣擦れ1」を再び使用しているが,「133」が草を渡す動作に付随する腕の動作であったのに対し,本音声は,意図を強調するためのジェスチャーとしての腕の動作を表現している.

また,「133」ではC3のピッチで再生していたのに対し,今回はD♯3のピッチで再生しており,「136」で示されたサルの積極的な呼びかけに対応して,動作そのものもより活発で軽快な印象となるようにしている.

\vskip\baselineskip
\begin{itemize}
  \item 138
\end{itemize}

これは,「137」での動作に伴って生じる,草の効果音である.

ここでは,「134」で使用した「collapsing-in-grass」を再び用いているが,「134」ではE3のピッチで再生していたのに対し,今回はG♯3のピッチで再生している.

これにより,草に触れた際の音がより軽快で鋭い印象となり,「137」での腕の動作が活発であることを,その結果として生じる音響からも補完的に示している.

\vskip\baselineskip
\begin{itemize}
  \item 139
\end{itemize}

これは,サルに差し出された草をくわえる前の主人公の鳴き声である.

ここでは,ハチドリの鳴き声の音声素材のうち,瞬間的な短い鳴き声の箇所を使用し,これを2つのMIDIノートとして並べ,B3のピッチで再生している(図◯参照).

短い鳴き声を2回連続させることで,単なる反射的な反応ではなく,行動に移る前の意識的な構えとして知覚されやすくなり,さらに比較的高いピッチで再生することにより,警戒を完全には解いていない状態で行動に踏み出そうとする,慎重さを伴った態度を聴覚的に示している.

\vskip\baselineskip
\begin{itemize}
  \item 140
\end{itemize}

これは,主人公が草を実際に食べる際に発する鳴き声である.

ここでは,「139」と同じ音声素材を使用しているが,「139」ではB3で再生していたのに対し,本音声ではC4で再生している.

このようにピッチを変更することで,「139」における主人公の心理状態を示す鳴き声とは異なり,本音声が動作に伴って生じる発声として知覚されることを意図している.

同一の音声素材であってもピッチを変化させることで,心理表現としての鳴き声と,動作 に伴う掛け声とを聴覚的に区別している.

\vskip\baselineskip
\begin{itemize}
  \item 141
\end{itemize}

これは,主人公が草を受け取る際の草の効果音である.

ここでは,サルが腕を動かす際の効果音として使用した「collapsing-in-grass」を再び使用しており,当素材をD4のピッチで0.5秒間再生した後,続けて瞬間的にF4のピッチで再生している(図◯参照).

これにより,主人公が一度草を口で保持した後,サルの手のひらから草を引き離し,受け取りを完了する一連の動作を聴覚的に表現している

\vskip\baselineskip
\begin{itemize}
  \item 142
\end{itemize}

これは,主人公が草を咀嚼する際の効果音である.

ここでは,「リンゴをかじる」という音声素材を便宜的に使用しているが,当素材は,咀嚼によって繊維質の物体がちぎれる印象を与える音響的特徴を持つため,葉を咀嚼する音の代替として使用可能であると判断した.

また,当素材はリンゴをかじる際の咀嚼音が1回だけ聴こえるものとなっているが,当音声では,このMIDIノートを計3つ配置しており,1つ目と2つ目の間に3秒間分,2つ目と3つ目の間には2秒間分の間を空けるようにしている(図◯参照).これにより,主人公が草を慎重に咀嚼しつつ,徐々に緊張や警戒を解いていく様子を聴覚的に表現している.

\vskip\baselineskip
\begin{itemize}
  \item 143
\end{itemize}

これは,主人公が草を飲み込む際の効果音である.

ここでは,「飲む」という音声素材を使用し,「142」の後に配置することで,主人公が草を咀嚼した後にそれを飲み込む動作を直接的に表現している.

これにより,摂食行為が完了したことが聴覚的に明示され,サルから差し出された草が実際に食べられたという出来事が確定的なものとして知覚されるようにしている.

\vskip\baselineskip
\begin{itemize}
  \item 144
\end{itemize}

これは,草を食べる動作が終了した後の主人公の鳴き声である.

ここでは,ハチドリの鳴き声の音声素材のうち,ピッチが低音から高音へと上昇する箇所を使用し,これをA3のピッチで再生した後,続けてD4のピッチで再生している(図◯参照).

同一の上昇音をより高いピッチで再提示する構成とすることで,行為の完了後に生じる安堵感や満足感が,単発の反応ではなく余韻を伴った状態として知覚されるよう意図している.

また,「118」と同様に上昇音を使用しているが,本箇所ではこれを反復構造として,より高いピッチで配置することで,突発的な動揺ではなく,行為完了後に生じる安定した感情表現として機能するよう設計している.

\vskip\baselineskip
\begin{itemize}
  \item 145
\end{itemize}

これは,主人公の草を食べる動作が終了した様子を受けたサルの鳴き声である.

ここでは,「132」および「136」と同じ「monkey」の箇所を用いているが,「132」ではD3,「136」ではE3で再生していたのに対し,本箇所ではC4のピッチで再生している.このピッチの上昇により,サルが自身の意図した行動を主人公が完了したことに対して示す肯定的な態度が,より明確に知覚されるようになっている.

\vskip\baselineskip
\begin{itemize}
  \item 146
\end{itemize}

これは,「145」のサルに続いて,主人公の動作の完了を受けたウサギの鳴き声である.

ここでは,音声素材をB3で再生しており,サルに続く形で,ウサギの肯定的な態度が知覚されるようにしている.

\vskip\baselineskip
\begin{itemize}
  \item 147
\end{itemize}

これは,この後の「148」においてサルおよびウサギに続いて鳴き声で反応する主体がシカであることを明示するために配置した,シカの足音である.

ここでは,「116」と同様に,足音が1歩分のみ聴こえる長さでMIDIノートを配置している.

シカの鳴き声はサルの登場以降再生されておらず,またシカの鳴き声が音響的にサルの鳴き声と混同される可能性があることから,本音声は,シカが移動している事実を示すことよりも,次に再生される鳴き声の主体がシカであることを聴覚的に予告することを主目的として配置している.

\vskip\baselineskip
\begin{itemize}
  \item 148
\end{itemize}

これは,主人公の動作完了に対するシカの鳴き声である.

ここでは,シカが自身の水飲み動作を終えた際に用いた鳴き声である「95」を流用しており,E3のピッチで再生している.

これにより,シカが主人公の動作の完了に対して肯定的な感情を示していることを聴覚的に表現しているが,同場面においてより高いピッチで発声していた「145」および「146」のサルやウサギとは異なり,感情の表出が比較的抑制された,穏やかな態度として知覚されるよう設計している.

このように,肯定的な反応であってもその示し方に差異を持たせることで,登場キャラクターごとの性格や反応様式の違いを表現している.

\vskip\baselineskip
\begin{itemize}
  \item 149
\end{itemize}

これは,本場面最後の余韻となる主人公の鳴き声である.

ここでは,ハチドリの鳴き声の音声素材のうち,相対的に音圧の抑えられた発声の後に,それよりもピッチが高く音圧の大きい発声が続く構造を持つ箇所を使用している.これを用いて,弱い発声が2回知覚された後に強い発声が現れるよう,1回目の発声のみを含むA3のMIDIノートと,2回目の発声までを含むA3のMIDIノートを,瞬間的な間隔を挟んで配置している(図◯参照).

このように,弱い発声を連続させた上で強い発声を導入する構造とすることで,弱い発声が基準として知覚された後に続く強い発声が相対的に際立つよう設計しており,主人公の鳴き声が勢いを伴った前向きな表出として知覚されやすくなっている.これにより,本場面の締めくくりとして,主人公の元気な様子を聴覚的に表現している.

\section{まとめ}


