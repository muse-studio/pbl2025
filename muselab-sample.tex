% !TEX root = _main.tex
% ========================================
% 卒業論文 本文
% ========================================

%1
\section{はじめに}

%2
\section{先行研究}

\subsection{物語論}
物語とは何かを分析するためのアプローチとして,プロップ(1968)やラボフ(1967)に代表される構造主義的アプローチが存在している.

プロップ(1968)はロシアの100の民話を分析し,民話の人物たちの多様な行動(例:「主人公が出発する」「敵が主人公を騙す」「主人公が試練に勝利する」)の中に,31種類の「機能」と呼ばれる不変の行動単位を見出した.一方,ラボフ(1967)はニューヨークの日常会話で語られる個人的な体験談を分析し,口承による個人経験談には,
\begin{enumerate}
    \item 抽象(話の要約),
    \item 指向(時・場所・人物などの状況設定),
    \item 出来事(核心となる事件の展開),
    \item 評価(話の「オチ」や語る意義を示す部分),
    \item 結果(事件の結末),
    \item 締めくくり(現在への結びつけによる終結宣言)
\end{enumerate}
という6つの構造的要素が典型的に順序立てて現れることを明らかにした.

これらは,物語を1つの構造物と見なし,それを構成する普遍的な要素とその配列における規則を発見しようとするアプローチであり,多様な物語表現の背後に形式的な共通パターンを抽出したという点で重要な進展をもたらすものであった.

しかしながらこのような構造主義的アプローチには主に次のような限界が指摘できる.

第一に,物語を形式的な完成度に依存して評価する傾向があり,物語を静的なものとしてしか捉えられていないこと,第二に,物語を静的な構造物として分析の対象とすることにより、物語が実際に語られる場面や社会的な文脈が看過されがちであること.第三に,特にラボフのモデルは典型的で首尾一貫した経験談を想定しており,相手の相づちや質問によって中断され,話の方向性を修正されるという過程を通じて複数人で作られていくという日常的な語りの特徴を捉えきれないことである.結果としてこのアプローチは,物語が語り手や聞き手の人生観や世界観にいかに影響を与え,意味を生成するかという,動的で機能的な側面を十分に説明することが難しい.

こうした構造主義の限界を考慮した上で,物語を語る行為そのものの場面やそれが聞き手に与える影響という事項を含めた物語のより本質的な意味を探ることを目的とする理論を考案したのが,シフ(2012)である.シフは,物語研究の焦点を名詞の「語り」から動詞の「語る」へと移すことを提唱している.

 シフはこの語ることの最も基本的な機能は「現在化」にあるとした。現在化とは,人生への理解が、現在の状況において,また物語の語り手と聞き手の相互行為を通じて,
\begin{enumerate}
    \item 宣言的に経験に存在感を与え,
    \item 時間的に過去・現在・未来に意味の連続性を構築し,
    \item 空間的(社会的)に他者と世界の理解を共創造する,
\end{enumerate}
という三側面を持つ包括的な行為である。

\section{作品の解説}
\subsubsection{第1場面}

\begin{itemize}
  \item 1
\end{itemize}

これは,嵐の中で主人公が墜落するという物語の序盤を象徴する,激しい雷雨の環境音である.

この音は同場面において一貫して流れ続けており,物理的な環境の様子を示すとともに,聴者に切迫感と不安感を与える役割を担っている.

また,この雷雨の音は場面内の他のすべての音の背景として配置されており,視点の変更に関わらず持続的に鳴り続けることで,序盤の空気感の一貫性を支えている.

\vskip\baselineskip
\begin{itemize}
  \item 2,3,4,5,6
\end{itemize}

これらは,全体を俯瞰する視点で,嵐の中でハチドリの群れが騒いでいる混沌とした状況を表現した音である.

この混沌さと群れとしての一体感を聴覚的に伝えるため,複数の異なるハチドリの鳴き声を重ねて再生し,それらすべてに共通してリバーブを付加している.

一般に,音源が遠方にある場合,空気中での音の減衰や反射の影響により,聴取点に到達する直接音の成分は低下し,反射音や残響音の比率が相対的に高くなる.このような拡散した音響的特徴は,リバーブを付加した表現によって部分的に再現することが可能である.

本作ではこの効果を利用し,俯瞰する視点において,一定の距離からハチドリの群れの鳴き声を聴いているという距離感を聴者に与えている.さらに,複数のハチドリの声をリバーブ付きで同時に再生することで,ハチドリが群れとして行動している様子が聴覚的に伝わるようにしている.

\vskip\baselineskip
\begin{itemize}
  \item 7
\end{itemize}

これは,主人公を中心とした視点への切り替えを示す風の音である.

これまでの群れ全体を眺める視点で用いていた雷雨の環境音に加え,より明瞭な風切り音を音声素材として使用している.これにより,翼が間近で空気を切っているという身体的な近接感を聴者に提示し,視点の移行を聴覚的に明示することを意図している.

鳥が飛行する際,翼や身体が空気を切ることで実際に風の音が生じるという物理的現象に基づき,従来の嵐の環境音の中でも際立って聴こえる音を加えることで,視点が群れ全体ではなく特定の個体に変化したという印象を与えている.

このようにして,風の音がすぐそばで発生しているかのような感覚を生じさせ,視点の切り替えを聴覚的に示している.

\vskip\baselineskip
\begin{itemize}
  \item 8
\end{itemize}

これは,主人公中心の視点に切り替わった直後に配置される,主人公の鳴き声である.

それより前の視点における群れの鳴き声(「2」,「3」,「4」,「5」,「6」)とは対照的に,ここではリバーブを意図的に付加せず,ドライな音質としている.これにより,音源が主人公自身,すなわち聴者の視点と近い位置にあることを示している.また,この鳴き声は,主人公が嵐の中で仲間を探し,呼びかけている様子を表している.この意図を強調するため,使用した音声素材の中から,ピッチが低音から高音へと上昇する部分を使用している.

これは,多くの言語において疑問文の語尾が上昇する傾向にあることを踏まえた表現であり,主人公が迷い,不安を抱えている心理状態を直感的に伝える効果を意図している.

\vskip\baselineskip
\begin{itemize}
  \item 9
\end{itemize}

これは,主人公と同じ群れに属するハチドリが,主人公に対して居場所を伝えている鳴き声である.

この音は主人公中心の視点から聴こえるものであるが,最初の視点における群れの鳴き声と同様にリバーブを付加している.これは,仲間のハチドリが主人公から一定の距離を隔てた位置にいることを聴覚的に示すためである.

これにより,主人公が激しい雷雨の中をさまよっていながらも,仲間が認識可能な範囲に存在していることを示している.

また,音声素材としては比較的落ち着いた鳴き声を選択している.激しい雷雨の混沌とした環境音(1)の中において,この鳴き声は,主人公にとって頼ることのできる仲間の存在を聴覚的に示す役割を果たしている.

\vskip\baselineskip
\begin{itemize}
  \item 10
\end{itemize}

これは,主人公が仲間に対して自身の居場所を伝えている際の鳴き声である.

ここでは,「9」で用いた落ち着いた仲間の声とは対照的に,より激しく切迫した印象を持つ鳴き声の音声素材を使用している.この素材は,主人公が仲間の位置を把握した直後の高揚した心理状態や,状況の緊急性を表現するためである.

さらに,この激しい鳴き声は,その直後に続く,仲間のもとへ向かって飛行する場面への予兆としての役割も担っている.鳴き声に含まれる切迫感が,次の行動への移行を導く導入となっている.

\vskip\baselineskip
\begin{itemize}
  \item 11
\end{itemize}

これは,「10」に続いて,主人公が仲間のもとへ向かって急速に飛行する際,体勢転換を行う瞬間に発生する風切り音である.

ここでは,視点の切り替えにおける通常の風切り音として用いた「7」よりも,さらに瞬間的で勢いのある突風の音を使用している.これにより,空気抵抗の急激な増大を表現し,主人公が全力で移動しようとしている印象を聴者に与えている.

また,この音は,「10」の鳴き声が持つ心理的な切迫感が,身体的な動作へと転化することを示している.この一連の流れは,内面的な緊張が外面的な行動へと移行する場面の転換点となっている.

\vskip\baselineskip
\begin{itemize}
  \item 12
\end{itemize}

これは,「11」における体勢転換と同時に,主人公が急速な飛行を開始し,その後障害物に衝突するまでの一連の動作に伴う風切り音である.

この音では,主人公中心の視点の場面で持続的に聴こえる風切り音である「7」とは異なる音声素材を使用している.これらの素材を重ねて配置することで,緊張感を伴った飛行が継続している中に,急加速という別の動きが加わっている印象を与えている.

また,後半(1:09〜1:14)では,この音の音量を−28 dBから−6 dBへと徐々に増大させている.この音量変化には,2つの意図がある.

第一に,主人公の飛行速度の増加に伴い,空気抵抗が増大することで風切り音が大きく聴こえるという物理的変化を再現することである.

第二に,主人公が障害物に衝突するという脅威に直面している状況を,音量の増大によって表現し,場面の緊迫感を聴覚的に高めることである.

\vskip\baselineskip
\begin{itemize}
  \item 13
\end{itemize}

これは,主人公が「10」における鳴き声よりもさらに切迫感を強め,仲間に向かって自身の存在を呼びかけている際の鳴き声である.

激しい雷雨の中を,仲間のもとへ急速に接近していく過程において,主人公の焦燥感や危機感が高まっている様子を聴者に与えている.

これにより,状況の切迫感が環境音だけでなく,主人公自身の鳴き声によっても表現され,物語全体の緊張がさらに高まっていることが示されている.

\vskip\baselineskip
\begin{itemize}
  \item 14
\end{itemize}

これは,主人公が障害物に激突した瞬間の衝突音である.

視覚情報を用いない本作品において,主人公が何かに衝突したという出来事を明確に示すため,この音声素材にはリバーブ処理を施している.

リバーブによって衝突音の持続時間が延長されることで.ドライな打撃音と比べて衝撃の大きさとその余韻が強調され,聴者に強い衝突の印象を与えることができる.

\vskip\baselineskip
\begin{itemize}
  \item 15
\end{itemize}

これは,主人公が墜落したときに発する鳴き声である.

主人公の墜落を強調するため,最初の鳴き声に続けて,同一の音声素材を6回繰り返し配置し,その都度ベロシティを徐々に低下させている.

このように擬似的にディレイ効果を与えることで,時間的な引き伸ばしが生じ,主人公がゆっくりと墜落していく過程に聴者の注意が向くよう意図している.

\vskip\baselineskip
\begin{itemize}
  \item 16
\end{itemize}

これは,墜落した主人公が最終的に地面へ到達した際の衝突音である.

この衝突が致命的なものではなく,ある程度衝撃を緩和した状態で着地したことを示すため,音量を比較的小さく(−1.6dB)設定している.

\vskip\baselineskip
\subsubsection{第2場面}
\begin{itemize}
  \item 17
\end{itemize}

これは,「1」とは異なり,雷を伴わない雨の環境音である.

「1」に比べて,雨が地表に当たる際のしずくの音が明瞭に聴こえる音声素材を使用しており,主人公が地上にいるという状況を聴覚的に示すことを意図している.

これにより,第1場面に見られたような切迫した緊張感は弱まっているが,雨は依然として降り続いており,完全に安心できる状態ではないことを示している.

\vskip\baselineskip
\begin{itemize}
  \item 18,19
\end{itemize}

これは,ウサギが主人公のもとへ近づいてくる際の足音である.

ここでは,芝生の上を歩行する足音と,草むら上を歩行する足音の音声素材を使用し,両者のノートを同じタイミングで配置することで,ウサギが跳ねて着地する際の音と,その際に草むらが揺れる音を表現している.そして,これらを一定の間隔を空けて計8回配置することで,草の上を跳ねながら移動しているウサギの様子を示している.

各足音の間隔は,1〜4歩目までは約2.2秒,4〜7歩目までは約0.6秒,7〜8歩目までは約0.7秒としており,「最初はゆっくりと近づき,ある程度距離が縮まった段階でさらに速く接近し,最後に踏ん張って一歩を踏み出す」という動きの変化を示している.

加えて,各足音のベロシティを17→22→36→49→59→67→86→124と段階的に上昇させており,ウサギが主人公に徐々に近づいてきていることが聴覚的に伝わるよう調整している.

また,最後の一歩にあたる8歩目では,それまでと比べてベロシティを大きく上昇させており,ウサギが主人公のもとに到達した瞬間を強調することを意図している.

\vskip\baselineskip
\begin{itemize}
  \item 20
\end{itemize}

これは,ウサギが主人公の匂いをかいでいる際に発する音である.

単純な呼吸音ではなく,ウサギの鳴き声を含む音声素材を使用しており,「18」で近づいてきた存在が敵対的な存在ではないことを,聴者に直感的に示している.

また,この音は,当該の動物がウサギであることを明確に伝えることまでは目的としていないが,「18」,「19」で近づいてきた何者かに関する情報が新たに付加されるきっかけとして機能している.

\vskip\baselineskip
\begin{itemize}
  \item 21
\end{itemize}

これは,主人公がウサギの存在に気づいて体を起こした際の鳴き声と,その後,体を起こした状態を維持できずに再び倒れ込んだ際の鳴き声である.

両者に共通して,同一のハチドリの鳴き声の音声素材を2回用いて,それぞれの素材を瞬間的に再生されるようノートを極端に短く設定している.このように短いノートを用いることで,連続した鳴き声の中で生じる,瞬間的な感情や反応を表現することを可能としている.

主人公が体を起こしたときの鳴き声では,2つの音をそれぞれG2,A3に配置している.これは「8」と同様に,疑問文の語尾が上昇する傾向を踏まえた表現であり,主人公がウサギの存在に気づき,驚いた心理状態を直感的に伝える効果を意図している.

一方,主人公が再び倒れ込んだときの鳴き声では,同じ音声素材を用いながら,音高をC3からF2へと下降させて配置している.一般に,落胆や疲労などのネガティヴな情動を伴う発声では,音高が高い位置から低い位置へと移行する傾向がみられる.この傾向を踏まえることで,主人公が力尽き,元気を失っている状態を表現している.

また,これらはいずれも瞬間的なピッチ移行によって構成されており,2つの音が分断された別個の音ではなく,同一のハチドリによる連続的な鳴き声として知覚されるように調整されている.

\vskip\baselineskip
\begin{itemize}
  \item 22
\end{itemize}

これは,主人公が「21」において体を起こしたときと,その後に再び倒れ込んだときの身体動作を示す音である.

この音は,「21」における2つの鳴き声と同時に再生されるよう配置されており,鳴き声のみでは捉えきれない主人公の身体的な状態や動きを,補完的に示す役割を担っている.

ここでは,「草むらを歩く」という音声素材を使用している.歩行によって草が揺れる際に生じるこの音は,映像作品において登場人物の体の動作を強調するために付加される,物理的な再現性よりも直感的なわかりやすさを優先した音の用法と同様であり,主人公が体を起こし,あるいは倒れ込むといった動作を聴覚的に表現することが可能となっている.

\vskip\baselineskip
\begin{itemize}
  \item 23
\end{itemize}

これは,主人公を体に乗せる直前のウサギの鳴き声である.

この鳴き声は,ウサギがこの後,主人公を巣穴へ運ぶために自身の体に乗せるという動作に先立って配置されており,次に何らかの行動が起こることを示す予兆として機能している.

\vskip\baselineskip
\begin{itemize}
  \item 24
\end{itemize}

これは,ウサギが主人公を体に乗せる際に発する掛け声としての鳴き声である.

G2のノートとG3のノートを並べて配置しており,G2は動作に先立つ溜めの声,G3は実際に力を発揮する瞬間の掛け声として機能している.

これにより,ウサギが主人公を体に乗せるために意識的に身体へ力を込めた動作を行っている様子を,聴覚的に示している.

\vskip\baselineskip
\begin{itemize}
  \item 25
\end{itemize}

これは,ウサギが自身の体を持ち上げていることに驚いた主人公の鳴き声である.

ここでは,「21」で用いた表現手法と同様に,極端に短くしたC3のノートとA3のノートを連続的に配置し,それらが同一の連続的な鳴き声として知覚されるよう調整している.

また,疑問文の語尾が上昇する傾向を踏まえたピッチ構成とすることで,主人公が驚いている心理状態を直観的に示している.

\vskip\baselineskip
\begin{itemize}
  \item 26
\end{itemize}

これは,ウサギが主人公を持ち上げる際の動作を示す音である.

ここでは,「手で穴を掘る1」という音声素材を使用している.この素材には,土に手を入れる動作と,続いて土を掘り出す動作という2段階の音が含まれており,それぞれが,ウサギが主人公を頭部付近に乗せ,その体を持ち上げる一連の動作に対応するよう構成している.

「22」と同様に,この音は物理的な音の再現性よりも,身体動作の構造や力のかかり方を聴覚的に強調して示すことを目的とした表現である.

\vskip\baselineskip
\begin{itemize}
  \item 27,28
\end{itemize}

これは,ウサギが主人公を乗せてその場を去っていくときの音である.

ここでは,「18」,「19」と同様に2つの音声素材を用い,約1秒間隔でノートを配置することで,ウサギが跳ねながら移動している様子を表現している.

また,ベロシティを65→60→52→48→42→33→24→11→9と段階的に下降させることで,音源が次第に遠ざかっていく印象を与え,ウサギが主人公を乗せてその場から離れていく様子を聴覚的に示している.

\vskip\baselineskip
\subsubsection{第3場面}
\begin{itemize}
  \item 29
\end{itemize}

これは,「1」および「17」と異なる,ウサギの巣穴から聴こえる外部の雨の環境音である.

ここでは「街の小雨」という音声素材を使用している.この素材は,雨粒が地表ではなく,舗装された道路や屋根のような硬質な面に衝突する際の音響的特徴を持っている.

この特徴により,主人公たちが直接雨にさらされる場所にいるのではなく,雨音が直接降雨から遮蔽された空間において間接的に知覚されていることが,聴覚的に伝わるようにしている.

また,この音には強めのリバーブをかけており,雨音が巣穴内部で反射や減衰を経て聴こえる印象を付加することで,主人公たちが雨に当たらない空間にいることを強調している.

\vskip\baselineskip
\begin{itemize}
  \item 30
\end{itemize}

これは,ウサギの巣穴の中を雨粒がしたたり落ちる際の音である.

ここでは「下水道」という音声素材を使用している.この素材は地下空間において水のしずくが落下する際の音響的特徴を持っており,屋外の降雨音とは異なる性質を示している.

これを用いることで,「29」と並行して,主人公たちが直接雨にさらされることのない場所にいるという印象を聴者に与えている.

\vskip\baselineskip
\begin{itemize}
  \item 31
\end{itemize}

これは,巣穴で休んでいた主人公が目覚めたときの戸惑いと,その直後に発せられる鳴き声である.

ここでは「8」「21」と同様に,同一の鳴き声の音声素材を2回用い,瞬間的にA2からB3へと再生されるようにしている.これによって,主人公が,自身がいつの間にか別の場所にいるという状況を即座に理解できず,戸惑っている様子を示している.

さらに,この鳴き声に続いて,主人公の注意がウサギの様子へと向けられる箇所があるが,ここでも鳴き声は瞬間的に再生されている.これにより,主人公がウサギに対して積極的な関心を示しているというよりも,状況を十分に把握できないまま,近くで発生している音や動きに反応している状態を聴覚的に表現している.

\vskip\baselineskip
\begin{itemize}
  \item 32
\end{itemize}

これは,ウサギが食べ物を食べているときの音である.

「rabbit-eating」という音声素材を使用しており,ウサギが何かしらの食べ物を摂取している状況を,音そのものによって直接的に示している.

また,この音は「31」において戸惑いを示している主人公の鳴き声と同時に配置されており,状況を把握できずにいる主人公の状態とは対照的な存在として,ウサギの落ち着いた様子を示している.

\vskip\baselineskip
\begin{itemize}
  \item 33
\end{itemize}

これは,主人公が体を起こしたときの音である.

「22」と同様に「草むらを歩く」の音声素材を使用しており,「31」における主人公の目覚めと同時に配置することで,鳴き声のみでは捉えきれない主人公の身体動作を補完的に示している.

\vskip\baselineskip
\begin{itemize}
  \item 34
\end{itemize}

これは,ウサギが食べ物を食べ終わったタイミングで発する鳴き声である.

「33」での音声の終了に加えて,ウサギの食べる動作が終わったことを,聴覚的に示している.

また,ここで一度ウサギが鳴くことは,その後に続く行動への予兆として機能している.

\vskip\baselineskip
\begin{itemize}
  \item 35
\end{itemize}

これは,ウサギが主人公に食べ物をわけようと,食べ物を転がす際に発する鳴き声である.

「24」と同じく,B2のノートとF3のノートを並べて配置しており,前者を動作の前の溜めの声,後者を動作を実行に移すときの掛け声としている.

\vskip\baselineskip
\begin{itemize}
  \item 36
\end{itemize}

これは,ウサギが放った食べ物が転がる音である.

ウサギが放った食べ物は硬質なものを想定しており,ここでは「ビリヤードでポケットする1」という音声素材を使用することで,その転がる音をビリヤードのボールの音によって表現している.

また,ビリヤードのボールが転がる硬質な音が,食べ物の硬さを示す要素となっている.

\vskip\baselineskip
\begin{itemize}
  \item 37
\end{itemize}

これは,ウサギの行いに新たな戸惑いを示す主人公の鳴き声である.

ここでは,主人公の鳴き声の音をF2→C♯2→A2の順に瞬間的に再生するようにしており,一度ピッチを下げてから再び上昇させることで,主人公が状況を即座に理解できず,一瞬立ち止まってから疑問を向けるような戸惑いを示している.

この鳴き声は,これまでの単純な上行のみのピッチ変化とは異なり,思考の揺らぎを含んだ反応として知覚される.

\vskip\baselineskip
\begin{itemize}
  \item 38
\end{itemize}

これは,「37」の主人公の困惑した態度への返答となるウサギの鳴き声である.

「32」に引き続き,音声素材をピッチの変化を伴わず短く鳴くだけの音として配置しており,主人公の鳴き声とは対照的に,落ち着いたウサギの状態を強調している.

\vskip\baselineskip
\begin{itemize}
  \item 39
\end{itemize}

これは,「38」のウサギに対する返答となる主人公の鳴き声である.

ここでは,主人公の鳴き声をB2からF2へと下降するように配置しており,一定の理解を示しつつも,なお戸惑いを残した反応を表現している(図○参照).

上昇するピッチによる疑問表現とは異なり,下降するピッチを用いることで,主人公が状況を暫定的に受け止めながらも,それを完全には整理しきれていない心理状態が示されている.

\vskip\baselineskip
\begin{itemize}
  \item 40
\end{itemize}

これは,なお戸惑いを残している主人公へのさらなる返答となるウサギの鳴き声である.

「38」ではピッチがB2であったのに対し,ここではD3に設定しており,それまでの返答よりも高いピッチを用いることで,ウサギが肯定的な態度を示している印象を聴者に与えている(図○参照).

また,この一連のウサギの返答は,食べ物をわけ与えられた状況を把握しきれていない主人公に対して,それを食べることを許可する役割を持っている.

\vskip\baselineskip
\begin{itemize}
  \item 41
\end{itemize}

これは,ウサギの返答を受けた主人公が,食べ物の摂取に移る直前に発する鳴き声である.

この後,主人公は実際にウサギからわけられた食べ物を摂取することになるが,この音は短く単一の音として鳴るものの,その次に続く咀嚼音の主体が主人公であることを示す役割を持っている.

\vskip\baselineskip
\begin{itemize}
  \item 42
\end{itemize}

これは,食べ物を実際に食べている主人公の咀嚼音である.

ここでは「ガムを噛む」という音声素材を使用し,咀嚼動作が聴覚的に明確に伝わることを重視している.

また,MIDIノートを周期的に配置せず,不規則な間隔で並べることで,自然な食事動作のリズムを表現している(図○参照).

\vskip\baselineskip
\begin{itemize}
  \item 43
\end{itemize}

これは,主人公が食べ物を飲み込む際の音である.

ここでは,「飲む」という音声素材を使用し,「42」の直後に配置することで,主人公が食べ物を飲み込む動作を明確に示している.

\vskip\baselineskip
\begin{itemize}
  \item 44
\end{itemize}

これは,食事が終わった直後の主人公の鳴き声である.

ここでは,音がそれぞれ瞬間的に F3→E3→D3→C3の順に再生されるようになっており,食事の動作の終了を示すとともに,まだ主人公に疲弊が残っている印象を聴者に与えている(図○参照).

また,この後に雷に怯える場面が続くため,本音によって主人公の疲弊が完全には回復していないことを示しておくことで,後続場面における主人公の心理状態との連続性を保っている.

\vskip\baselineskip
\begin{itemize}
  \item 45
\end{itemize}

これは,食事を終えた主人公の様子を受けたウサギの鳴き声である.

ピッチをE3と,音声素材のもとのピッチであるC3よりも高めに設定することで,ウサギの喜びを含んだ肯定的な反応を示している.

また,この音は,食事が無事に行われたことに対する肯定を,主人公に代わって表現する役割を持っている.

\vskip\baselineskip
\begin{itemize}
  \item 46
\end{itemize}

これは,突然鳴った雷に動揺する主人公の鳴き声である.

C3の音を瞬間的に再生した直後にA3の音を再生することで,雷という突発的な刺激に対し,主人公が反射的に声を上げる様子を示している(図○参照).

また,この場面は第1場面で描かれた主人公の困難を想起させるものであり,主人公の心身がまだ十分に回復していないことを示唆している.

\vskip\baselineskip
\begin{itemize}
  \item 47
\end{itemize}

これは,「46」と同時に再生される落雷の音である.

主人公の鳴き声と同時に配置することで,雷という突発的な現象が,主人公の動揺を引き起こしている状況を明確にしている.

\vskip\baselineskip
\begin{itemize}
  \item 48
\end{itemize}

これは,雷に取り乱した後の主人公の鳴き声である.

短く鳴くだけの音としており,突発的な動揺の後に生じる,一時的な緊張の収束を示している.

ただし,ここでは完全に落ち着いた状態を示すものではなく,その後に続く反応へと移行する過程として位置づけられている.

\vskip\baselineskip
\begin{itemize}
  \item 49
\end{itemize}

これは,主人公の雷への反応に対するウサギの鳴き声である.

これまで同場面において用いられてきたウサギの鳴き声である「38」,「40」,「45」と比べて,最もピッチの低いF2で音声素材を再生しており,ウサギが主人公を心配している印象を聴者に与えている.

一般に,低いピッチの音は高い音に比べて興奮や緊張を抑えた印象として知覚されやすく,本作ではこの傾向を利用して,ウサギの落ち着いた心配の態度を表現している.

\vskip\baselineskip
\begin{itemize}
  \item 50
\end{itemize}

これは,ウサギの返答を受けて,精神的な疲弊が再び示す主人公の鳴き声である.

瞬間的にA2で鳴き声を再生した直後にC2での再生を行うことで,主人公が気力を再び失いつつある様子を聴覚的に表現している.

また,本場面の最後に配置されるこの鳴き声は,激しい落雷の後に残る緊張感の余韻を形成している.

\vskip\baselineskip
subsubsection{第4場面}
\begin{itemize}
  \item 51
\end{itemize}
これは,第3場面の次の日の朝を告げるスズメの鳴き声である.

スズメは日の出前後に鳴き始める行動が観察されており,鳥が活発に声を出すこの時間帯の音が,視聴者に自然な朝の訪れを想起させる効果を持つ.

この音を用いることで,平穏な朝の印象を示し,これまでの激しい雨とは対照的な場面を構築している.

加えて,このスズメの鳴き声は物語の登場人物ではなく環境音として扱っており,朝の屋外空間に鳴き声が広がっている印象を表現するために,音声素材にリバーブをかけている.

\vskip\baselineskip
\begin{itemize}
  \item 52
\end{itemize}

これは,移動中のウサギの鳴き声である.

跳ねながらの移動に伴う着地音である「53」「54」と同じタイミングでこの鳴き声を再生しており,移動の主体がウサギであることを聴覚的に示している.

\vskip\baselineskip
\begin{itemize}
  \item 53,54
\end{itemize}

「18」「19」および「27」「28」と同じく,芝生上を接地する音と,草むら上を接地する音の音声素材を同じタイミングで再生することによって,ウサギが跳ねて着地する様子をより豊かに表現している.

また,今回は視点がウサギ中心のものであり,ウサギが近づいてくる様子や,逆に遠ざかっていく様子の表現はないため,ベロシティは時間経過とともに変化させておらず,一定にしている.

\vskip\baselineskip
\begin{itemize}
  \item 55
\end{itemize}

これは,ウサギに声をかける主人公の鳴き声である.

ここでは,B2で一度鳴き声をやや長めに再生した後,短い間を挟んでG2を再生しており,最初の音を保ち,その後に間を設けることで,これが反射的ではなく落ち着いた反応であることを示している(図◯参照).

このような時間的余裕を持たせた表現により,主人公が動揺から回復しつつあり,比較的穏やかな状態にあることを聴覚的に示している.

また,同場面ではそれまでウサギの存在を示す音のみが配置されていた中で,ここで主人公の鳴き声を加えることで,その空間に主人公の存在があることを明確にしている.

\vskip\baselineskip
\begin{itemize}
  \item 56
\end{itemize}

これは,「55」の主人公への返答となるウサギの鳴き声である.

ピッチを「43」と同じD2としており,ウサギの肯定的な感情を表現している

\vskip\baselineskip
\begin{itemize}
  \item 57
\end{itemize}

これは,この後に登場するシカが現れる前触れとなる草の揺れの音である.

ここでは,「風に揺れる草木2」という音声素材を使用しているが,草が揺れる音を配置することで,何者かの気配があるという印象を聴者に与えている.

\vskip\baselineskip
\begin{itemize}
  \item 58
\end{itemize}

これは,「57」の草の揺れに戸惑いを伴う反応をする主人公の鳴き声である.

「37」と同じく,F2→C♯2→A2の順に主人公の鳴き声の音を瞬間的に再生しており,

一度ピッチを下げてから再び上昇させることで,状況を即座に理解できず,

一瞬立ち止まってから反応している主人公の戸惑いを聴覚的に示している.

\vskip\baselineskip
\begin{itemize}
  \item 59
\end{itemize}

これは,「58」の主人公の反応に対するウサギの鳴き声である.

ピッチをF3に設定することで,状況に対して注意を向けつつも,過度な動揺を示さないウサギの様子を表現している.

また,本鳴き声では,低いピッチの音の後に高いピッチの音を再生するといった操作は行っておらず,これにより,疑問を含みながらも落ち着いた態度を保っているという,

ウサギのこれまでの反応との一貫性を持たせている.

\vskip\baselineskip
\begin{itemize}
  \item 60
\end{itemize}

これは,主人公とウサギのもとに向かってくるシカの足音である.

使用条件に合致するシカの足音の音声素材が確認できなかったため,比較的大型の動物が走っていることを表現する意図のもと,ここでは「馬が走る1」という馬の足音の音声素材を便宜的に使用している.

また,ボリュームを時間経過とともに−20 dBから−2.1 dBへと段階的に上昇させており,これによって,シカが徐々にこちらへ近づいてきている様子を聴覚的に示している(図◯参照).

加えて,もとのC3のピッチでは移動速度が速く感じられたため,それより低いC♯2にピッチを下げて音声を再生している.

\vskip\baselineskip
\begin{itemize}
  \item 61
\end{itemize}
これは,シカの鳴き声である.
ここでは,「baby-deer-calling-mama」という音声素材を使用しており,子鹿の鳴き声であるため,やや高い声のシカの鳴き声となっている.
「57」における草の揺れの音や,「60」の足音によって示されていた存在について,「60」の足音が終わったタイミングでこの鳴き声を再生することで,その足音の主体が誰であるかをここで示している.

\vskip\baselineskip
\begin{itemize}
  \item 62
\end{itemize}

これは,シカに対するウサギの反応となる鳴き声である.

A2の音声を再生した後にF3の音声を再生するようにしており,主人公の鳴き声(例:「37」)で用いてきた表現と同じ要領で,突然現れた存在に対するウサギの困惑を聴覚的に示している(図◯参照).

また,最初の音を瞬間的に再生しないことで,困惑を表現しつつも,これまで示してきたウサギの落ち着いた態度との一貫性も保っている.

\vskip\baselineskip
\begin{itemize}
  \item 63
\end{itemize}

これは,この次の「64」での移動の合図となるシカの鳴き声である.

シカはこの後さらに主人公とウサギに接近するが,事前に鳴き声を配置することで,その際の歩行の主体がシカであることをあらかじめ示している.

また,ここでは「61」と同じ音声素材を使用しているが,別の箇所の鳴き声を再生することで,シカの鳴き声が単調に聴こえることを避けている.

\vskip\baselineskip
\begin{itemize}
  \item 64
\end{itemize}

これは,「60」に引き続いてさらに主人公とウサギのもとに接近するシカの足音である.

ピッチを「60」のA♯2よりさらに下げたC♯2に設定しており,シカが先ほどよりもゆっくりとした速度で接近していることを聴覚的に示している.

また,ボリュームを時間経過とともに−2.1 dBから6 dBへと段階的に上昇させており,「60」と同じく,シカの接近を音量の変化で表現している.

\vskip\baselineskip
\begin{itemize}
  \item 65
\end{itemize}

これは,シカの接近に反応する主人公の鳴き声である.

「48」と同様に,瞬間的に鳴く短い音となっており,驚きを示しつつも,強い動揺には至っていない様子を表現している.

\vskip\baselineskip
\begin{itemize}
  \item 66
\end{itemize}

これは,「65」での主人公と同じタイミングでシカに反応するウサギの鳴き声である.

ピッチをE3と,もとのC3よりやや高めに設定することで,落ち着きを残しつつも,状況をまだ把握しきれていない様子を表現している.

\vskip\baselineskip
\begin{itemize}
  \item 67
\end{itemize}

これは,同時に再生される「69」においてシカに匂いを嗅がれた際に驚く主人公の鳴き声である.

こちらも「65」と同じく短く鳴く音となっており,過度な動揺には至らない驚きを示している様子を表現しているが,聴覚的な単調さを避けるために別のハチドリの音声素材を使用している.

\vskip\baselineskip
\begin{itemize}
  \item 68
\end{itemize}

これは,「67」での主人公と同じタイミングでシカの行動に反応するウサギの鳴き声である.

ピッチを,直前の鳴き声である「66」のE3よりも低めのB2に設定しており,焦りを示していないが,身構えているウサギの様子を聴覚的に示している.

\vskip\baselineskip
\begin{itemize}
  \item 69
\end{itemize}

これは,シカが主人公とウサギの匂いを嗅ぐ音である.

ここでは,「イノシシが鼻をフンフン」という音声素材を使用しているが,シカが匂いを嗅ぐ音の素材が確認できなかったため,嗅ぎ取り動作として知覚される音声の特徴を優先し,イノシシの音を便宜的に使用している.

加えて,最初にC3で音声を再生し,短い間を挟んでA2で再度再生することで,一度匂いを確認した後,より慎重に再確認する動作を表現している.後半ではピッチを下げることで,それまでよりもゆっくりとした嗅ぎ取り動作を想起させ,動物の自然な行動として知覚されるようにしている(図◯参照).

また,この音は「67」および「68」と同時に再生されており,主人公とウサギの反応が,このシカの行動に対する反射的なものであることを示している.

\vskip\baselineskip
\begin{itemize}
  \item 70
\end{itemize}

これは,「69」での匂いを嗅ぐ動作が終わった直後のシカの鳴き声である.

「63」と同じ音声素材を,同じピッチで再生しているが,シカがこれまでと同様の鳴き方をすることで,直前の行動によって生じた緊張が過度に持続せず,危険な存在ではないと知覚される方向へ聴取の印象を誘導している.

また,「63」ではベロシティを63に設定していたが,ここではそれを117に設定しており,シカがより主人公とウサギの近くにいることを聴覚的に示している.

\vskip\baselineskip
\begin{itemize}
  \item 71
\end{itemize}

これは,「70」のシカの鳴き声に対する反応となる主人公の鳴き声である.

前回の主人公の鳴き声である「65」を基準として,「65」ではピッチがE3であったのに対し,ここではA3に設定している.このように前回の音よりもピッチを上げることによって,主人公の警戒がわずかに和らいだ状態であるという印象を聴者に与えている.

\vskip\baselineskip
\begin{itemize}
  \item 72
\end{itemize}

これは,シカが川への道案内をするために,再び主人公とウサギのもとから歩き始める場面の足音である.

「60」と同じくピッチをC♯2に設定しており,ゆっくりとした歩行の様子を表現している.

また,この場面ではシカが物理的に主人公とウサギから距離を取り始めるため,ボリュームを時間経過とともに6.0 dBから−2.0 dBへと減少させており,遠ざかっていく様子を聴覚的に示している(図◯参照).

\section{まとめ}


